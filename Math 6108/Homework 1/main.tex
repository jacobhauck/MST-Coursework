\documentclass{homework}
\newcommand{\R}{\textbf{R}}
\newcommand{\dee}{\;\text{d}}
\newcommand{\eps}{\varepsilon}
\newcommand{\pl}[2]{\frac{\partial #1}{\partial #2}}
\newcommand{\dl}[2]{\frac{\text{d} #1}{\text{d} #2}}
\newcommand{\sgn}{\text{sgn}}
\newcommand{\bigoh}{\mathcal{O}}
\usepackage{enumitem}

\newcommand{\hwclass}{Math 6108}
\newcommand{\hwname}{Jacob Hauck}
\newcommand{\hwtype}{Homework}


\newcommand{\hwnum}{1}
\renewcommand{\questiontype}{Problem}

\begin{document}
	\maketitle
	
	\question If $\vec{u}$ and $\vec{v}$ are orthogonal, unit vectors, then $\vec{u} + \vec{v}$ and $\vec{u} - \vec{v}$ are orthogonal.
	
	\begin{proof}
		Since
		\begin{align*}
			(\vec{u}+\vec{v}, \vec{u} - \vec{v}) &= (\vec{u},\vec{u}) - (\vec{u}, \vec{v}) + (\vec{v},\vec{u}) - (\vec{v},\vec{v})\\
			&= \lVert \vec{u}\rVert^2 - (\vec{u}, \vec{v}) + (\vec{u}, \vec{v}) - \lVert \vec{v}\rVert^2 = 1-1 =0,
		\end{align*}
		it follows that $\vec{u} + \vec{v}$ and $\vec{u} - \vec{v}$ are orthogonal.
	\end{proof}
	
	\question If $P \in \R^{n\times n}$ is a projection, then $I-P$ is projection.
	\begin{proof}
		Since $P$ is a projection, $P^2 = P$. Thus,
		\begin{equation*}
			(I-P)^2 = (I-P)(I-P) = I^2 - IP - PI + P^2 = I - 2P + P = I - P,
		\end{equation*}
		so $I-P$ is also a projection.
	\end{proof}
	
	\question Let $U$ and $V$ be $n\times n$ unitary matrices. Then $UV$ is an $n \times n$ unitary matrix.
	
	\begin{proof}
		It suffices to show that $(UV)^*(UV) = I$. This is the case because
		\begin{equation*}
			(UV)^*(UV) =V^*U^*UV = V^*IV = V^*V = I
		\end{equation*}
		by the unitarity of $U$ and $V$.
	\end{proof}
	
	\question Let $A = \{a_{ij}\}$ and $B = \{b_{ij}\}$ be $n \times n$ matrices. Suppose that $AB = \{c_{ij}\}$, and $BA = \{d_{ij}\}$. Then, by definition,
	\begin{equation*}
		c_{ij} = \sum_{k=1}^n a_{ik}b_{kj}, \qquad d_{ij} = \sum_{k=1}^n b_{ik}a_{kj}.
	\end{equation*} 
	Hence, by the definition of trace,
	\begin{equation*}
		\tr(AB) = \sum_{i=1}^n c_{ii} = \sum_{i=1}^n\sum_{k=1}^n a_{ik}b_{ki}= \sum_{k=1}^n\sum_{i=1}^nb_{ki}a_{ik} = \sum_{k=1}^nd_{kk} = \tr(BA).
	\end{equation*}
	
	\question Let $A \in \C^{n\times n}$ be a matrix whose columns $\{\vec{a}_i\}_{i=1}^n$ form an orthogonal set.
	
	\begin{enumerate}
		\item $A^*A$ is a diagonal matrix.
		\begin{proof}
			Since
			\begin{equation*}
				A^*A = \left[\begin{matrix}
					\vec{a}_1^* \\ \vec{a}_2^* \\ \vdots \\ \vec{a}_n^*
				\end{matrix}\right]
				\left[\begin{matrix}
					\vec{a}_1 & \vec{a}_2 & \cdots & \vec{a}_n
				\end{matrix}\right],
			\end{equation*}
			if $b_{ij}$ is the entry of $A^*A$ in the $i$th row and $j$th column, then $b_{ij} = \vec{a}_i^*\vec{a}_j = (\vec{a}_i, \vec{a}_j) = 0$ if $i \ne j$. Thus, $A^*A$ is diagonal.
		\end{proof}
		
		\item $AA^*$ is not necessarily diagonal.
		\begin{proof}
			Suppose that
			\begin{equation*}
				A = \left[\begin{matrix}
					1 & 2 \\
					-1 & 2
				\end{matrix}\right].
			\end{equation*}
			Then $(\vec{a}_1, \vec{a}_2) = 1\cdot 2 - 1\cdot2 = 0$, so the columns of $A$ form an orthogonal set, but
			\begin{equation*}
				AA^* = \left[\begin{matrix}
					1 & 2 \\
					-1 & 2
				\end{matrix}\right]
				\left[\begin{matrix}
					1 & - 1\\
					2 & 2
				\end{matrix}\right] 
				= \left[\begin{matrix}
					5 & 3 \\
					3 & 5
				\end{matrix}\right],
			\end{equation*}
			which is clearly not diagonal.
		\end{proof}
	\end{enumerate}
	
	\question If a matrix $A \in \R^{n\times n}$ (or $\C^{n\times n}$) has the form
	\begin{equation*}
		A = \mat{
			1 & x_1 & x_1^2 & \dots & x_1^{n-1} \\ 
			1 & x_2 & x_2^2 & \dots & x_2^{n-1} \\
			\vdots & \vdots & \vdots & \ddots & \vdots \\
			1 & x_n & x_n^2 & \dots & x_n^{n-1}
		},
	\end{equation*}
	where $x_1,x_2,\dots,x_n \in \R$ (or $\C$) then $A$ is called a Vandermonde matrix. The determinant of $A$ is given by
	\begin{equation}
		\label{eq:vandermonde_det}
		\det(A) = \prod_{1\le i < j \le n} (x_j - x_i).
	\end{equation}
	\begin{proof}
		We use induction. For the base case\footnote{We can also say that \eqref{eq:vandermonde_det} is true for $n=1$ under the convention that the product over the empty set is $1$, and that a $1\times 1$ Vandermonde matrix is given by $A = \mat{1}$, but this complicates the induction step a bit by generating two cases.}, consider $n = 2$. Then
		\begin{equation*}
			A = \mat{1 & x_1 \\ 1 & x_2}, \qquad x_1, x_2\in \R \text{ (or $\C$)},
		\end{equation*}
		so
		\begin{equation*}
			\det(A) = x_2 -x_1 = \prod_{1 \le i < j \le 2} (x_j - x_i);
		\end{equation*}
		that is, \eqref{eq:vandermonde_det} holds for $n = 2$.
		
		Now suppose for induction that, for some $n \ge 2$, the determinant of any Vandermonde matrix in $\R^{n\times n}$  (or $\C^{n\times n}$) is given by \eqref{eq:vandermonde_det}. Let $A \in \R^{(n+1)\times(n+1)}$ (or $\C^{(n+1)\times(n+1)})$. The determinant of $A$ is preserved by adding a scalar multiple of one column to another. Let $C_i$ denote the $i$th column of $A$. If we perform the following sequence of column operations, which preserve the determinant,
		\begin{align*}
			C_{n+1} &\gets C_{n+1} - x_1C_n, \\
			C_{n} &\gets C_{n} - x_1C_{n-1}, \\
			&\vdots \\
			C_3 &\gets C_3 - x_1C_2, \\
			C_2 &\gets C_2 - x_1C_1, \\
		\end{align*}
		then we find that
		\begin{equation*}
			\det(A) = \det\left(\mat{
				1 & 0 & 0 & 0 & \dots & 0 \\
				1 & x_2 - x_1 & x_2(x_2-x_1) & x_2^2(x_2-x_1) & \dots & x_2^{n-1}(x_2-x_1) \\
				1 & x_3 - x_1 & x_3(x_3-x_1) & x_3^2(x_3-x_1) & \dots & x_3^{n-1}(x_3-x_1) \\
				\vdots & \vdots & \vdots & \vdots & \ddots & \vdots \\
				1 & x_{n+1} - x_1 & x_{n+1}(x_{n+1}-x_1) & x_{n+1}^2(x_{n+1}-x_1) & \dots & x_{n+1}^{n-1}(x_{n+1}-x_1)
			}\right).
		\end{equation*}
		Using the Laplace expansion for the determinant on the first row of the matrix on the right-hand side we get
		\begin{equation*}
			\det(A) = \det\left(\mat{
				x_2 - x_1 & x_2(x_2-x_1) & x_2^2(x_2-x_1) & \dots & x_2^{n-1}(x_2-x_1) \\
				x_3 - x_1 & x_3(x_3-x_1) & x_3^2(x_3-x_1) & \dots & x_3^{n-1}(x_3-x_1) \\
				\vdots & \vdots & \vdots & \ddots & \vdots \\
				x_{n+1} - x_1 & x_{n+1}(x_{n+1}-x_1) & x_{n+1}^2(x_{n+1}-x_1) & \dots & x_{n+1}^{n-1}(x_{n+1}-x_1)
			}\right).
		\end{equation*}
		Factoring $x_{j} - x_1$ out of the $i$th row of the above matrix on the right-hand side, for $j =2,3,\dots,n+1$, and applying the multilinearity of the determinant gives
		\begin{equation*}
			\det(A) = \prod_{j=2}^{n+1} (x_j -x_1)\det\left(\mat{
				1 & x_2 & x_2^2 & \dots & x_2^{n-1} \\ 
				1 & x_3 & x_3^2 & \dots & x_3^{n-1} \\
				\vdots & \vdots & \vdots & \ddots & \vdots \\
				1 & x_n & x_n^2 & \dots & x_n^{n-1}
			}\right).
		\end{equation*}
		The right factor in the product above is the determinant of a Vandermonde matrix in $\R^{n\times n}$ (or $\C^{n\times n})$, so, by the induction hypothesis, it follows that 
		\begin{equation*}
			\det(A) = \prod_{j=2}^{n+1}(x_j-x_1)\prod_{2\le i < j \le n+1}(x_j - x_i) = \prod_{1 \le i < j \le n+1}(x_j - x_i),
		\end{equation*}
		which proves the claim by induction.
	\end{proof}
	
	\question Let $\vec{x}_1, \vec{x}_2, \dots, \vec{x}_k \in \C^n$ be a set of nonzero, orthogonal vectors. Then $\vec{x}_1,\vec{x}_2,\dots, \vec{x}_k$ are linearly independent.
	
	\begin{proof}
		Let $c_1, c_2, \dots, c_k \in \C$ satisfy
		\begin{equation*}
			c_1\vec{x}_1 + c_2\vec{x}_2 + \dots + c_k\vec{x}_k = 0.
		\end{equation*}
		Taking the inner product of both sides with $\vec{x}_i$, for $i = 1,2,\dots,k$, gives
		\begin{equation*}
			c_i(\vec{x}_i, \vec{x}_i) = (0, \vec{x}_i) = 0, \qquad i = 1,2,\dots, k.
		\end{equation*}
		Since $\vec{x}_i \ne 0$ for $i=1,2,\dots,k$, it follows that $(\vec{x}_i, \vec{x}_i) \ne0$, and $c_i = 0$ for $i = 1,2,\dots, k$. Therefore, $\vec{x}_1,\vec{x}_2,\dots,\vec{x}_k$ are linearly independent.
	\end{proof}
	
\end{document}