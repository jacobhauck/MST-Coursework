\documentclass{homework}
\newcommand{\R}{\textbf{R}}
\newcommand{\dee}{\;\text{d}}
\newcommand{\eps}{\varepsilon}
\newcommand{\pl}[2]{\frac{\partial #1}{\partial #2}}
\newcommand{\dl}[2]{\frac{\text{d} #1}{\text{d} #2}}
\newcommand{\sgn}{\text{sgn}}
\newcommand{\bigoh}{\mathcal{O}}
\usepackage{enumitem}

\newcommand{\hwclass}{Math 6108}
\newcommand{\hwname}{Jacob Hauck}
\newcommand{\hwtype}{Homework}


\newcommand{\hwnum}{2}
\renewcommand{\questiontype}{Problem}

\begin{document}
	\maketitle
	
	\question 
	
	Consider the IVP
	\begin{equation}
		\label{eq:ivp}
		y' = f(t,y), \qquad y(0) = a.
	\end{equation}
	Let $k > 0$ be the time step for a numerical scheme to approximate $y'$. Assume that $f$ is $L$-Lipschitz in $y$ for all $t$.
	
	\begin{enumerate}
		\item Consider the scheme
		\begin{equation}
			y^{n+1} = y^n + k f\big(t_{n+1}, y^{n+1}\big).
		\end{equation}
		Suppose that $y(t_n) = y^n$. Using the Taylor expansion of $y$ about $t_n$,
		\begin{equation*}
			y(t_{n+1}) = y(t_n) + ky'(t_n) + R_1(k),
		\end{equation*}
		where the remainder $R_1(k) = \bigoh(k^2)$ as $k\to 0$.
		Further expanding $y'$ about $t_{n+1}$ and using the ODE gives
		\begin{align*}
			y(t_{n+1}) &= y(t_n) + k\left[y'(t_{n+1}) + R_2(k)\right] + R_1(k) \\
			&= y(t_n) + ky'(t_{n+1}) + kR_2(k) + R_1(k) \\
			&= y(t_n) + kf(t_{n+1}, y(t_{n+1})) + kR_2(k) + R_1(k),
		\end{align*}
		where the remainder $R_2(k) = \bigoh(k)$ as $k \to 0$.
		Using the assumption that $y(t_n) = y^n$ and the definition of the scheme, we have
		\begin{align*}
			y(t_{n+1}) &= y^n + kf\big(t_{n+1}, y^{n+1}\big) + k\left[f(t_{n+1},y(t_{n+1})) - f\big(t_{n+1},y^{n+1}\big)\right] + kR_2(k) + R_1(k) \\
			&= y^{n+1} + k\left[f(t_{n+1},y(t_{n+1})) - f\big(t_{n+1},y^{n+1}\big)\right] + kR_2(k) + R_1(k).
		\end{align*}
		Thus,
		\begin{equation*}
			\text{LTE} = \big|y(t_{n+1}) - y^{n+1}\big| = k\left|f(t_{n+1},y(t_{n+1})) - f\big(t_{n+1},y^{n+1}\big) + kR_2(k) + R_1(k)\right|.
		\end{equation*}
		We can easily show that $\text{LTE} \to 0$ as $k\to 0$, that is, that the scheme is consistent.
		
		By the Lipschitz condition on $f$,
		\begin{align*}
			\text{LTE} = \big|y(t_{n+1})-y^{n+1}\big| &\le k\left|f(t_{n+1},y(t_{n+1})) - f\big(t_{n+1},y^{n+1}\big)\right| + |kR_2(k) + R_1(k)|\\
			&\le kL\big|y(t_{n+1})-y^{n+1}\big| + |kR_2(k) + R_1(k)|.
		\end{align*}
		For all $k < \frac{1}{L}$, we have $1-kL > 0$, so
		\begin{equation*}
			\text{LTE} \le \frac{|kR_2(k) + R_1(k)|}{1-kL}, \qquad k < \frac{1}{L}.
		\end{equation*}
		This implies that
		\begin{equation*}
			0 \le \lim_{k\to 0}\text{LTE} \le \lim_{k\to0}\frac{|kR_2(k) + R_1(k)|}{1-kL} = 0
		\end{equation*}
		because $kR_2(k) + R_1(k) \to 0$ as $k \to 0$, and $1-kL \to 1$ as $k\to 0$. That is, $\text{LTE} \to 0$ as $k\to0$, and the scheme is consistent.
		
		\item Consider the scheme
		\begin{equation}
			y^{n+1} = y^{n-1} + 2kf(t_n, y_n).
		\end{equation}
		Suppose that $y(t_{n-1}) = y^{n-1}$, and $y(t_n) = y^n$. Using the Taylor expansion of $y$ about $t_n$ to the left and to the right, we have
		\begin{align*}
			y(t_{n+1}) &= y(t_n) + ky'(t_n) + R_1(k) \\
			y(t_{n-1}) &= y(t_n) - ky'(t_n) + R_2(k),
		\end{align*}
		where the remainders $R_1(k)$ and $R_2(k)$ satisfy $R_1(k) = \bigoh(k^2)$ and $R_2(k) = \bigoh(k^2)$ as $k \to 0$.
		
		By the ODE and the assumptions that $y(t_{n-1}) = y^{n-1}$ and $y(t_n) = y^n$, this implies that
		\begin{align*}
			y(t_{n+1}) - y^{n-1} &= y(t_{n+1}) -y(t_{n-1}) \\
			&= 2ky'(t_n) +R_1(k) - R_2(k) \\
			&= 2kf(t_n, y(t_n)) + R_1(k) - R_2(k) \\
			&= 2kf(t_n, y^n) + R_1(k) - R_2(k).
		\end{align*}
		Therefore, the $\text{LTE}$ is given by
		\begin{equation*}
			\text{LTE} = \big|y^{n+1} - y(t_{n+1})\big| = |R_1(k) - R_2(k)|.
		\end{equation*}
		Since both $R_1(k) \to 0$ and $R_2(k) \to 0$ as $k\to 0$, it follows that $\text{LTE}\to 0$ as $k\to0$. That is, the scheme is consistent.
		
		\item Consider the scheme
		\begin{equation}
			
		\end{equation}
	\end{enumerate}
	
\end{document}