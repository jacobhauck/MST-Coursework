\documentclass{article}
\usepackage[margin=2.5cm]{geometry}
\usepackage{hyperref}
\newcommand{\R}{\textbf{R}}
\newcommand{\dee}{\;\text{d}}
\newcommand{\eps}{\varepsilon}
\newcommand{\pl}[2]{\frac{\partial #1}{\partial #2}}
\newcommand{\dl}[2]{\frac{\text{d} #1}{\text{d} #2}}
\newcommand{\sgn}{\text{sgn}}
\newcommand{\bigoh}{\mathcal{O}}

\title{Math 5604 Final Project}
\author{Jacob Hauck}

\begin{document}
	\maketitle
	
	\section{Introduction}
	
	In this project, we consider the two-dimensional Allen-Cahn equation \cite{yang_2018} with periodic boundary conditions, which is given by
	\begin{align}
		u_t &= \varepsilon^2 \Delta u - F(u),& & (x,y) \in (0,2\pi)^2,\quad t \in (0,T] \\
		u(x,y, 0) &= u_0(x,y),& & (x,y) \in [0,2\pi]^2 \\
		u(x, 0,t) &= u(x, 2\pi, t), & & x \in [0,2\pi],\quad t \in [0,T] \\
		u(0,y,t) &= u(2\pi,y,t), & & y \in [0,2\pi],\quad t \in [0,T]
	\end{align}
	where $\varepsilon > 0$ is a parameter, $F(u) = u^3 - u$, and $u_0(x,y) = 0.05\sin(x)\sin(y)$. 
	
	This equation is a model of a reaction-diffusion system involving a mixture of two substances with different properties, such as oil and water or metals in certain alloys. It is meant to describe the phenomenon of phase separation, which is the process of the substances separating from a well-mixed initial state into coarser ``cells'' (see \href{https://www.youtube.com/watch?v=t1swj0QJUTw}{here} for a visualization). The function $u$ describes the relative density of one of the substances, with $u = 1$ corresponding to 100\% density of one substance and $u=-1$ corresponding to 100\% density of the other substance.
	
	\section{Discretization}
	We approximate $u(x,y,t)$ by the numerical solution $u^n_{i,j}$:
	\begin{equation}
		u(x_i,y_j, t_n) \approx u^n_{i,j},
	\end{equation}
	where we are using a uniform grid of $(M+1)^2$ spatial sample points $\{(x_i,y_j)\}$ defined by $x_i = hi$ for $i = 0,1,\dots, M$ and $y_j = hj$ for $j =0,1,\dots,M$, where the spatial step size $h = \frac{2\pi}{M}$. We use $N$ evenly-spaced time sample points $\{t_n\}$ defined by $t_n = kn$, where the temporal step size $k = \frac{T}{N}$.
	
	We use the 5-point discretization of the Laplace operator, which is given in this case by
	\begin{equation}
		D_5u^n_{i,j} = \frac{u^n_{i+1,j} + u^n_{i-1,j} + u^n_{i,j+1} + u^n_{i,j-1} - 4u^n_{i,j}}{h^2},\qquad i=0,1,2,\dots,M,\;j=0,1,2,\dots, M
	\end{equation}
	where we take $u^n_{-1,j} = u^n_{M,j}$, $u^n_{M+1,j} = u^n_{0,j}$, $u^n_{i,-1} = u^n_{i,M}$, and $u^n_{i,M+1} = u^n_{i,0}$ to account for the periodic boundary conditions.
	
	For the temporal discretization, we compare two different schemes. The first is the convex splitting scheme, given for the Allen-Cahn equation by
	\begin{equation}
		\frac{u^{n+1}_{i,j} - u^n_{i,j}}{k} = \varepsilon^2D_5u^{n+1}_{i,j} + u^n_{i,j} - \left(u^{n+1}_{i,j}\right)^3.
	\end{equation}
	The second is the Crank-Nicolson scheme, which is given by
	\begin{equation}
		\frac{u^{n+1}_{i,j} - u^n_{i,j}}{k} = \varepsilon^2\frac{D_5u^{n+1}_{i,j} + D_5u^n_{i,j}}{2} + \frac{u^{n+1}_{i,j} + u^n_{i,j}}{2} - \frac{\left(u^{n+1}_{i,j}\right)^3 + \left(u^n_{i,j}\right)^3}{2}.
	\end{equation}
	
	\section{Implementation}
	Let $\{a_{i,j}\}$ be a matrix with indices $i,j=1,2,\dots, M+1$. We pack this matrix into a vector $\{Va_m\}$, where $m = 1,2,\dots, (M+1)^2$ by setting
	\begin{equation}
		\label{eq:V}
		Va_{m(i,j)} = a_{i,j}, \qquad  m(i,j) = (i-1)(M+1)+j.
	\end{equation}
	Note that this is well-defined because of the restricting $i,j=1,2,\dots, M+1$. It is easy to see that $V$ is invertible with inverse
	\begin{equation}
		\label{eq:V_inv}
		V^{-1}b_{i,j} = b_{m(i,j)}, \qquad b \in \R^{(M+1)^2}, \; i,j=1,2,\dots, M+1.
	\end{equation}
	Defining the matrix of unknowns $U^n_{i,j} = u^n_{i-1,j-1}$, we can rewrite our two schemes in vector form by applying $V$ to both sides. In order to do put everything in terms of $VU^n$, we must translate the action of $D_5$ into vector terms. Define $D_5'$ so that $D_5'U^n_{i,j} = D_5u^n_{i-1,j-1}$ by setting
	\begin{equation}
		D_5'U^n_{i,j} = \frac{U^n_{i+1,j} + U^n_{i-1,j} + U^n_{i,j+1} + U^n_{i,j-1} - 4U^n_{i,j}}{h^2}, \qquad i,j =1,2,\dots, M+1,
	\end{equation}
	where we take $U^n_{0,j} = U^n_{M+1,j}$, $U^n_{M+2,j} = U^n_{1,j}$, $U^n_{i,0} = U^n_{i,M+1}$, and $U^n_{i,M+2} = U^n_{i,1}$ to match the periodic boundary conditions in $D_5$. Furthermore, if we define $VD_5' = V\circ D_5' \circ V^{-1}$, then the convex-splitting scheme becomes
	\begin{equation*}
		VU^{n+1} - VU^n = k\varepsilon^2VD_5'VU^{n+1} + kVU^n - k(VU^{n+1})^3,
	\end{equation*}
	and the Crank-Nicolson scheme becomes
	\begin{equation*}
		VU^{n+1} - VU^n = \frac{k\varepsilon^2}{2}\left(VD_5'VU^{n+1} + VD_5'VU^n\right) + \frac{k}{2}\left(VU^{n+1} + VU^n\right) - \frac{k}{2}\left((VU^{n+1})^3 + (VU^n)^3\right).
	\end{equation*}
	In order to solve these two nonlinear systems for $VU^{n+1}$, we apply Newton's method using $VU^n$ as the initial guess. For this, we need to define a function $f_n$ whose root is $VU^{n+1}$ and calculate its Jacobian. We can define these functions as follows:
	\begin{equation*}
		f^\text{CS}_n(z) = z - k\varepsilon^2VD_5'z + kz^3 - (1+k)VU^n,  \qquad z \in \R^{(M+1)^2},
	\end{equation*}
	and
	\begin{equation*}
		f^\text{CN}_n(z) = \left(1-\frac{k}{2}\right)z - \frac{k\varepsilon^2}{2}VD_5'z + \frac{k}{2}z^3 - \frac{k\varepsilon^2}{2}VD_5'VU^n -\left(1+\frac{k}{2}\right)VU^n + \frac{k}{2}(VU^n)^3, \qquad z \in \R^{(M+1)^2}.
	\end{equation*}
	Calculating the Jacobians of these functions is straightforward except for the $VD_5'z$ terms:
	\begin{equation*}
		Df^\text{CS}_n(z) = I - k\varepsilon^2D(VD_5'z) + \mathrm{diag}(3kz^2),
	\end{equation*}
	and
	\begin{equation*}
		Df^\text{CN}_n(z) = \left(1-\frac{k}{2}\right)I - \frac{k\varepsilon^2}{2}D(VD_5'z) + \mathrm{diag}\left(\frac{3k\varepsilon^2}{2}z^2\right).
	\end{equation*}
	To evaluate $VD_5'z$, we make use of \eqref{eq:V_inv} and the definition of $VD_5'$:
	\begin{align*}
		VD_5'z_{m(i,j)} &= V(D_5'V^{-1}(z))_{m(i,j)} = (D_5'V^{-1}(z))_{i,j} \\
		&= \frac{V^{-1}z_{i+1,j} + V^{-1}z_{i-1,j} + V^{-1}z_{i,j+1} +V^{-1}z_{i,j-1} - 4V^{-1}z_{i,j}}{h^2} \\
		&= \frac{z_{m(i+1,j)} + z_{m(i-1,j)} + z_{m(i,j+1)} + z_{m(i,j-1)} - 4z_{m(i,j)}}{h^2}
	\end{align*}
	for $i = 1,2,\dots, M+1$, where we account for the periodic boundary conditions by setting $m(0,j) = m(M+1,j)$, $m(M+2,j) = m(1,j)$, $m(i,0) = m(i,M+1)$, and $m(i,M+2) = m(i,1)$.
	
	Now we can compute the Jacobian of $z \mapsto VD_5'z$:
	\begin{equation*}
		D(VD_5'z)_{m(i,j), m(i',j')} = \begin{cases}
			\frac{1}{h^2} & i = i' \text{ and } \big(j = j' + 1 \text{ or } j = j'-1 \text{ or BC}\big) \\
			\frac{1}{h^2} & j = j' \text{ and } \big(i = i' + 1 \text{ or } i = i'-1 \text{ or BC}\big) \\
			-\frac{4}{h^2} & i = i' \text{ and } j = j'
		\end{cases}
	\end{equation*}
	for $i,j,i',j' = 1,2,\dots M + 1$.
	
	\bibliographystyle{plain}
	\bibliography{references}
\end{document}