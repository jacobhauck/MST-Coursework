\documentclass{homework}
\newcommand{\R}{\textbf{R}}
\newcommand{\dee}{\;\text{d}}
\usepackage{enumitem}

\newcommand{\hwclass}{Math 6418}
\newcommand{\hwname}{Jacob Hauck}
\newcommand{\hwtype}{Homework}

\newcommand{\dist}{\mathcal{D}}

\newcommand{\hwnum}{1}
\renewcommand{\questiontype}{Problem}

\begin{document}
	\maketitle
	
	\question 
	
	Consider the IVP
	\begin{equation}
		y' = 3 + e^{-t} - y, \quad t > 0; \qquad y(0) = 1.
	\end{equation}
	
	\begin{arabicparts}
		\questionpart Multiplying both sides by the integrating factor $e^t$ gives
		\begin{equation}
			y'e^t + ye^t = 3e^t + 1.
		\end{equation}
		The left-hand side is $(ye^t)'$, so integrating on both sides gives
		\begin{equation}
			ye^t = 3e^t + t + C,
		\end{equation}
		for some constant $C$, so $y(t) = 3 + (t + C)e^{-t}$. The initial condition $y(0) = 1$ implies that $C = -2$, so
		\begin{equation}
			y(t) = 3 + (t-2)e^{-t}.
		\end{equation}
		
		\questionpart 
	\end{arabicparts}
	
	\question
	
	Consider the IVP
	\begin{equation}
		y' = \frac{3t^2+10t+1}{2(y+1)}, \quad t > 0; \qquad y(0) = -2.
	\end{equation}
	
	\begin{arabicparts}
		\questionpart Multiplying both sides by $2(y+1)$ gives
		\begin{equation}
			2(y+1)(y+1)' = 3t^2+10t+1.
		\end{equation}
		The left-hand side is $\left((y+1)^2\right)'$, so integrating on both sides gives
		\begin{equation}
			(y+1)^2 = t^3+5t^2+t+C
		\end{equation}
		for some constant $C$. The initial condition $y(0) = -2$ implies that $C = 1$. Therefore,
		\begin{equation}
			y(t) = -1 \pm \sqrt{t^3+5t^2+t+1}.
		\end{equation}
		The initial condition forces us to choose a negative sign after taking the square root; thus,
		\begin{equation}
			y(t) = -1 - \sqrt{t^3+5t^2+t+1}.
		\end{equation}
		
	\end{arabicparts}
\end{document}