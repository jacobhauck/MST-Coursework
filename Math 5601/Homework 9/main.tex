\documentclass{homework}
\usepackage{enumitem}

\newcommand{\hwclass}{Math 6108}
\newcommand{\hwname}{Jacob Hauck}
\newcommand{\hwtype}{Homework}

\newcommand{\R}{\textbf{R}}
\newcommand{\dee}{\;\text{d}}
\newcommand{\eps}{\varepsilon}
\newcommand{\pl}[2]{\frac{\partial #1}{\partial #2}}
\newcommand{\dl}[2]{\frac{\text{d} #1}{\text{d} #2}}
\newcommand{\sgn}{\text{sgn}}
\newcommand{\bigoh}{\mathcal{O}}


\newcommand{\hwnum}{9}
\renewcommand{\questiontype}{Problem}

\begin{document}
	\maketitle
	
	\question
	\newcommand{\atwo}{A^{(2)}}
	Let $A$ be a nonsingular matrix, and let $\atwo$ be the matrix from the lecture slides in the second step of Gaussian elimination. Then there exists $s \ge 2$ such that $a_{2s}^{(2)} \ne 0$.
	\begin{proof}
		Suppose on the contrary. By the Gaussian elimination process, we know that $a_{21}^{(2)} = 0$. If there is no $s \ge 2$ such that $a_{2s}^{(2)} \ne 0$, then the whole second row of $\atwo$ is zero. Hence, expanding by cofactors along the second row, we see that the determinant of $\atwo$ is
		\begin{equation}
			\det\left(\atwo\right) = 0\cdot\det(B_1) + 0\cdot\det(B_2) + \cdots +0\cdot\det(B_n) = 0,
		\end{equation}
		where $B_i$ is the cofactor corresponding to $a^{(2)}_{2i}$. Then $\atwo$ is singular.
		
		This is a contradiction because $\atwo$ was obtained from $A$ by elementary row operations, and $A$ was nonsingular, and applying row operations to a nonsingular matrix must result in a nonsingular matrix.
	\end{proof}
	
	\question
	Let $A = \{a_{ij}\}$, and consider the SOR iteration for solving $Ax = b$:
	\begin{equation}
		x^{(k+1)}_i = (1-\sigma)x^{(k)}_i + \frac{\sigma}{a_{ii}}\left[b_i - \sum_{j=1}^{i-1}a_{ij}x^{(k+1)}_j - \sum_{j=i}^na_{ij}x^{(k)}_j\right], \qquad i=1,2,\dots,n.
	\end{equation}
	If $L$ is the lower-triangular part of $A$ and $U$ is the upper-triangular part, and $D$ is the diagonal, so that $A = L + U +D$, then the SOR iteration becomes
	\begin{align}
		x^{(k+1)} &= (1-\sigma)x^{(k)} + \sigma D^{-1}\left[b -Lx^{(k+1)} - Ux^{(k)}\right] \\
		\implies (D + \sigma L)x^{(k+1)} &= ((1-\sigma)D-\sigma U)x^{(k)} + \sigma b \\
		\implies x^{(k+1)} &= \left(L + \frac{1}{\sigma}D\right)^{-1}\left(\frac{1}{\sigma}D - D - U\right)x^{(k)} + \left(L + \frac{1}{\sigma}D\right)^{-1}b \\
		&= -\left(L+\frac{1}{\sigma}D\right)^{-1}\left(D - \frac{1}{\sigma}D +U\right)x^{(k)} +\left(L+\frac{1}{\sigma}D\right)^{-1}b.
	\end{align}
	Define $M = L + \frac{1}{\sigma}D$, and $N = -\left(D-\frac{1}{\sigma}D+U\right)$. Then
	\begin{equation}
		x^{(k+1)} = M^{-1}Nx^{(k)} + M^{-1}b,
	\end{equation}
	and
	\begin{equation}
		A = L + D + U = \left(L+\frac{1}{\sigma}D\right) + \left(D - \frac{1}{\sigma}D + U\right) = M - N.
	\end{equation}
	Therefore, SOR is an iterative method that uses the $M$ and $N$ defined above.
	
	\question
	To program the Jacobi and Gauss-Seidel methods, I split the iterative solver into two parts. In the first part, the matrix $B$ and vector $c$ in the iterative form $x^{(k+1)} = Bx^{(k)} + c$ are computed. In the second part, the iteration is performed. The iteration can be done the same way regardless of how $B$ and $c$ are computed, so it is implemented once in the \verb*|solve_iterative.m| file. The Jacobi and Gauss-Seidel methods calculate $B$ and $c$ from $A$ and $b$ differently. These calculations are in the \verb*|jacobi.m| and \verb*|gauss_seidel.m| files. Here is a copy of the code for convenience.
	\lstinputlisting[language=MATLAB, numbers=left, frame=single, basicstyle=\small\ttfamily, showstringspaces=false]{solve_iterative.m}
	
	\lstinputlisting[language=MATLAB, numbers=left, frame=single, basicstyle=\small\ttfamily, showstringspaces=false]{jacobi.m}
	
	\lstinputlisting[language=MATLAB, numbers=left, frame=single, basicstyle=\small\ttfamily, showstringspaces=false]{gauss_seidel.m}
	
	The solution of the given system is $x = 0$. The Jacobi method converges to $x=0$, but the Gauss-Seidel method does not; instead, it alternates between $x_0$ and $-x_0$. See \verb*|output.txt| for the MATLAB output from the Jacobi and Gauss-Seidel iterations.
\end{document}