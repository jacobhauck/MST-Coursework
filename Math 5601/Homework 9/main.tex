\documentclass{homework}
\usepackage{enumitem}

\newcommand{\hwclass}{Math 6418}
\newcommand{\hwname}{Jacob Hauck}
\newcommand{\hwtype}{Homework}

\newcommand{\dist}{\mathcal{D}}
\newcommand{\R}{\textbf{R}}
\newcommand{\dee}{\;\text{d}}


\newcommand{\hwnum}{9}
\renewcommand{\questiontype}{Problem}

\begin{document}
	\maketitle
	
	\question
	\newcommand{\atwo}{A^{(2)}}
	Let $A$ be a nonsingular matrix, and let $\atwo$ be the matrix from the lecture slides in the second step of Gaussian elimination. Then there exists $s \ge 2$ such that $a_{2s}^{(2)} \ne 0$.
	\begin{proof}
		Suppose on the contrary. By the Gaussian elimination process, we know that $a_{21}^{(2)} = 0$. If there is no $s \ge 2$ such that $a_{2s}^{(2)} \ne 0$, then the whole second row of $\atwo$ is zero. Hence, expanding by cofactors along the second row, we see that the determinant of $\atwo$ is
		\begin{equation}
			\det\left(\atwo\right) = 0\cdot\det(B_1) + 0\cdot\det(B_2) + \cdots +0\cdot\det(B_n) = 0,
		\end{equation}
		where $B_i$ is the cofactor corresponding to $a^{(2)}_{2i}$. Then $\atwo$ is singular.
		
		This is a contradiction because $\atwo$ was obtained from $A$ by elementary row operations, and $A$ was nonsingular, and applying row operations to a nonsingular matrix must result in a nonsingular matrix.
	\end{proof}
	
	\question
\end{document}