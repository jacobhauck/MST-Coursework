\documentclass{homework}
\usepackage{enumitem}

\newcommand{\hwclass}{Math 6418}
\newcommand{\hwname}{Jacob Hauck}
\newcommand{\hwtype}{Homework}

\newcommand{\dist}{\mathcal{D}}

\newcommand{\hwnum}{1}
\renewcommand{\questiontype}{Problem}

\begin{document}
\maketitle

\question
\begin{alphaparts}
	\questionpart See \verb*|bisect.m| -- also copied here for convenience.
	\begin{lstlisting}[language=MATLAB, numbers=left, frame=single, basicstyle=\small\ttfamily]
function result = bisect(f, a, b, epsilon, epsilon_f, max_it)
  for k = 0:max_it
    x_k = (a + b) / 2;
    fk = f(x_k);
    max_error = (b - a) / 2;

    fprintf( ...
      "k = %d, x_k = %.5g, max error = %.5g, f(x_k) = %.5g\n", ...
      k, x_k, max_error, fk ...
    );

    if (b - a) / 2 < epsilon || abs(fk) < epsilon_f
      break;
    elseif f(a) * fk < 0  % root lies in [a, x_k]
      b = x_k;
    else  % if root is not in [a, x_k], it must be in [x_k, b]
      a = x_k;
     	end
  end

  result = x_k;
	\end{lstlisting}
	\questionpart \begin{enumerate}[label=\textbf{(\arabic*)}]
		\item The following is copied from MATLAB output. For $\varepsilon = 10^{-2}$:
		\begin{lstlisting}[basicstyle=\small\ttfamily, frame=single]
>> bisect(@(x) atan(x), -4.9, 5.1, 1e-2, 0, 50)
k = 0, x_k = 0.1, max error = 5, f(x_k) = 0.099669
k = 1, x_k = -2.4, max error = 2.5, f(x_k) = -1.176
k = 2, x_k = -1.15, max error = 1.25, f(x_k) = -0.85505
k = 3, x_k = -0.525, max error = 0.625, f(x_k) = -0.48345
k = 4, x_k = -0.2125, max error = 0.3125, f(x_k) = -0.20939
k = 5, x_k = -0.05625, max error = 0.15625, f(x_k) = -0.056191
k = 6, x_k = 0.021875, max error = 0.078125, f(x_k) = 0.021872
k = 7, x_k = -0.017188, max error = 0.039062, f(x_k) = -0.017186
k = 8, x_k = 0.0023437, max error = 0.019531, f(x_k) = 0.0023437
k = 9, x_k = -0.0074219, max error = 0.0097656, f(x_k) = -0.0074217

ans =

  -0.0074
		\end{lstlisting}
		For $\varepsilon = 10^{-4}$:
		\begin{lstlisting}[basicstyle=\small\ttfamily, frame=single]
>> bisect(@(x) atan(x), -4.9, 5.1, 1e-4, 0, 50)
k = 0, x_k = 0.1, max error = 5, f(x_k) = 0.099669
k = 1, x_k = -2.4, max error = 2.5, f(x_k) = -1.176
k = 2, x_k = -1.15, max error = 1.25, f(x_k) = -0.85505
k = 3, x_k = -0.525, max error = 0.625, f(x_k) = -0.48345
k = 4, x_k = -0.2125, max error = 0.3125, f(x_k) = -0.20939
k = 5, x_k = -0.05625, max error = 0.15625, f(x_k) = -0.056191
k = 6, x_k = 0.021875, max error = 0.078125, f(x_k) = 0.021872
k = 7, x_k = -0.017188, max error = 0.039062, f(x_k) = -0.017186
k = 8, x_k = 0.0023437, max error = 0.019531, f(x_k) = 0.0023437
k = 9, x_k = -0.0074219, max error = 0.0097656, f(x_k) = -0.0074217
k = 10, x_k = -0.0025391, max error = 0.0048828, f(x_k) = -0.0025391
k = 11, x_k = -9.7656e-05, max error = 0.0024414, f(x_k) = -9.7656e-05
k = 12, x_k = 0.001123, max error = 0.0012207, f(x_k) = 0.001123
k = 13, x_k = 0.0005127, max error = 0.00061035, f(x_k) = 0.0005127
k = 14, x_k = 0.00020752, max error = 0.00030518, f(x_k) = 0.00020752
k = 15, x_k = 5.4932e-05, max error = 0.00015259, f(x_k) = 5.4932e-05
k = 16, x_k = -2.1362e-05, max error = 7.6294e-05, f(x_k) = -2.1362e-05

ans =

  -2.1362e-05
		\end{lstlisting}
		For $\varepsilon = 10^{-8}$:
		\begin{lstlisting}[basicstyle=\small\ttfamily, frame=single]
>> bisect(@(x) atan(x), -4.9, 5.1, 1e-8, 0, 50)
k = 0, x_k = 0.1, max error = 5, f(x_k) = 0.099669
k = 1, x_k = -2.4, max error = 2.5, f(x_k) = -1.176
k = 2, x_k = -1.15, max error = 1.25, f(x_k) = -0.85505
k = 3, x_k = -0.525, max error = 0.625, f(x_k) = -0.48345
k = 4, x_k = -0.2125, max error = 0.3125, f(x_k) = -0.20939
k = 5, x_k = -0.05625, max error = 0.15625, f(x_k) = -0.056191
k = 6, x_k = 0.021875, max error = 0.078125, f(x_k) = 0.021872
k = 7, x_k = -0.017188, max error = 0.039062, f(x_k) = -0.017186
k = 8, x_k = 0.0023437, max error = 0.019531, f(x_k) = 0.0023437
k = 9, x_k = -0.0074219, max error = 0.0097656, f(x_k) = -0.0074217
k = 10, x_k = -0.0025391, max error = 0.0048828, f(x_k) = -0.0025391
k = 11, x_k = -9.7656e-05, max error = 0.0024414, f(x_k) = -9.7656e-05
k = 12, x_k = 0.001123, max error = 0.0012207, f(x_k) = 0.001123
k = 13, x_k = 0.0005127, max error = 0.00061035, f(x_k) = 0.0005127
k = 14, x_k = 0.00020752, max error = 0.00030518, f(x_k) = 0.00020752
k = 15, x_k = 5.4932e-05, max error = 0.00015259, f(x_k) = 5.4932e-05
k = 16, x_k = -2.1362e-05, max error = 7.6294e-05, f(x_k) = -2.1362e-05
k = 17, x_k = 1.6785e-05, max error = 3.8147e-05, f(x_k) = 1.6785e-05
k = 18, x_k = -2.2888e-06, max error = 1.9073e-05, f(x_k) = -2.2888e-06
k = 19, x_k = 7.2479e-06, max error = 9.5367e-06, f(x_k) = 7.2479e-06
k = 20, x_k = 2.4796e-06, max error = 4.7684e-06, f(x_k) = 2.4796e-06
k = 21, x_k = 9.5367e-08, max error = 2.3842e-06, f(x_k) = 9.5367e-08
k = 22, x_k = -1.0967e-06, max error = 1.1921e-06, f(x_k) = -1.0967e-06
k = 23, x_k = -5.0068e-07, max error = 5.9605e-07, f(x_k) = -5.0068e-07
k = 24, x_k = -2.0266e-07, max error = 2.9802e-07, f(x_k) = -2.0266e-07
k = 25, x_k = -5.3644e-08, max error = 1.4901e-07, f(x_k) = -5.3644e-08
k = 26, x_k = 2.0862e-08, max error = 7.4506e-08, f(x_k) = 2.0862e-08
k = 27, x_k = -1.6391e-08, max error = 3.7253e-08, f(x_k) = -1.6391e-08
k = 28, x_k = 2.2352e-09, max error = 1.8626e-08, f(x_k) = 2.2352e-09
k = 29, x_k = -7.0781e-09, max error = 9.3132e-09, f(x_k) = -7.0781e-09

ans =

  -7.0781e-09
		\end{lstlisting}
		\item The maximum error $M_k$ after $k$ iterations of the bisection method is given by \begin{equation}
			M_k = \frac{b-a}{2^{k+1}}
		\end{equation}
		To obtain a maximum error less than $\varepsilon > 0$, we need that $k$ satisfies the inequality
		\begin{equation}
			M_k < \varepsilon \iff \frac{b-a}{\varepsilon} < 2^{k+1}
		\end{equation}
		Thus, we need
		\begin{equation}
			\label{eq:bisection_iteration_lower_bound}
			k > \log_2\left(\frac{b-a}{2\varepsilon}\right)
		\end{equation}
		Since $k$ must be an integer, the least number of iterations needed to guarantee an error no greater than $\varepsilon$ is given by the ceiling of the left side of (\ref{eq:bisection_iteration_lower_bound}), that is, the smallest integer greater than LHS(\ref{eq:bisection_iteration_lower_bound}):
		\begin{equation}
			k = \left\lceil \log_2\left(\frac{b-a}{2\varepsilon}\right) \right\rceil
		\end{equation}
		For $[a,b] = [-4.9,5.1]$ and $\varepsilon = 10^{-2}$, this gives $k = \lceil 8.9658 \rceil = 9$; for $\varepsilon = 10^{-4}$, it gives $k = \lceil 15.6096\rceil = 16$; and for $\varepsilon = 10^{-8}$ it gives $k=\lceil 28.8974\rceil = 29$. These are exactly the number of iterations that were executed in the numerical experiments.
	\end{enumerate}
	
\end{alphaparts}

\question

\begin{alphaparts}
	\questionpart See \verb*|fixed.m| -- also copied here for convenience.
	\begin{lstlisting}[language=MATLAB, numbers=left, frame=single, basicstyle=\small\ttfamily]
function result = fixed(g, x0, epsilon, epsilon_f, max_it)
  x_k = x0;
  x_next = g(x_k);
  fprintf("k = 0, x_k = %.5g, error = unknown, f(x_k) = %.5g\n", x_k, x_next);

  for k = 1:max_it
    x_k = x_next;
    x_next = g(x_k);

    fprintf( ...
      "k = %d, x_k = %.5g, error = %.5g, f(x_k) = %.5g\n", ...
      k, x_k, abs(x_next - x_k), x_next ...
    )

    if abs(x_next - x_k) < epsilon || abs(x_next) < epsilon_f
      break;
    end
  end

  result = x_k;

	\end{lstlisting}
	The following outputs are copied from MATLAB. For $x_0 = 5$:
	\begin{lstlisting}[basicstyle=\small\ttfamily, frame=single]
>> fixed(@(x) x - atan(x), 5, 0, 0, 10)
k = 0, x_k = 5, error = unknown, f(x_k) = 3.6266
k = 1, x_k = 3.6266, error = 1.3017, f(x_k) = 2.3249
k = 2, x_k = 2.3249, error = 1.1646, f(x_k) = 1.1603
k = 3, x_k = 1.1603, error = 0.85945, f(x_k) = 0.30082
k = 4, x_k = 0.30082, error = 0.29221, f(x_k) = 0.008611
k = 5, x_k = 0.008611, error = 0.0086108, f(x_k) = 2.1282e-07
k = 6, x_k = 2.1282e-07, error = 2.1282e-07, f(x_k) = 3.2028e-21
k = 7, x_k = 3.2028e-21, error = 3.2028e-21, f(x_k) = 0
k = 8, x_k = 0, error = 0, f(x_k) = 0
k = 9, x_k = 0, error = 0, f(x_k) = 0
k = 10, x_k = 0, error = 0, f(x_k) = 0

ans =

  0
  	\end{lstlisting}
  	For $x_0 = -5$:
  	\begin{lstlisting}[basicstyle=\small\ttfamily, frame=single]
>> fixed(@(x) x - atan(x), -5, 0, 0, 10)
k = 0, x_k = -5, error = unknown, f(x_k) = -3.6266
k = 1, x_k = -3.6266, error = 1.3017, f(x_k) = -2.3249
k = 2, x_k = -2.3249, error = 1.1646, f(x_k) = -1.1603
k = 3, x_k = -1.1603, error = 0.85945, f(x_k) = -0.30082
k = 4, x_k = -0.30082, error = 0.29221, f(x_k) = -0.008611
k = 5, x_k = -0.008611, error = 0.0086108, f(x_k) = -2.1282e-07
k = 6, x_k = -2.1282e-07, error = 2.1282e-07, f(x_k) = -3.2028e-21
k = 7, x_k = -3.2028e-21, error = 3.2028e-21, f(x_k) = 0
k = 8, x_k = 0, error = 0, f(x_k) = 0
k = 9, x_k = 0, error = 0, f(x_k) = 0
k = 10, x_k = 0, error = 0, f(x_k) = 0

ans =

  0
  	\end{lstlisting}
  	For $x_0 = 1$:
  	\begin{lstlisting}[basicstyle=\small\ttfamily, frame=single]
>> fixed(@(x) x - atan(x), 1, 0, 0, 10)
k = 0, x_k = 1, error = unknown, f(x_k) = 0.2146
k = 1, x_k = 0.2146, error = 0.2114, f(x_k) = 0.0032063
k = 2, x_k = 0.0032063, error = 0.0032063, f(x_k) = 1.0987e-08
k = 3, x_k = 1.0987e-08, error = 1.0987e-08, f(x_k) = 0
k = 4, x_k = 0, error = 0, f(x_k) = 0
k = 5, x_k = 0, error = 0, f(x_k) = 0
k = 6, x_k = 0, error = 0, f(x_k) = 0
k = 7, x_k = 0, error = 0, f(x_k) = 0
k = 8, x_k = 0, error = 0, f(x_k) = 0
k = 9, x_k = 0, error = 0, f(x_k) = 0
k = 10, x_k = 0, error = 0, f(x_k) = 0

ans =

  0
  	\end{lstlisting}
  	For $x_0 = -1$:
  	\begin{lstlisting}[basicstyle=\small\ttfamily, frame=single]
>> fixed(@(x) x - atan(x), -1, 0, 0, 10)
k = 0, x_k = -1, error = unknown, f(x_k) = -0.2146
k = 1, x_k = -0.2146, error = 0.2114, f(x_k) = -0.0032063
k = 2, x_k = -0.0032063, error = 0.0032063, f(x_k) = -1.0987e-08
k = 3, x_k = -1.0987e-08, error = 1.0987e-08, f(x_k) = 0
k = 4, x_k = 0, error = 0, f(x_k) = 0
k = 5, x_k = 0, error = 0, f(x_k) = 0
k = 6, x_k = 0, error = 0, f(x_k) = 0
k = 7, x_k = 0, error = 0, f(x_k) = 0
k = 8, x_k = 0, error = 0, f(x_k) = 0
k = 9, x_k = 0, error = 0, f(x_k) = 0
k = 10, x_k = 0, error = 0, f(x_k) = 0

ans =

  0
  	\end{lstlisting}
  	For $x_0 = 0.1$: 
	\begin{lstlisting}[basicstyle=\small\ttfamily, frame=single]
>> fixed(@(x) x - atan(x), 0.1, 0, 0, 10)
k = 0, x_k = 0.1, error = unknown, f(x_k) = 0.00033135
k = 1, x_k = 0.00033135, error = 0.00033135, f(x_k) = 1.2126e-11
k = 2, x_k = 1.2126e-11, error = 1.2126e-11, f(x_k) = 0
k = 3, x_k = 0, error = 0, f(x_k) = 0
k = 4, x_k = 0, error = 0, f(x_k) = 0
k = 5, x_k = 0, error = 0, f(x_k) = 0
k = 6, x_k = 0, error = 0, f(x_k) = 0
k = 7, x_k = 0, error = 0, f(x_k) = 0
k = 8, x_k = 0, error = 0, f(x_k) = 0
k = 9, x_k = 0, error = 0, f(x_k) = 0
k = 10, x_k = 0, error = 0, f(x_k) = 0

ans =

  0
	\end{lstlisting}
	
	\questionpart It appears that the algorithm converges to $0$ from all the initial guesses that I experimented with. The ones that start closer to $0$ are a few iterations ahead of the ones that start farther from $0$.
	
	First, set $G = [-R, R]$, where $R > 0$ is large enough that $x_0 \in [-R, R]$. By the Fundamental Theorem of Calculus, if $x \in G$, then
	\begin{equation}
		|g(x)| = \left|x - \tan^{-1}(x)\right| = \left|\int_0^x\left(1 - \frac{1}{1+t^2}\right)\;\text{d}t\right| \le |x| \le R
	\end{equation}
	Therefore, $g(G) \subseteq G$. Furthermore, $g$ is $L$-Lipschitz on $[-R, R]$ with $L = 1 - \frac{1}{1+R^2} < 1$ because
	\begin{equation}
		g'(x) = 1-\frac{1}{1+x^2} \le 1-\frac{1}{1+R^2}
	\end{equation}
	if $x \in G$. Therefore, $g$ is a contraction on $G$, so the fixed point method must converge for any $x_0 \in G$. Since $G = [-R,R]$, and $R > 0$ was arbitrary, it follows that the fixed point method should converge for all initial guesses.
	
	Second, if the initial guess $x_0$ is farther from the fixed point $z=0$, then the error bound
	\begin{equation}
		|x_k - z| \le \frac{L^k}{1-L}|x_1 - x_0|
	\end{equation}
	is looser as $L$ gets bigger, and we need to choose a bigger $L$ when $x_0$ is farther from $0$ because we need to choose $R$ large enough so that $x_0 \in [-R, R]$ in order for the fixed point theorem to apply with the initial guess $x_0$. The looser bound for $x_0$ farther from $0$ suggests that the algorithm will require more iterations when $x_0$ is farther from $0$.
\end{alphaparts}

\question
First, we need to show that $g(G) \subseteq G$. Note that
\begin{equation}
	g'(x) = \frac{1}{3}\left(x^2 - 2x -\frac{5}{4}\right), \qquad g''(x) = \frac{2}{3}(x - 1)
\end{equation}
The roots of $g'$ are $1 \pm \frac{1}{2}\sqrt{4 + 5} = 1 \pm \frac{3}{2}$, and the only root of $g''$ is $1$. Since $1\pm \frac{3}{2} \notin [0,2]$, the Extreme Value Theorem implies that
\begin{gather}
	\max_{x\in G} g(x) = \max\{g(0), g(2)\} = \max\left\{\frac{4}{3}, \frac{1}{18}\right\}\\
	\min_{x\in G} g(x) = \min\{g(0), g(2)\} = \min\left\{\frac{4}{3},\frac{1}{18}\right\} = \frac{1}{18}
\end{gather}
Therefore, $g(G) \subseteq \left[\frac{1}{18},\frac{4}{3}\right]\subseteq G$. Furthermore, the Extreme Value Theorem also implies that
\begin{gather}
	\max_{x\in G} g'(x) = \max\{g'(0), g'(2), g'(1)\} = \max\left\{-\frac{5}{12},-\frac{9}{12}\right\} = -\frac{5}{12} \\
	\min_{x\in G} g'(x) = \min\{g'(0), g'(2), g'(1)\} = \min\left\{-\frac{5}{12},-\frac{9}{12}\right\} = -\frac{9}{12}
\end{gather}
Therefore, $|g'| \le \frac{9}{12}$ on $G$, so $g$ is $L$-Lipschitz on $G$ with $L = \frac{9}{12} < 1$. By the Contraction Mapping Theorem, there is a unique fixed point $z$ of $g$ on $G$, and for any $x_0 \in G$, the sequence $\{x_k\}_{k=0}^\infty$ defined recursively by $x_{k+1} = g(x_k)$ converges to $z$ as $k \to \infty$.
\end{document}