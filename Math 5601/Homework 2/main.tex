\documentclass{homework}
\usepackage{enumitem}

\newcommand{\hwclass}{Math 6418}
\newcommand{\hwname}{Jacob Hauck}
\newcommand{\hwtype}{Homework}

\newcommand{\dist}{\mathcal{D}}

\newcommand{\hwnum}{2}
\renewcommand{\questiontype}{Problem}

\begin{document}
\maketitle

\question Let $R > 0$, and define $f(x) = 1 - \frac{1}{Rx}$ for $x > 0$. Then clearly $f(x) = 0$ if and only if $x = \frac{1}{R}$, so calculating a zero of $f$ is equivalent to calculating the reciprocal of $R$.

Let $\{x_k\}$ be the sequence of approximate solutions of $f(x) = 0$ obtained by using Newton's method. Then, by definition,
\begin{equation}
	x_{k+1} = x_k - \frac{f(x_k)}{f'(x_k)} = x_k - \left(1-\frac{1}{Rx_k}\right)\cdot Rx_k^2 = x_k - Rx_k^2 + x_k = x_k(2-Rx_k)
\end{equation}

\question 
\end{document}