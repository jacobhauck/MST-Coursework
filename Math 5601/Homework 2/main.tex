\documentclass{homework}
\usepackage{enumitem}

\newcommand{\hwclass}{Math 6108}
\newcommand{\hwname}{Jacob Hauck}
\newcommand{\hwtype}{Homework}


\newcommand{\hwnum}{2}
\renewcommand{\questiontype}{Problem}

\begin{document}
\maketitle

\question Define $f(x) = 1 - \frac{1}{Rx}$. Then $f(x) = 0$ if and only if $x = \frac{1}{R}$, so solving $f(x) = 0$ is equivalent to computing $\frac{1}{R}$. Since $f'(x) = \frac{1}{Rx^2}$, the Newton's method for solving $f(x)=0$ is given by
\begin{equation}
	\begin{aligned}
		x_{k+1} &= x_k - \frac{f(x_k)}{f'(x_k)} = x_k - Rx_k^2\left(1-\frac{1}{Rx_k}\right) = x_k - Rx_k^2 +x_k \\
		&= x_k(2- Rx_k)
	\end{aligned}
\end{equation}

\question

\end{document}