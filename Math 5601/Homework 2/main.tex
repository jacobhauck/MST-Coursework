\documentclass{homework}
\usepackage{enumitem}

\newcommand{\hwclass}{Math 6418}
\newcommand{\hwname}{Jacob Hauck}
\newcommand{\hwtype}{Homework}

\newcommand{\dist}{\mathcal{D}}

\newcommand{\hwnum}{2}
\renewcommand{\questiontype}{Problem}

\usepackage{booktabs}
\usepackage{xcolor}
\usepackage{pifont}
\newcommand{\xmark}{\ding{55}}

\begin{document}
\maketitle

\question Let $R > 0$, and define $f(x) = 1 - \frac{1}{Rx}$ for $x > 0$. Then clearly $f(x) = 0$ if and only if $x = \frac{1}{R}$, so calculating a zero of $f$ is equivalent to calculating the reciprocal of $R$.

Let $\{x_k\}$ be the sequence of approximate solutions of $f(x) = 0$ obtained by using Newton's method. Then, by definition,
\begin{equation}
	x_{k+1} = x_k - \frac{f(x_k)}{f'(x_k)} = x_k - \left(1-\frac{1}{Rx_k}\right)\cdot Rx_k^2 = x_k - Rx_k^2 + x_k = x_k(2-Rx_k)
\end{equation}

\question 
See \verb*|newton.m|; also included here for convenience.
\lstinputlisting[language=MATLAB, numbers=left, frame=single, basicstyle=\small\ttfamily, showstringspaces=false]{newton.m}
The iteration appears to converge (quickly) for some starting values and diverge for others. See Table \ref{table:newton} for a summary of the results. The full outputs from the MATLAB console can be found in \verb*|outputs.txt|. In particular, there seems to be a cutoff $c \approx 1.391$ such that the method converges if $|x_0| < c$ and diverges if $|x_0| > c$.

\newcommand{\conv}{\textcolor{green!50!black}{\checkmark}}
\newcommand{\nconv}{\textcolor{red!50!black}{\xmark}}
\begin{table}[t]
	\centering
	\begin{tabular}{@{}ll@{}}
		\toprule
		$x_0$ & Converged \\
		\midrule
		0.5  & \conv \\ 
		1 & \conv \\
		1.3 & \conv \\
		1.4 & \nconv \\ 
		1.35 & \conv \\ 
		1.375 &  \conv \\
		1.3875 & \conv \\
		1.39375 & \nconv \\
		1.390625 & \conv \\
		1.3921875 & \nconv \\
		\bottomrule
	\end{tabular}
	\caption{Convergence of Newton's method for $f(x) = \tan^{-1}(x)$}
	\label{table:newton}
\end{table}

\question See \verb*|secant.m|; also included here for convenience. Outputs from the MATLAB console can be found in \verb*|outputs.txt|.
\lstinputlisting[language=MATLAB, numbers=left, frame=single, basicstyle=\small\ttfamily, showstringspaces=false]{secant.m}
\end{document}