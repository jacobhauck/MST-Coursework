\documentclass{homework}
\usepackage{enumitem}

\newcommand{\hwclass}{Math 6418}
\newcommand{\hwname}{Jacob Hauck}
\newcommand{\hwtype}{Homework}

\newcommand{\dist}{\mathcal{D}}
\newcommand{\R}{\textbf{R}}
\newcommand{\dee}{\;\text{d}}

\usepackage{float}
\usepackage{booktabs}

\renewcommand{\hwtype}{Final Project}
\newcommand{\hwnum}{}
\renewcommand{\questiontype}{Problem}


\begin{document}
	\maketitle
	
	Consider the following second-order ODE with Dirichlet boundary conditions:
	\begin{align}
		\label{eq:ode}
		\dl{}{x}\left(c(x)\dl{u(x)}{x}\right) &= f(x),\qquad a \le x \le b, \\
		\label{eq:bc}
		u(a) = g_a,\quad u(b) &= g_b.
	\end{align}
	
	\question
	Consider the second-order ODE (\ref{eq:ode}). Multiplying by $v \in H^1([a,b])$ and integrating by parts gives
	\begin{equation}
		\label{eq:ibp}
		\int_a^bfv = c(b)u'(b)v(b) - c(a)u'(a)v(a) - \int_a^b cu'v.
	\end{equation}
	
	\begin{alphaparts}
		\questionpart Suppose we have the boundary conditions
		\begin{equation}
			u'(a) = p_a, \qquad u(b) = g_b.
		\end{equation}
		Equation (\ref{eq:ibp}) still holds, and we can impose the condition $v(b) = 0$ because we already know that $u(b) = p_b$. Since $u'(a) = p_a$, equation (\ref{eq:ibp}) becomes
		\begin{equation}
			\int_a^b fv = -c(a)p_av(a) - \int_a^bcu'v'
		\end{equation}
		for all $v \in H^1([a,b])$ such that $v(b) = 0$, which is our weak formulation of (\ref{eq:ode}) with the given boundary conditions.
		
		\questionpart Suppose we have the boundary conditions
		\begin{equation}
			u'(a) = p_a, \qquad u'(b) + q_bu(b) = p_b.
		\end{equation}
		Equation (\ref{eq:ibp}) still holds. Since $u'(b) = p_b - q_bu(b)$, and $u'(a) = p_a$, we get
		\begin{equation}
			\int_a^bfv = c(b)(p_b - q_bu(b))v(b) - c(a)p_av(a) - \int_a^bcu'v'
		\end{equation}
		for all $v \in H^1([a,b])$, which is our weak formulation of (\ref{eq:ode}) with the given boundary conditions.
		
		\questionpart Suppose we have the boundary conditions
		\begin{equation}
 			u'(a) = p_a,\qquad u'(b) = p_b.
		\end{equation}
		Equation (\ref{eq:ibp}) still holds. Since $u'(a) = p_a$, and $u'(b) = p_b$, we get
		\begin{equation}
			\label{eq:weak_neumann}
			\int_a^bfv = c(b)p_bv(b) - c(a)p_av(a) - \int_a^bcu'v'
		\end{equation}
		for all $v \in H^1([a,b])$, which is our weak formulation of (\ref{eq:ode}) with the given boundary conditions.
		
		We note that solutions of this formulation are not unique. Indeed, if $u \in H^1([a,b])$ satisfies (\ref{eq:weak_neumann}) for all $v \in H^1([a,b])$, then so does $u + \alpha$, where $\alpha \in \R$ is any real number because $(u+\alpha)' = u'$ regardless of what $\alpha$ is, and the weak formulation depends only on $u'$.
	\end{alphaparts}
	
	\question Consider the Poisson equation
	\begin{equation}
		\label{eq:poisson}
		\nabla \cdot (c\nabla u) = f \; \text{in}\; D.
	\end{equation}
	Using integration by parts, we have
	\begin{equation}
		\label{eq:ibp2}
		\int_D fv = \int_D \nabla \cdot(c\nabla u)v = \int_{\partial D} cv \nabla u \cdot n \dee S - \int_D c\nabla u \cdot \nabla v,
	\end{equation}
	where $\text{d}S$ is the surface measure on $\partial D$, and $v \in H^1\left(\overline{D}\right)$.
	
	\begin{alphaparts}
		\questionpart Suppose that we have the boundary condition
		\begin{equation}
			u = g \;\text{on}\;\partial D.
		\end{equation}
		Equation (\ref{eq:ibp2}) still holds. Since we know the value of $u$ on $\partial D$, we can set $v = 0$ on $\partial D$. Then we get
		\begin{equation}
			\int_D fv = - \int_D c\nabla u\cdot \nabla v
		\end{equation}
		for all $v \in H^1\left(\overline{D}\right)$ such that $v = 0$ on $\partial D$, which is our weak formulation of (\ref{eq:poisson}) with the given boundary condition.
		
		\questionpart Suppose that we have the boundary condition
		\begin{equation}
			\nabla u\cdot n + qu = p \;\text{on}\;\partial D,
		\end{equation}
		where $n$ is the outward unit normal vector to $\partial D$, and $p$ and $q$ are functions on $\partial D$. Equation (\ref{eq:ibp2}) still holds. Since $\nabla u \cdot n = p - qu$ on $\partial D$, it follows that
		\begin{equation}
			\int_D fv = \int_{\partial D} cv(p-qu)\dee S - \int_D c\nabla u\cdot\nabla v
		\end{equation}
		for all $v \in H^1\left(\overline{D}\right)$, which is our weak formulation of (\ref{eq:poisson}) with the given boundary condition.
	\end{alphaparts}
	
	\question
	If $u \in C^2[a,b]$, then
	\begin{align}
		\label{eq:lin_interp_value}
		\lVert u - I_hu\rVert_\infty &\le \frac{1}{8}h^2\lVert u'' \rVert_\infty,\\
		\label{eq:lin_interp_deriv}
		\lVert (u - I_hu)'\rVert_\infty &\le \frac{1}{2}h\lVert u''\rVert_\infty.
	\end{align}
	\begin{proof}
		Consider the interval $[x_i, x_{i+1}]$, where $1 \le i \le N$. Restricted to this interval, $I_hu$ is the degree-1 Lagrange polynomial interpolation of $u$ on with nodes $x_i$ and $x_{i+1}$. By the error formula for Lagrange polynomial approximation in the slides,
		\begin{equation}
			u(x) - I_hu(x) = \frac{f''(\xi(x))(x-x_i)(x-x_{i+1})}{2}
		\end{equation}
		for some $\xi(x) \in [x_i,x_{i+1}]$. Then
		\begin{equation}
			|u(x) - I_hu(x)| \le \lVert f''\rVert_\infty \cdot \frac{1}{2}(x-x_i)(x_{i+1}-x).
		\end{equation}
		The function $g(x) = (x-x_i)(x_{i+1} -x)$ is a downward-opening parabola, so it achieves maximum halfway between its roots $x_i$ and $x_{i+1}$. Therefore,
		\begin{align}
			|u(x) - I_hu(x)| &\le \lVert f'' \rVert_\infty \cdot\frac{\left(\frac{x_i + x_{i+1}}{2} - x_i\right)\left(x_{i+1}-\frac{x_i + x_{i+1}}{2}\right)}{2} \\
			&=  \lVert f''\rVert_\infty \frac{(x_{i+1} - x_i)^2}{8} = \frac{h^2}{8}\lVert f''\rVert_\infty.
		\end{align}
		Since this holds for all $x \in [x_i, x_{i+1}]$ and all $1\le i \le N$, it holds for all $x \in [a,b]$. Therefore, the inequality (\ref{eq:lin_interp_value}) follows.
		
		Let $1 \le i \le N$, and let $x \in [x_i, x_{i+1}]$. By the Mean Value Theorem, there exists $ c\in [x_i, x_{i+1}]$ such that
		\begin{equation}
			u'(c) = \frac{u(x_{i+1}) - u(x_i)}{x_{i+1} - x_i} = (I_hu)'(x).
		\end{equation}
		Then $(I_hu)'$ is the degree-0 Lagrange polynomial interpolation of $u'$ with node $c$. By the error formula for Lagrange polynomial approximation in the slides,
		\begin{equation}
			u'(x) - (I_hu)'(x) = u''(\xi(x))(x-c)
		\end{equation}
	\end{proof}
\end{document}