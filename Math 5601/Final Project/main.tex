\documentclass{homework}
\usepackage{enumitem}

\newcommand{\hwclass}{Math 6108}
\newcommand{\hwname}{Jacob Hauck}
\newcommand{\hwtype}{Homework}

\newcommand{\R}{\textbf{R}}
\newcommand{\dee}{\;\text{d}}
\newcommand{\eps}{\varepsilon}
\newcommand{\pl}[2]{\frac{\partial #1}{\partial #2}}
\newcommand{\dl}[2]{\frac{\text{d} #1}{\text{d} #2}}
\newcommand{\sgn}{\text{sgn}}
\newcommand{\bigoh}{\mathcal{O}}

\usepackage{float}
\usepackage{booktabs}

\renewcommand{\hwtype}{Final Project}
\newcommand{\hwnum}{}
\renewcommand{\questiontype}{Problem}


\begin{document}
	\maketitle
	
	Consider the following second-order ODE with Dirichlet boundary conditions:
	\begin{align}
		\label{eq:ode}
		\dl{}{x}\left(c(x)\dl{u(x)}{x}\right) &= f(x),\qquad a \le x \le b, \\
		\label{eq:bc}
		u(a) = g_a,\quad u(b) &= g_b.
	\end{align}
	
	\question
	Consider the second-order ODE (\ref{eq:ode}).
	
	\begin{alphaparts}
		\questionpart Suppose we have the boundary conditions
		\begin{equation}
			u'(a) = p_a, \qquad u(b) = g_b.
		\end{equation}
		Multiplying the ODE by a test function $v$ and integrating by parts, we have
		\begin{equation}
			\int_a^b f(x)v(x)\dee x = \int_a^b \dl{}{x}\left(c(x)\dl{u(x)}{x}\right) v(x)\dee x = c(x)u'(x)v(x)\Big\vert_a^b - \int_a^b c(x)u'(x)v'(x)\dee x.
		\end{equation}
		If we choose $v(b) = 0$, then we are led to the weak formulation: find $u \in C^1([a,b])$ with $u(b) = g_b$ such that 
		\begin{equation}
			-c(a)p_av(a) - \int_a^b c(x)u'(x)v'(x)\dee x = \int_a^b f(x)v(x)\dee x
		\end{equation}
		for all $v \in C^1([a,b])$ with $v(b) = 0$.
		
		\questionpart Suppose we have the boundary conditions
		\begin{equation}
			u'(a) = p_a, \qquad u'(b) + q_b u(b) = p_b.
		\end{equation}
		Multiplying the ODE by a test function $v$ and integrating by parts, we have
		\begin{equation}
			\int_a^b f(x)v(x)\dee x = \int_a^b \dl{}{x}\left(c(x)\dl{u(x)}{x}\right) v(x)\dee x = c(x)u'(x)v(x)\Big\vert_a^b - \int_a^b c(x)u'(x)v'(x)\dee x.
		\end{equation}
	\end{alphaparts}
	
	\question
	
	\question
	If $u \in C^2[a,b]$, then
	\begin{align}
		\lVert u - I_hu\rVert_\infty &\le \frac{1}{8}h^2\lVert u'' \rVert_\infty,\\
		\lVert (u - I_hu)'\rVert_\infty &\le \frac{1}{2}h\lVert u''\rVert_\infty.
	\end{align}
	\begin{proof}
		Consider the interior $(x_i, x_{i+1})$ of the $i$th element of $[a,b]$, where $1 \le i\le N$. For $x \in (x_i, x_{i+1})$, Taylor's Theorem implies that there is some $\xi \in (x_i, x_{i+1})$ such that
		\begin{equation}
			u(x) = u(x_i) + (x-x_i)u'(x_i) + \frac{1}{2}(x-x_i)^2u''(\xi).
		\end{equation}
	\end{proof}
\end{document}