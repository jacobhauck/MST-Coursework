\documentclass{homework}
\usepackage{enumitem}

\newcommand{\hwclass}{Math 6108}
\newcommand{\hwname}{Jacob Hauck}
\newcommand{\hwtype}{Homework}

\newcommand{\R}{\textbf{R}}
\newcommand{\dee}{\;\text{d}}
\newcommand{\eps}{\varepsilon}
\newcommand{\pl}[2]{\frac{\partial #1}{\partial #2}}
\newcommand{\dl}[2]{\frac{\text{d} #1}{\text{d} #2}}
\newcommand{\sgn}{\text{sgn}}
\newcommand{\bigoh}{\mathcal{O}}

\newcommand{\hwnum}{5}

\begin{document}
	\maketitle
	
	\question Let $S = \{(t,y) \in \R^2 \mid |t| < 11\}$. Define
	\begin{equation}
		f(t,y) = \frac{t}{9}\cos(2y) + t^2.
	\end{equation}
	Then $f$ is differentiable with respect to $y$, with
	\begin{equation}
		\frac{\partial f}{\partial y} = -\frac{2t}{9}\sin(2y)
	\end{equation}
	If $|t| < 11$, then $\left|\frac{\partial f}{\partial y}\right| <\frac{22}{9}$ for all $y \in \R$. This implies that $f$ is $\frac{22}{9}$-Lipschitz in $y$ over $S$, so, by Theorem (I) in the notes, the IVP
	\begin{equation}
		\label{eq:ivp}
		y' = f(t,y), \qquad y(0) = 1
	\end{equation}
	has a unique solution defined for $|t| < 11$. Then (\ref{eq:ivp}) certainly has a unique solution defined for $|t| \le 10$.
	
	\question 
	\begin{alphaparts}
		\questionpart The following MATLAB code (copied from \verb*|forward_euler.m|) implements the forward Euler method
		\begin{equation}
			y_{j+1} = y_j + hf(t_j, y_j),\qquad y_0 = y_a
		\end{equation}
		for the IVP $y' = f(t,y)$ on $[a,b]$ with $y(a) = y_a$, where $t_j = a+ jh$:
		\lstinputlisting[language=MATLAB, numbers=left, frame=single, basicstyle=\small\ttfamily, showstringspaces=false]{forward_euler.m}
		
		\questionpart The following code can be entered in the Command Window in MATLAB to use the above script to solve the IVP $y' = y^\frac{1}{3}$ on $[0,2]$ with $y(0) = 0$:
		\begin{lstlisting}[language=MATLAB, numbers=left, frame=single, basicstyle=\small\ttfamily, showstringspaces=false]
forward_euler(@(t, y) y^(1/3), 0, 2, 0, h)
		\end{lstlisting}
		Running this command with various values of the parameter \verb*|h| (I tried $.1$, $.01$, $.001$, and $.0001$) gives the same output, so I will just describe it rather than copy it: $y_j = 0$ for all $j$. This is, indeed, the exact values $y(t_j)$ for the solution $y(t) = y_1(t) = 0$ of the IVP. There is, however, \textit{another} solution of the IVP:
		\begin{equation}
			y(t) = y_2(t) = \left(\frac{2}{3}t\right)^\frac{3}{2},
		\end{equation}
		which is in fact a solution because $y_2(0) = 0$, and $y_2'(t) = \frac{3}{2}\left(\frac{2}{3}\right)^\frac{3}{2}t^\frac{1}{2} = \left(\frac{2}{3}t\right)^\frac{1}{2} = [y_2(t)]^\frac{1}{3}$ for $t \in [0,2]$.
		
		It is evident from the definition of the forward Euler method that it will never be able to approximate this solution no matter what value of $h$ is chosen. Indeed, we have $y_0 = y_a = 0$, and if $y_j = 0$, then
		\begin{equation}
			y_{j+1} = y_j + hf(t_j, y_j) = 0 + h\cdot 0^\frac{1}{3} = 0,
		\end{equation}
		so by mathematical induction it follows that $y_j = 0$ for all $j$, which is precisely what we observed in the numerical experiments (of course, one could argue that floating-point round-off error \textit{might} cause the stored values $\{\tilde{y}_j\}$ to differ slightly from $0$, ruining the above explanation; however, in standard floating-point representations, $0$ can be represented exactly, and floating-point arithmetic standards ensure that multiplication by exactly $0$ is exactly $0$, and addition of exactly $0$ is exactly the same as before, so the above argument still works).
	\end{alphaparts}
	
	\question Consider the forward Euler method
	\begin{equation}
		y_{j+1}^h = y_{j}^h + f(x_j, y_{j}^h), \qquad h > 0,\quad j \in \{0,1,2,\dots, N\}
	\end{equation}
	for approximating the solution $y(x)$ of $y' = f(x,y)$ with $y(0) = \alpha$. Suppose that
	\begin{equation}
		y_j^h - y(x_j) = \sum_{m=1}^\infty c_mh^m
	\end{equation}
	for some $\{c_m\}$ independent of $h$. To find a third-order approximation $z_j^h$ of $y(x_j)$ using $y_j^h$, $y_j^{\frac{h}{2}}$, and $y_j^{\frac{h}{3}}$, we take a linear combination of them and attempt to find coefficients that make the combination a third-order approximation. To this end, let
	\begin{equation}
		z_j^h = a_1y_j^h + a_2y_j^\frac{h}{2} + a_3 y_j^{\frac{h}{3}}.
	\end{equation}
	Consider the difference $z_j - y(x_j)$:
	\begin{align}
		z_j^h - y(x_j) &= (a_1 + a_2+a_3 - 1)y(x_j) + \sum_{n=1}^3 a_n \left(y_j^\frac{h}{n} - y(x_j)\right) \\
		&= (a_1+a_2+a_3-1)y(x_j) + \sum_{n=1}^3 a_n\sum_{m=1}^\infty c_m\left(\frac{h}{n}\right)^m \\
		&= (a_1 + a_2+a_3-1)y(x_j) + \sum_{m=1}^\infty \left(\sum_{n=1}^3 \frac{a_n}{n^m}\right)h^m \\
		&= \left(-1  + \sum_{n=1}^3a_n\right)y(x_j) +\left(\sum_{n=1}^3\frac{a_n}{n}\right)h + \left(\sum_{n=1}^3\frac{a_n}{n^2}\right)h^2 + O(h^3).
	\end{align}
	Evidently, $z_j^h$ will be a third-order approximation if we choose $a_1$, $a_2$, and $a_3$ such that the first three terms in the last line above are all zero. That is, we must choose the coefficients to satisfy
	\begin{align}
		\left[\begin{matrix}
			1 & 1 & 1 \\
			1 & \frac{1}{2} & \frac{1}{3} \\
			1 & \frac{1}{4} & \frac{1}{9}
		\end{matrix}\right]\left[\begin{matrix}
		a_1 \\ a_2 \\ a_3
		\end{matrix}\right] = \left[\begin{matrix}
		1 \\ 0 \\ 0
		\end{matrix}\right].
	\end{align}
	This implies that
	\begin{equation}
		\left[\begin{matrix}
			18 & 12 \\
			9 & 4
		\end{matrix}\right]\left[\begin{matrix}
		a_2 \\ a_3
		\end{matrix}\right] = -36a_1\left[\begin{matrix}
			1 \\ 1
		\end{matrix}\right],
	\end{equation}
	which gives
	\begin{equation}
		\left[\begin{matrix}
			a_2 \\ a_3
		\end{matrix}\right] = -\frac{1}{36}\cdot (-36a_1)\cdot\left[\begin{matrix}
			4 & -12 \\
			-9 & 18
		\end{matrix}\right]\cdot\left[\begin{matrix}
			1 \\ 1
		\end{matrix}\right] = a_1\left[\begin{matrix}
			-8\\9
		\end{matrix}\right],
	\end{equation}
	or $a_2 = -8 a_1$, and $a_3 = 9a_1$. Since we must have $a_1 + a_2 + a_3 =1$, it follows that $a_1 - 8a_1 + 9a_1 = 1$, so $a_1 = \frac{1}{2}$, $a_2 = -4$, and $a_3 = \frac{9}{2}$. Therefore,
	\begin{equation}
		z_j^h = \frac{1}{2}y_j^h - 4y_j^{\frac{h}{2}} + \frac{9}{2}y_j^\frac{h}{3}
	\end{equation}
	is a third-order approximation of $y(x_j)$.
\end{document}