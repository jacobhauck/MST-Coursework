\documentclass{homework}
\usepackage{enumitem}

\newcommand{\hwclass}{Math 6418}
\newcommand{\hwname}{Jacob Hauck}
\newcommand{\hwtype}{Homework}

\newcommand{\dist}{\mathcal{D}}
\newcommand{\R}{\textbf{R}}
\newcommand{\dee}{\;\text{d}}


\newcommand{\hwnum}{7}
\renewcommand{\questiontype}{Problem}

\begin{document}
	\maketitle
	
	\question
	Let $x_0, x_1, x_2$ and $w_0, w_1, w_2$ be the nodes and weights of the the three-point Gaussian quadrature for $\int_{-1}^1 f(x)\dee x$. Then the quadrature must be exact for $f(x)= x^n$, $n\in\{0,1,2,3,4,5\}$. That is,
	\begin{equation}
		\int_{-1}^1 x^n \dee x = \sum_{j=0}^2 w_j x_j^n, \qquad n\in\{0,1,2,3,4,5\}.
	\end{equation}
	Since
	\begin{equation}
		\int_{-1}^1 x^n \dee x = \left.\frac{x^{n+1}}{n+1}\right\vert_{-1}^1 = \begin{cases}
			\frac{2}{n+1} & n \text{ even} \\
			0 & n \text{ odd},
		\end{cases}
	\end{equation}
	we obtain the following system of six equations in the six unknowns $x_0, x_1, x_2$ and $w_0, w_1, w_2$:
	\begin{alignat*}{2}
		2 &= w_0 + w_1 + w_2 &\qquad 0 & = w_0x_0 + w_1x_1 + w_2x_2 \\[0.3em]
		\frac{2}{3} &= w_0x_0^2 + w_1x_1^2 + w_2x_2^2 &\qquad 0 & = w_0x_0^3 + w_1x_1^3 + w_2x_2^3 \\[0.3em]
		\frac{2}{5} &= w_0x_0^4 + w_1x_1^4 + w_2x_2^4 &\qquad 0 & = w_0x_0^5 + w_1x_1^5 + w_2x_2^5.
	\end{alignat*}
	Using the following \verb*|solve| command in MATLAB gives the solution of this nonlinear system of equations. Note that the system is symmetric with respect to permutation of the index $j \in \{0,1,2\}$. Therefore, MATLAB returns $o(S_3) = 3! = 6$ solutions. Since the quadrature is also symmetric with respect to permutations of the index $j$, each solution results in the same quadrature, so we just use the first one returned by MATLAB.
	\lstinputlisting[language=MATLAB, numbers=left, frame=single, basicstyle=\small\ttfamily, showstringspaces=false]{output.txt}
	
	Thus, we get $x_0 = \frac{\sqrt{15}}{5}$, $x_1 = -\frac{\sqrt{15}}{5}$, $x_2 = 0$, and $w_0 = w_1 = \frac{5}{9}$, $w_2 = \frac{8}{9}$.
	
	\question Let $x_0$, $x_1$, $x_2$ and $w_0$, $w_1$, $w_2$ be the same as in the previous problem. Let $u_4(x)$ be a polynomial of degree $3$ on $[-1,1]$ that is orthogonal to $\mathrm{span}\{1, x,x^2\}$. Then $x_0$, $x_1$, and $x_2$ are the roots of $u_4$. We can find such a polynomial using the Gram-Schmidt process on $\{1,x,x^2,x^3\}$.
	
	Let $u_1(x) = 1$. Note that $(u_1, u_1) = 2$, and for any continuous function $f$, $(f, u_1) = \int_{-1}^1 f(x)\dee x$. By the Gram-Schmidt process, we obtain $u_2(x)$ orthogonal to $u_1(x)$ via
	\begin{equation}
		u_2(x) = x - \frac{(x, u_1)}{(u_1, u_1)}u_1(x) = x
	\end{equation}
	because $(x, u_1) = \int_{-1}^1x\dee x = 0$. Next, we can find $u_2$ orthogonal to both $u_1$ and $u_2$ via
	\begin{equation}
		u_3(x) = x^2 - \frac{(x^2, u_2)}{(u_2, u_2)}u_2(x) - \frac{(x^2, u_1)}{(u_1, u_1)}u_1(x).
	\end{equation}
	The last term is just the constant function $\frac{1}{3}$. As for the second term, note that
	\begin{equation}
		(u_2, u_2) = \int_{-1}^1 x^2\dee x = \frac{2}{3}, \qquad (x^2, u_2) = \int_{-1}^1x^3\dee x = 0,
	\end{equation}
	so $u_3(x) = x^2 - \frac{1}{3}$. Lastly, to obtain $u_4(x)$ of degree 3 and orthogonal to $\mathrm{span}\{1,x,x^2\}$, we use
	\begin{equation}
		u_4(x) = x^3 - \frac{(x^3, u_3)}{(u_3, u_3)}u_3(x) - \frac{(x^3, u_2)}{(u_2, u_2)}u_2(x) - \frac{(x^3, u_1)}{(u_1,u_1)}u_1(x).
	\end{equation}
	Since $x^3$ is odd, the last term is 0. Since
	\begin{equation}
		(x^3, u_2) = \int_{-1}^1 x^4\dee x = \frac{2}{5},
	\end{equation}
	the second term is $\frac{3}{5}x$ (after dividing by the value of $(u_2,u_2)$ from above). Lastly, since $u_3(x)$ is even, $x^3u_3(x)$ is odd, so $(x^3, u_3) = 0$. This gives
	\begin{equation}
		u_4(x) = x^3 - \frac{3}{5}x.
	\end{equation}
	The roots of $u_4$, and the nodes of the Gaussian quadrature with three points on $[-1,1]$, are clearly $x_0 = \sqrt{\frac{3}{5}} = \frac{\sqrt{15}}{5}$, $x_1 = -\sqrt{\frac{3}{5}} = -\frac{\sqrt{15}}{5}$, and $x_2 = 0$, the same as we got in Problem 1.
	
	To obtain the weights, we can now integrate the Lagrange basis polynomials for interpolation at the points $x_0$, $x_1$ and $x_2$. That is,
	\begin{equation}
		w_0 = \int_{-1}^1 L_0(x)\dee x = \int_{-1}^1 \frac{(x-x_1)(x-x_2)}{(x_0 - x_1)(x_0 - x_2)}\dee x = \frac{5}{6}\left[\frac{x^3}{3} - \frac{(x_1+x_2)x^2}{2} + x_1x_2x\right]_{-1}^1 = \frac{5}{9},
	\end{equation}
	and
	\begin{equation}
		w_1 =\int_{-1}^1 L_1(x)\dee x = \int_{-1}^1 \frac{(x-x_0)(x-x_2)}{(x_1 - x_0)(x_1 - x_2)}\dee x = \frac{5}{6}\left[\frac{x^3}{3} - \frac{(x_0 + x_2)x^2}{2} + x_0x_2x\right]_{-1}^1 = \frac{5}{9},
	\end{equation}
	and
	\begin{align}
		w_2 &= \int_{-1}^1 L_2(x) \dee x = \int_{-1}^1\frac{(x-x_0)(x-x_1)}{(x_2-x_0)(x_2-x_1)}\dee x = -\frac{5}{3}\left[\frac{x^3}{3} - \frac{(x_0+x_1)x^2}{2} + x_0x_1x\right]_{-1}^1 \\
		&=\frac{5}{3}\cdot\left(\frac{6}{5} - \frac{2}{3}\right) = 2 - \frac{10}{9} = \frac{8}{9}.
	\end{align}
	These are the same weights that we obtained in Problem 1.
	
	\question
	Let $I(h)$ be an approximation of $\int_a^bf(x)\dee x$ depending on a parameter $h$ such that the error satisfies
	\begin{equation}
		\label{eq:error_estimate}
		I(h) - \int_a^bf(x)\dee x = c_1h + c_2h^2 +\bigoh(h^3)
	\end{equation}
	for some constants $c_1$ and $c_2$. If $I(h)$, $I\left(\frac{h}{2}\right)$, and $I\left(\frac{h}{3}\right)$ are known, then we can use a linear combination to obtain a third-order ($\bigoh(h^3)$) approximation of the integral:
	\begin{equation}
		Q(h) = a_1I(h) + a_2I\left(\frac{h}{2}\right) + a_3I\left(\frac{h}{3}\right).
	\end{equation}
	Now we just need to determine what $a_1$, $a_2$ and $a_3$ should be so that
	\begin{equation}
		Q(h) - \int_a^bf(x)\dee x = \bigoh(h^3).
	\end{equation}
	By (\ref{eq:error_estimate}), we have
	\begin{align*}
		Q(h) - \int_a^bf(x)\dee x &= a_1I(h) + a_2I\left(\frac{h}{2}\right) + a_3I\left(\frac{h}{3}\right) - \int_a^bf(x)\dee x \\
		&= a_1\left[I(h) - \int_a^bf(x)\dee x\right] + a_2\left[I\left(\frac{h}{2}\right) - \int_a^bf(x)\dee x\right] + a_3\left[I\left(\frac{h}{3}\right) - \int_a^bf(x)\dee x\right] \\
		&\quad- (1 - a_1 - a_2 - a_3)\int_a^bf(x)\dee x \\
		&=a_1(c_1h + c_2h^2 + \bigoh(h^3)) + a_2\left(\frac{c_1h}{2} + \frac{c_2h^2}{4} + \bigoh(h^3)\right) + a_3\left(\frac{c_1h}{3} + \frac{c_2h^2}{9} + \bigoh(h^3)\right) \\
		&\quad -(1-a_1-a_2-a_3)\int_a^bf(x)\dee x\\
		&= (a_1 + a_2 + a_3 -1)\int_a^bf(x)\dee x + \left(a_1 + \frac{a_2}{2} + \frac{a_3}{3}\right)c_1h + \left(a_1 + \frac{a_2}{4} + \frac{a_3}{9}\right)c_2h^2 + \bigoh(h^3).
	\end{align*}
	Thus, the error between $Q(h)$ and the integral is $\bigoh(h^3)$ as long as $a_1$, $a_2$, and $a_3$ are chosen such that
	\begin{align}
		1 &= a_1 + a_2 + a_3, \\
		0 &= a_1 + \frac{1}{2}a_2 + \frac{1}{3}a_3, \\
		0 &= a_1 + \frac{1}{4}a_2 + \frac{1}{9}a_3.
	\end{align}
	Substituting $a_1 = 1 - a_2 - a_3$ from the first equation into the last two, we get the system of equations
	\begin{align}
		1 &= \frac{1}{2}a_2 + \frac{2}{3}a_3, \\
		1 &= \frac{3}{4}a_2 + \frac{8}{9}a_3.
	\end{align}
	Therefore, $\frac{1}{4}a_2 = -\frac{2}{9}a_3$, so $a_2 =-\frac{8}{9}a_3$. Then $a_3 =\frac{9}{2}$, and $a_2 = -4$. Finally, this gives $a_1 = 1 - a_2 - a_3 = \frac{1}{2}$. Hence,
	\begin{equation}
		Q(h) = \frac{1}{2}I(h) -4I\left(\frac{h}{2}\right) + \frac{9}{2}I\left(\frac{h}{3}\right)
	\end{equation}
	is an approximation of $\int_a^bf(x)\dee x$ with $\bigoh(h^3)$ accuracy.
\end{document}