\documentclass[]{beamer}

\usepackage{etoolbox}
\usepackage{xcolor}
\newtoggle{handout}
\togglefalse{handout}

\iftoggle{handout}{
	\usepackage{pgfpages}
	\pgfpagesuselayout{4 on 1}[border shrink=5mm]
}{}

\setbeamertemplate{footline}
{
	\hbox{\begin{beamercolorbox}[wd=1\paperwidth,ht=2.25ex,dp=1ex,right]{framenumber}%
			\usebeamerfont{framenumber}\insertframenumber{} / \inserttotalframenumber\hspace*{2ex}
	\end{beamercolorbox}}%
	\vskip0pt%
}

\definecolor{bgcolor}{HTML}{d4ffdf}
\definecolor{bulletcolor}{HTML}{004f14}

\setbeamertemplate{blocks}[rounded][shadow=true]
\setbeamertemplate{navigation symbols}{}
\setbeamercolor{frametitle}{fg=black,bg=bgcolor}
\setbeamercolor{structure}{fg=bulletcolor}
\setbeamercolor{block title}{bg=blue!30,fg=black}
\setbeamercolor{block body}{bg=blue!10}

\usefonttheme{serif}



\title{Bifurcation analysis of a discrete-time prey-predator model}
\author{by Pravaiz Naik, Zohreh Eskandari, Hossein Shahkari and Kolade Owolabi}
\institute{Presented by Jacob Hauck}
\date{}

\begin{document}
	\frame{\titlepage}
	
	\begin{frame}{Outline}
		\begin{itemize}
			\item Description and interpretation of a discrete-time predator-prey model
			\vfill
			
			\item Determination of fixed points
			\vfill
			
			\item Bifurcation analysis
			\begin{itemize}
				\item Period-doubling bifurcation
				\item Neimarck-Sacker bifurcation
			\end{itemize}
			\vfill
			
			\item Numerical investigations
			\begin{itemize}
				\item Bifurcation diagram of period-doubling bifurcation
				\item Phase portrait changes at Neimark-Sacker bifurcation
			\end{itemize}
		\end{itemize}
	\end{frame}
	
	\begin{frame}{Model Description}
		\begin{align*}
			x_p(n+1) &= x_p(n)\left[1 + r\left(1 - \frac{x_p(n)}{k}\right) - ay_p(n)\right] \\[0.3em]
			y_p(n+1) &= y_p(n)\left[1 - b + \frac{cx_p(n)}{y_p(n)}\right]
		\end{align*}
		\begin{itemize}
			\item $n \in \mathbb{Z}$: discrete time step
			\item $x_p(n)$: number of prey; $y_p(n)$: number of predators
			\item $r > 0$: intrinsic growth rate of prey
			\item $k > 0$: carrying capacity of prey
			\item $a > 0$: predation rate
			\item $b > 0$: death rate of predators
			\item $c > 0$: conversion rate (of prey into predators)
		\end{itemize}
	\end{frame}
	
	\begin{frame}{Fixed Points}
		$(0,0)$ is a fixed point, but not an interesting one.
		\vfill
		
		Ecologically, the important fixed points occur when $x_p > 0$, and $y_p > 0$, when predator and prey are in equilibrium.
		\vfill
		
		There is one such fixed point:
		\begin{equation*}
			\mathcal{P}_* = \left(\frac{rkb}{ack+br}, \frac{crk}{ack+br}\right).
		\end{equation*} 
	\end{frame}
	
	\begin{frame}{Period-doubling Bifurcations}
		On the time scale $\mathbb{Z}$, fixed points are also 1-periodic solutions. In general, if $x(n)$ is a solution of
		\begin{equation*}
			x(n+1) = f(x(n))
		\end{equation*}
		such that $x(n+p) = x(n)$, where $p$ is the smallest integer that makes this true, then $x_0 = x(0)$ is called a \textbf{periodic point of minimal period $p$}.
		\vfill
		
		A \textbf{period-doubling bifurcation} occurs when the stability of a fixed point changes and a pair of periodic points of minimal period 2 emerge.
		\vfill
		
		See Section 3.4 of \textit{Dynamics and Bifurcations}.
	\end{frame}
	
	\begin{frame}{Period-doubling Bifurcation in the Predator-Prey Model}
		
	\end{frame}
	
	\begin{frame}{Neimarck-Sacker Bifurcations}
		
	\end{frame}
	
	\begin{frame}{Neimarck-Sacker Bifurcation in the Predator-Prey Model}
		
	\end{frame}
	
	\begin{frame}{Period-Doubling Bifurcation Diagram}
		
	\end{frame}
	
	\begin{frame}{Phase Portraits Near the Neimarck-Sacker Bifurcation}
		
	\end{frame}
\end{document}