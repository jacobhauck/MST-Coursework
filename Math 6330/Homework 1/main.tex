\documentclass[nonumber]{homework}
\usepackage{enumitem}

\newcommand{\hwclass}{Math 6108}
\newcommand{\hwname}{Jacob Hauck}
\newcommand{\hwtype}{Homework}

\newcommand{\R}{\textbf{R}}
\newcommand{\dee}{\;\text{d}}
\newcommand{\eps}{\varepsilon}
\newcommand{\pl}[2]{\frac{\partial #1}{\partial #2}}
\newcommand{\dl}[2]{\frac{\text{d} #1}{\text{d} #2}}
\newcommand{\sgn}{\text{sgn}}
\newcommand{\bigoh}{\mathcal{O}}

\newcommand{\hwnum}{1}


\begin{document}
	\maketitle
	
	\question*{1.3 (iv)}
	Let $\ts = \{\sqrt{n} \mid n \in \N_0\}$. Then for $t = \sqrt{n} \in \ts$,
	\begin{itemize}
		\item the next point to the right of $\sqrt{n}$ is $\sqrt{n+1}$, so $\sigma(t) = \sigma(\sqrt{n}) = \sqrt{n+1} = \sqrt{t^2 + 1}$,
		\item the next point to the left of $\sqrt{n}$ is $\sqrt{n-1}$ if $n > 0$. If $n = 0$, then there is no point in $\ts$ to the left of $t=0$, so
		\begin{align*}
			\rho(t) &= \rho(\sqrt{n}) = \begin{cases}
				\sqrt{n-1} & n > 0 \\
				0 & n = 0
			\end{cases}\\
			&= \begin{cases}
				\sqrt{t^2-1} & t > 0 \\
				0 & t = 0.
			\end{cases}
		\end{align*}
		\item $\mu(t) = \sigma(t) - t = \sqrt{t^2 + 1} - t$.
	\end{itemize}
	Every point in $\ts$ is right-scattered because $\sigma(t) = \sqrt{t^2 + 1} > t$. If $t > 0$, then $t$ is left-scattered because $\rho(t) = \sqrt{t^2-1} < t$. The point $0 \in \ts$ is not left-scattered because $\rho(0) =0$, and it is not left-dense either because $0 = \inf \ts$.
	
	\question*{1.4 (ii)}
	Let $\ts = \{0\} \cup [1,2]$. Then $\ts$ is a time-scale, and $1 \in \ts$ does not satisfy $\rho(\sigma(1)) = 1$. Indeed, $\sigma(1) = 1$, and $\rho(1) = 0$, so $\rho(\sigma(1)) = 0 \ne 1$.
	
	Given any time-scale $\ts$ and $t \in \ts$, then $\rho(\sigma(t)) = t$ if and only if $t$ is not left-scattered or $t$ is right-scattered.
	
	\begin{proof}
		Suppose that $t$ is left-scattered and not right-scattered. Then $\sigma(t) = t$, so $\rho(\sigma(t)) = \rho(t) \ne t$. Hence, $\rho(\sigma(t))$ implies that $t$ is not left-scattered or $t$ is right-scattered.
		
		Conversely, if $t$ is right-scattered, then $\sigma(t) \in \ts$ is left-scattered with $\rho(\sigma(t)) = t$. If $t$ is not right-scattered and not left-scattered, then $\rho(t) = t$ and $\sigma(t) = t$, so $\rho(\sigma(t)) = t$.
	\end{proof}
	
	\question*{1.14 (i)}
	
	Define $f : \ts \to \R$ by $f(t) = t^2$. Then $f^\Delta(t) = t + \sigma(t)$.
	
	\begin{proof}
		Let $t \in \ts$, and let $\varepsilon > 0$ be given. Set $\delta = \varepsilon$. Then for all $s \in (t-\delta, t+\delta) \cap \ts$,
		\begin{align*}
			|f(\sigma(t)) - f(s) - (t+\sigma(t))(\sigma(t) - s)| &= |\sigma(t)^2 - s^2 - (t+\sigma(t))(\sigma(t)-s)| \\
			&= |ts + \sigma(t)s - s^2-t\sigma(t)| \\
			&= |(s-t)(\sigma(t) - s)| \\
			&\le \varepsilon|\sigma(t)-s|,
		\end{align*}
		so $f^\Delta(t) = t + \sigma(t)$ by definition.
	\end{proof}
	
	\question*{1.19 (ii)}
	Let $\ts = \{\sqrt{n} \mid n \in \N_0\}$, and define $f : \ts \to \R$ by $f(t) = t^2$. Recall from 1.3 (iv) that $\sigma(t) = \sqrt{t^2+1}$, and every point in $\ts$ is right-scattered. Note that every point $t \in \ts$ is (topologically) isolated, so $f$ is continuous on $\ts$. Therefore, by Theorem 1.16, $f$ is differentiable everywhere on $\ts$, and for $t \in \ts$,
	\begin{equation*}
		f^\Delta(t) = \frac{f(\sigma(t)) - f(t)}{\sigma(t) - t} = \frac{\big(\sqrt{t^2+1}\big)^2 - t^2}{\sqrt{t^2+1} - t} = \sqrt{t^2+1} + t.
	\end{equation*}
	
	\question*{1.19 (iii)}
	Let $\ts = \left\{\frac{n}{2} \mid n \in \N_0\right\}$, and define $f:\ts \to\R$ by $f(t) = t^2$. Then for $t = \frac{n}{2} \in \ts$, the next point to the right of $t$ is $\frac{n+1}{2} = t + \frac{1}{2}$. Hence, $\sigma(t) = t + \frac{1}{2}$. Moreover, every point in $\ts$ is right-scattered, and every point in $\ts$ is (topologically) isolated, so $f$ is continuous on $\ts$. By Theorem 1.16, for $t \in \ts$,
	\begin{equation*}
		f^\Delta(t) = \frac{f(\sigma(t)) - f(t)}{\sigma(t) - t} = \frac{\left(t+\frac{1}{2}\right)^2 - t^2}{t+\frac{1}{2} - t} = 2\left(t+\frac{1}{4}\right) = 2t + \frac{1}{2}.
	\end{equation*}
	
	\question*{1.21 (iv)}
	Suppose that $f : \ts \to \R$ is differentiable at $t \in \ts$, and $f(t)f(\sigma(t)) \ne 0$. Then $\frac{1}{f}$ is differentiable at $t$, and
	\begin{equation*}
		\left(\frac{1}{f}\right)^\Delta(t) = -\frac{f^\Delta(t)}{f(t)f(\sigma(t))}.
	\end{equation*}
	\begin{proof}
		We know from Theorem 1.16 that $f$ is continuous at $t$. Since $f(t) \ne 0$ by assumption, it follows that $f$ is bounded away from 0 in a neighborhood of $t$. That is, there exists $C > 0$ and $\delta_0 > 0$ such that for all $s \in (t-\delta_0, t+\delta_0)\cap \ts$, we have $|f(s)| \ge C$.
		
		Let $\varepsilon > 0$ be given, and set 
		\begin{equation*}
			\varepsilon^*= \varepsilon\left(\frac{1}{C|f(\sigma(t))|} + \frac{\left|f^\Delta(t)\right|}{C\big|f(t)f(\sigma(t))\big|}\right)^{-1}.
		\end{equation*}
		Since $f$ is continuous and delta-differentiable at $t$, we can choose $\delta  \in (0, \delta_0]$ such that for all $s \in (t-\delta, t+\delta) \cap \ts$,
		\begin{enumerate}
			\item $|f(\sigma(t)) - f(s) - f^\Delta(t)(\sigma(t) - s)| \le \varepsilon^* |\sigma(t) - s|$,
			\item $|f(t) - f(s)| \le \varepsilon^*$.
		\end{enumerate}
		Note also that $|f(s)| \ge C$ for all $s \in (t-\delta, t+\delta)\cap\ts$ because $\delta < \delta_0$.
		
		Then
		\begin{align*}
			&\left|\frac{1}{f(\sigma(t))} - \frac{1}{f(s)} - \left(-\frac{f^\Delta(t)}{f(t)f(\sigma(t))}\right)(\sigma(t) - s)\right|\\[0.5em]
			&\qquad\qquad\qquad\qquad= \left|\frac{f(t)f(s) - f(t)f(\sigma(t)) + f(s)f^\Delta(t)(\sigma(t)-s)}{f(t)f(\sigma(t))f(s)}\right| \\[0.5em]
			&\qquad\qquad\qquad\qquad= \left|\frac{f(t)\big[f(s) - f(\sigma(t)) + f^\Delta(t)(\sigma(t)-s)\big] + (f(s) - f(t))f^\Delta(t)(\sigma(t) - s)}{f(t)f(\sigma(t))f(s)}\right| \\[0.5em]
			&\qquad\qquad\qquad\qquad \le \frac{\varepsilon^*|\sigma(t)-s|}{\big|f(\sigma(t))f(s)|} + \frac{\varepsilon^*\left|f^\Delta(t)\right|\cdot|\sigma(t)-s|}{\big|f(t)f(\sigma(t))f(s)\big|} \\[0.5em]
			&\qquad\qquad\qquad\qquad \le \left(\frac{1}{C|f(\sigma(t))|} + \frac{\left|f^\Delta(t)\right|}{C\big|f(t)f(\sigma(t))\big|}\right)\varepsilon^*|\sigma(t)-s| \\[0.5em]
			&\qquad\qquad\qquad\qquad = \varepsilon|\sigma(t)-s|,
		\end{align*}
		so
		\begin{equation*}
			\left(\frac{1}{f}\right)^\Delta(t) = -\frac{f^\Delta(t)}{f(t)f(\sigma(t))}
		\end{equation*}
		by definition.
	\end{proof}
	
	\question*{1.22}
	Let $x$, $y$ and $z$ be delta-differentiable at $t$. Then $xyz$ is delta-differentiable at $t$, and
	\begin{equation*}
		(xyz)^\Delta = x^\Delta yz + xy^\Delta z + xyz^\Delta \quad \text{at }t.
	\end{equation*}
	
	\begin{proof}
		By the product rule, $yz$ is delta-differentiable at $t$. By the product rule again, $xyz = x(yz)$ is also delta-differentiable at $t$. Furthermore, at $t$, the product rule gives (putting $\sigma$ always on the second term)
		\begin{equation*}
			(xyz)^\Delta = (x(yz))^\Delta = x^\Delta yz + x^\sigma (yz)^\Delta = x^\Delta yz + x^\sigma y^\Delta z + x^\sigma y^\sigma z^\Delta,
		\end{equation*}
		as desired.
	\end{proof}
	
	\question*{1.26}
	
	\begin{enumerate}[label={\bf (\roman*)}]
		\item Let $\ts = \left\{\frac{1}{n} \mid n \in \N\right\}\cup\{0\}$, and let $f(t) = \sigma(t)$. Recall from class that
		\begin{equation*}
			\sigma(t) = \begin{cases}
				\frac{t}{1-t} & \text{if } 0 \le t < 1, \\
				1 & \text{if } t = 1.
			\end{cases}
		\end{equation*}
		Since we $1 \not \in \ts^\kappa$, we can get $f^\Delta = \sigma^\Delta$ by taking $f(t) = \frac{t}{1-t}$. We can apply the product rule easily once we know $\left(\frac{1}{1-t}\right)^\Delta$. By Theorem 1.24,
		\begin{equation*}
			\left(\frac{1}{1-t}\right)^\Delta = \frac{1}{(\sigma(t) -1)(t-1)}.
		\end{equation*}
		Hence,
		\begin{equation*}
			f^\Delta(t) = \frac{\sigma(t)}{(\sigma(t) - 1)(t-1)} + \frac{1}{1-t} = \frac{1}{(\sigma(t) -1)(t-1)}, \quad t \in \ts^\kappa.
		\end{equation*}
		
		\item Let $\ts = \{\sqrt{n}\mid n \in \N_0\}$, and let $f(t) = t^2$. Recall from 1.3 (iv) that $\sigma(t) = \sqrt{t^2+1}$. Then, by Theorem 1.24,
		\begin{equation*}
			f^\Delta(t) = \sigma(t)t^0 + (\sigma(t))^0t = t + \sigma(t) = t + \sqrt{t^2+1},
		\end{equation*}
		which agrees with the calculation in 1.19 (ii).
		
		\item Let $\ts = \left\{\frac{n}{2} \mid n \in \N_0\right\}$, and let $f(t) = t^2$. Recall from 1.19 (iii) that $\sigma(t) = t + \frac{1}{2}$. Then, by Theorem 1.24,
		\begin{equation*}
			f^\Delta(t) = t + \sigma(t) = 2t + \frac{1}{2},
		\end{equation*}
		which agrees with the calculation in 1.19 (iii).
		
		\item Let $\ts = \{\sqrt[3]{n} \mid n \in \N_0\}$, and let $f(t) = t^3$. If $t = \sqrt[3]{n} \in \ts$, then
		\begin{equation*}
			\sigma(t) = \sigma(\sqrt[3]{n}) = \sqrt[3]{n+1} = \sqrt[3]{t^3+1}.
		\end{equation*}
		By Theorem 1.24, we have
		\begin{align*}
			f^\Delta(t) &= \sum_{\nu=0}^2 (\sigma(t))^\nu t^{2-\nu} = t^2 + \sigma(t)t + (\sigma(t))^2 \\
			&= t^2 + (t^3+1)^\frac{1}{3}t + (t^3+1)^\frac{2}{3}.
		\end{align*}
		
	\end{enumerate}
	
\end{document}