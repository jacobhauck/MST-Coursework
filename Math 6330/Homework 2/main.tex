\documentclass[nonumber, anonymous]{homework}
\usepackage{enumitem}

\newcommand{\hwclass}{Math 6108}
\newcommand{\hwname}{Jacob Hauck}
\newcommand{\hwtype}{Homework}

\newcommand{\R}{\textbf{R}}
\newcommand{\dee}{\;\text{d}}
\newcommand{\eps}{\varepsilon}
\newcommand{\pl}[2]{\frac{\partial #1}{\partial #2}}
\newcommand{\dl}[2]{\frac{\text{d} #1}{\text{d} #2}}
\newcommand{\sgn}{\text{sgn}}
\newcommand{\bigoh}{\mathcal{O}}

\newcommand{\hwnum}{2}


\begin{document}
	\question*{1.59(ii)}
	
	$\sigma$ and $\mu$ are always rd-continuous, and $\rho$ is not necessarily rd-continuous.
	
	\begin{enumerate}
		\item Suppose that $t \in \ts$ is right-dense. Given $\varepsilon>0$, we can find $s_1,s_2 \in \ts$ such that $t < s_1 < s_2 < t + \varepsilon$. Set $\delta = s_1 -t$. If $s > t$, then $s -t < \delta$ implies that $\sigma(s) - t \le \sigma(s_1) - t \le s_2 - t < \varepsilon$. Therefore, $\lim\limits_{s\to t^+}\sigma(s) = t$.
		
		\item Now suppose that $t \in \ts$ is left-dense. Given $\varepsilon > 0$, if $s < t$ and $t -s < \varepsilon$, then $0 \le t - \sigma(s) \le t - s < \varepsilon$. Therefore, $\lim\limits_{s\to t^-}\sigma(s) = t$.
	\end{enumerate}
	
	If $t$ is left-dense, then by 2., $\lim\limits_{s\to t^-}\sigma(s) = t$, which is finite. 
	
	If $t$ is right-dense, then $\sigma(t) = t$, so $\sigma$ is right-continuous by 1. If, in addition, $t$ is left-scattered or $t = \inf\ts$, then $\sigma$ is trivially left-continuous at $t$; otherwise, $t$ is left-dense, in which case $\sigma$ must be left-continuous at $t$ by 2. Thus, $\sigma$ is continuous at $t$.
	
	The rd-continuity of $\mu$ follows from that of $\sigma$, as $\mu(t) = \sigma(t) - t$; the identity function $t \mapsto t$ is rd-continuous, and rd-continuity is preserved under linear combination.
	
	To see that $\rho$ may not be rd-continuous, consider the time-scale $\ts = \{0\}\cup[1,2]$. In this case, we have
	\begin{equation*}
		\rho(t) = \begin{cases}
			0 & \text{if } t \in \{0, 1\}, \\
			t & \text{if } t \in (1,2].
		\end{cases}
	\end{equation*}
	Then the point $1 \in \ts$ is right-dense, but clearly $\rho$ is not continuous at $1$, as $\rho(1) = 0$, but $\lim\limits_{s\to t^+}\rho(s) = 1$.
	
	
	\question*{1.59(iii)}
	
	rd-continuous implies regulated, so $\sigma$ and $\mu$ are always regulated. We can show that $\rho$ is also always regulated by virtually the same argument from above.
	
	\begin{enumerate}
		\item Suppose that $t \in \ts$ is right-dense. Given $\varepsilon > 0$, if $s > t$ and $s - t < \varepsilon$, then $0 \le \rho(s) - t \le s - t < \varepsilon$. Therefore, $\lim\limits_{s\to t^+}\rho(s) = t$.
		
		\item Now suppose that $t \in \ts$ is left-dense. Given $\varepsilon>0$, we can find $s_1,s_2 \in \ts$ such that $t-\varepsilon < s_1 < s_2 < t$. Set $\delta = t-s_2$. If $s > t$, then $s -t < \delta$ implies that $t-\rho(s) \le t-\rho(s_2) \le t-s_1 < \varepsilon$. Therefore, $\lim\limits_{s\to t^-}\rho(s) = t$.
	\end{enumerate}
	
	Together, 1. and 2. imply that $\rho$ is regulated.
\end{document}