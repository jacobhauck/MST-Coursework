\documentclass{homework}
\usepackage{enumitem}

\newcommand{\hwclass}{Math 6108}
\newcommand{\hwname}{Jacob Hauck}
\newcommand{\hwtype}{Homework}

\newcommand{\R}{\textbf{R}}
\newcommand{\dee}{\;\text{d}}
\newcommand{\eps}{\varepsilon}
\newcommand{\pl}[2]{\frac{\partial #1}{\partial #2}}
\newcommand{\dl}[2]{\frac{\text{d} #1}{\text{d} #2}}
\newcommand{\sgn}{\text{sgn}}
\newcommand{\bigoh}{\mathcal{O}}

\newcommand{\hwnum}{4}

\begin{document}
	\maketitle
	
	\question
	Define the \textbf{Fourier transform operator} $\fourier: L^1(\R) \to L^\infty(\R)$ by
	\begin{equation}
		\label{eq:fourier_transform}
		\fourier(f)(y) = \frac{1}{\sqrt{2\pi}}\int_{-\infty}^\infty e^{iyx}f(x)\dee x.
	\end{equation}
	
	\begin{arabicparts}
		\questionpart We note that the function $x \mapsto e^{iyx}f(x)$ is clearly integrable if $f$ is, so the integral in (\ref{eq:fourier_transform}) exists for all $y$. We show that $\fourier (f) \in L^\infty(\R)$ as claimed, and $\norm{\fourier f}_{L^\infty} \le \frac{1}{\sqrt{2\pi}}\norm{f}_{L^1}$. Indeed, for $y \in \R$,
		\begin{align}
			|\fourier(f)(y)| &= \frac{1}{\sqrt{2\pi}}\left|\int_{-\infty}^\infty e^{iyx}f(x)\dee x\right| \\
			&\le \frac{1}{\sqrt{2\pi}}\int_{-\infty}^\infty \left|e^{iyx}f(x)\right|\dee x = \frac{1}{\sqrt{2\pi}}\int_{-\infty}^\infty |f(x)|\dee x = \frac{1}{\sqrt{2\pi}}\norm{f}_{L^1}.
		\end{align}
		Therefore, $\norm{\fourier f}_{L^\infty} \le \frac{1}{\sqrt{2\pi}}\norm{f}_{L^1}$.
		
		\questionpart Suppose that $f \in C^2(\R)$, and $f, f', f'' \in L^1(\R)$, and $f(x), f'(x), f''(x)\to 0$ as $x\to\pm\infty$. Then there exists a constant $C$ such that $|y^2\fourier(f)(y)| \le C$ for all $y \in \R$. Furthermore, $\fourier(f) \in L^1(\R)$.
		
		\begin{proof}
			Since $f'' \in L^1(\R)$, we can take its Fourier transform, which yields
			\begin{equation}
				\fourier(f'')(y) = \frac{1}{\sqrt{2\pi}}\int_{-\infty}^\infty e^{iyx}f''(x)\dee x.
			\end{equation}
			We can integrate by parts because $f', f\in L^1(\R)$ and are continuous, and $f(x), f'(x)\to 0$ as $x \to\pm \infty$. This gives
			\begin{align}
				\fourier(f'')(y) &= \frac{1}{\sqrt{2\pi}}\left[f'(x)e^{iyx}\Big\vert_{-\infty}^\infty - iy\int_{-\infty}^\infty e^{iyx}f'(x)\dee x\right]\\
				&= \frac{iy}{\sqrt{2\pi}}\left[-f(x)e^{iyx}\Big\vert_{-\infty}^\infty + iy\int_{-\infty}^\infty e^{iyx}f(x)\dee x\right] \\
				&= -y^2\fourier(f)(y).
			\end{align}
			By the reasoning in 1.1), it follows that
			\begin{equation}
				|y^2\fourier (f)(y)| = |\fourier(f'')(y)| \le \frac{1}{\sqrt{2\pi}}\norm{f''}_{L^1}
			\end{equation}
			for all $y \in \R$.
			
			Thus, if $C = \frac{1}{\sqrt{2\pi}}\norm{f''}_{L^1}$, then $|\fourier(f)(y)|\le \frac{C}{y^2}$ for all $y \in \R$. On the other hand, $\fourier(f) \in L^\infty(\R)$ by part 1.1), so $\fourier(f)$ is dominated by the integrable function
			\begin{equation}
				\phi(y) = \begin{cases}
					\norm{\fourier(f)}_{L^\infty} & y \in [-1,1], \\
					\frac{C}{y^2} & \text{otherwise}.
				\end{cases}
			\end{equation}
			By the integral comparison test, $\fourier(f) \in L^1(\R)$.
		\end{proof}
		
		\questionpart Formally, $\fourier^2(f)(y) = f(-y)$.
		
		\begin{proof}
			We note that if $f \in C^1\cap L^1(\R)$, and $f' \in L^1(\R)$, and $f(x)\to 0$ as $x\to\pm\infty$, then we can use integration by parts to show that
			\begin{align}
				\fourier(f')(y) &= \frac{1}{\sqrt{2\pi}}\int_{-\infty}^\infty e^{iyx}f'(x)\dee x = \frac{1}{\sqrt{2\pi}}\left[e^{iyx}f(x)\Big\vert_{-\infty}^\infty - iy\int_{-\infty}^\infty e^{iyx}f(x)\dee x\right] \\
				&= -iy\fourier(f)(y).
			\end{align}
			On the other hand, let $f\in L^1(\R)$, and define $g(x) = ixf(x)$. If $g \in L^1(\R)$ as well, then
			\begin{align}
				\dl{}{y} \frac{1}{\sqrt{2\pi}}\fourier(f)(y) &= \dl{}{y} \frac{1}{\sqrt{2\pi}}\int_{-\infty}^\infty e^{iyx}f(x)\dee x =\frac{1}{\sqrt{2\pi}}\int_{-\infty}^\infty \pl{}{y}\left[e^{iyx}f(x)\right]\dee x \\
				&= \frac{1}{\sqrt{2\pi}}\int_{-\infty}^\infty e^{iyx}ixf(x)\dee x = \fourier(g)(y).
			\end{align}
			If we take $f(x) = e^{-ax^2}$, then $f$ satisfies the above assumptions. Since $f'(x) = -2axf(x)$,
			\begin{equation}
				2ai \dl{}{y}\fourier(f)(y) = 2ai\fourier(i(\cdot)f(\cdot))(y) = \fourier(-2a(\cdot)f(\cdot))(y) = \fourier(f')(y) =-iy\fourier(f)(y).
			\end{equation}
			Hence, $\fourier(f)(y)$ is the unique solution of the IVP
			\begin{equation}
				u' = -\frac{y}{2a}u,\qquad u(0) = \fourier(f)(0).
			\end{equation}
			The general solution of the differential equation is
			\begin{equation}
				u(y) = u(0)e^{-\frac{y^2}{4a}}.
			\end{equation}
			Since
			\begin{equation}
				\fourier(f)(0) = \frac{1}{\sqrt{2\pi}}\int_{-\infty}^\infty e^{-ax^2}\dee x = \frac{1}{\sqrt{2a}},
			\end{equation}
			it follows that
			\begin{equation}
				\label{eq:gaussian_fourier_transform}
				\fourier(f)(y) = \frac{1}{\sqrt{2a}} e^{-\frac{y^2}{4a}}.
			\end{equation}
			Thus, if $\phi_a(x) = e^{-ax^2}$, then, formally,
			\begin{equation}
				\fourier(1)(y) = \fourier\left(\lim_{a\to 0^+}\phi_a\right)(y) = \lim_{a\to 0^+}\fourier(\phi_a)(y)= \lim_{a\to 0^+}\frac{1}{\sqrt{2a}}e^{-\frac{y^2}{4a}}.
			\end{equation}
			We would like to interpret the last limit formally as a constant multiple of the Dirac delta function. Clearly,
			\begin{equation}
				\lim_{a\to0^+} \frac{1}{\sqrt{2a}}e^{-\frac{y^2}{4a}} = \begin{cases}
					0 & y \ne 0, \\
					\infty & y = 0.
				\end{cases}
			\end{equation}
			At the same time, for any $a > 0$,
			\begin{equation}
				\int_{-\infty}^\infty \frac{1}{\sqrt{2a}}e^{-\frac{y^2}{4a}}\dee y = \frac{1}{\sqrt{2a}}\sqrt{4a\pi} = \sqrt{2\pi},
			\end{equation}
			so it makes sense formally that we should have $\fourier(1)(y) = \sqrt{2\pi}\delta(y)$.
			
			Now, if we consider applying the Fourier transform twice to a function $f$, we get
			\begin{align}
				\fourier\fourier(f)(y) &= \frac{1}{2\pi}\int_{-\infty}^\infty\int_{-\infty}^\infty e^{ixy}e^{izx} f(z)\dee z\dee x \\
				&= \frac{1}{\sqrt{2\pi}}\int_{-\infty}^\infty f(z) \frac{1}{\sqrt{2\pi}}\int_{-\infty}^\infty e^{ix(y+z)}\dee x\dee z\\
				&= \frac{1}{\sqrt{2\pi}}\int_{-\infty}^\infty f(z) \fourier(1)(y+z)\dee z \\
				&= \int_{-\infty}^\infty f(z)\delta(y+z)\dee z \\
				&= \int_{-\infty}^\infty f(z-y)\delta(z)\dee z \\
				&= f(-y).
			\end{align}
		\end{proof}
		
		\questionpart Define $g(y) = f(-y)$ for some function $f$. Based on the formal result from part 1.3), we see immediately that
		\begin{equation}
			\fourier^4(f)(y) = \fourier^2(\fourier^2(f))(y) = \fourier^2(g)(y) = g(-y) = f(y).
		\end{equation}
		Since $f$ was arbitrary, it follows formally that $\fourier^4 = I$, the identity operator.
		
		\questionpart Let $p(x) = x^4$. By the Spectral Mapping Theorem,
		\begin{equation}
			p(\sigma(\fourier)) = \sigma(p(\fourier)).
		\end{equation}
		Since $p(\fourier) = \fourier^4 = I$, the spectrum of $p(\fourier)$ is just $\sigma(I) = \{1\}$, as the operator $I - \lambda I = (1-\lambda)I$ is invertible, with inverse $\frac{1}{1-\lambda}I$, if and only if $\lambda \ne 1$. Therfore, if $\lambda \in \sigma(\fourier)$, then $p(\lambda) = 1$, that is, $\lambda^4 = 1$. The possible solutions of this equation are $1,-1,i,-i$, so $\sigma(\fourier) \subseteq \{1,-1,i,-i\}$.
		
		\questionpart If we reuse the result in equation (\ref{eq:gaussian_fourier_transform}) with $a = \frac{1}{2}$, we see that if $f(x) = e^{-\frac{1}{2}x^2}$, then
		\begin{equation}
			\fourier(f)(y) = e^{-\frac{1}{2}y^2}
		\end{equation}
		as well. Thus, $\fourier f = f$, so $f$ is an eigenfunction of $\fourier$ with corresponding eigenvalue $1$.
	\end{arabicparts}
	
	\question
	On this question, we will reuse the notation from Question 2 of Homework 3.
	
	Let $\dot{L}^2(-\pi, \pi) = \{f \in L^2(-\pi, \pi) : f = \bar{f}\text{ and } \mathrm{mean}(f) = 0\}$, where $\mathrm{mean}(f) = \frac{1}{2\pi}\int_{-\pi}^\pi f$. Consider the following problem.
	\begin{equation}
		\label{eq:q2_problem}
		\text{Let } f\in \dot{L}^2(-\pi,\pi). \text{ Find } u\in H \text{ such that } -u'' = f,
	\end{equation}
	where $H$ is the space defined in Homework 3.
	\begin{arabicparts}
		\questionpart Let $f \in \dot{L}^2(-\pi, \pi)$. Then $f \in L^2(-\pi,\pi)$, and, recalling from Homework 3, there exists $\{f_j\} \subset \C$ such that
		\begin{equation}
			f = \sum_j f_j e_j, \qquad f_j = (f, e_j).
		\end{equation}
		Since $e_0 = \text{constant}$, we have $f_0 = (f,e_0) \propto \mathrm{mean}(f) = 0$, so $f_0 = 0$. Furthermore, by an argument we used several times in Homework 3, the fact that $f = \bar{f}$ implies that $f_{-j} = \bar{f}_j$. Lastly, by Parseval's identity,
		\begin{equation}
			\sum_{j\ne 0}j^{-2}|f_j|^2 \le \sum_{j\ne0}|f_j|^2 = \lVert f\rVert_2^2 < \infty,
		\end{equation}
		so $f \in H^{-1}$ from Homework 3 because $\{f_j\}_{j\ne0} \in S_{H^{-1}}$. Therefore, $\dot{L}^2(-\pi,\pi) \subseteq H^{-1}$.
		
		We claim that for $f \in \dot{L}^2(-\pi,\pi)$ and $u \in H$,
		\begin{equation}
			-u'' = f \qquad \iff \qquad B(u,v) = f(v) \quad\forall v\in H,
		\end{equation}
		where we define the action of $f$ on $v$ in the same way as in Homework 3, and
		\begin{equation}
			B(u,v) = \sum_{j\ne0}j^2u_j\bar{v}_j,\qquad \{u_j\} = \varphi(u),\quad\{v_j\} = \varphi(v),
		\end{equation}
		where $\varphi : H \to S_H$ is define as in Homework 3. We use essentially the same formal argument that we used on 2.5) in Homework 3.
		
		Suppose that $-u'' = f$, and let $\{f_j\} = \psi(f)$, $\{u_j\} = \varphi(u)$. Formally differentiating the Fourier series for $u$, we have
		\begin{equation}
			-\sum_{j\ne0}f_je_j = -f = u'' = \sum_{j\ne0} -j^2u_je_j.
		\end{equation}
		Therefore, $f_j = j^2u_j$ for all $j$, and for any $v \in H$,
		\begin{equation}
			B(u,v) = \sum_{j\ne0}j^2u_j\bar{v}_j = \sum_{j\ne0}f_j\bar{v}_j = f(v).
		\end{equation}
		On the other hand, suppose that $B(u,v) = f(v)$ for all $v \in H$. Clearly, $e_j + e_{-j} \in H$, and $e_j - e_{-j} \in H$, so
		\begin{equation}
			j^2u_j + j^2u_{-j} = f_j + f_{-j},\qquad j^2u_j - j^2u_{-j} = f_j - f_{-j},
		\end{equation}
		which implies that $j^2u_j = f_j$ for all $j$. By the same formal differentiation reasoning, it follows that $-u'' = f$. Additionally, we have $u \in H$ because
		\begin{equation}
			\bar{u}_{-j} = \frac{\bar{f}_{-j}}{j^2} = \frac{f_j}{j^2} = u_j, \qquad \sum_{j\ne0}j^2|u_j|^2 = \sum_{j\ne0}j^2\left|\frac{f_j}{j^2}\right|^2 \le \sum_{j\ne0}|f_j|^2 < \infty,
		\end{equation}
		which implies that $\{u_j\} \in S_H$.
		
		The function $B$ is bilinear because for any $\alpha,\beta \in \R$, and any $u,v,w\in H$,
		\begin{equation}
			B(\alpha u + \beta v, w) = \sum_{j\ne0}j^2(\alpha u_j + \beta v_j)\bar{w}_j = \alpha\sum_{j\ne0}j^2u_j\bar{w}_j + \beta\sum_{j\ne0}j^2v_j\bar{w}_j = \alpha B(u,w) + \beta B(v,w),
		\end{equation}
		and
		\begin{equation}
			B(w, \alpha u + \beta v) =\sum_{j\ne0}j^2w_j\overline{\alpha u_j + \beta v_j} = \alpha\sum_{j\ne 0}j^2w_j\bar{u}_j + \beta \sum_{j\ne0}j^2w_j\bar{v}_j = \alpha B(w,u) + \beta B(w,v)
		\end{equation}
		because $\varphi(\alpha u + \beta v) = \alpha\varphi(u) + \beta\varphi(v)$.
		
		The function $B$ is also continuous because for any $u,v \in H$,
		\begin{equation}
			|B(u,v)| = \left|\sum_{j\ne0}j^2u_j\bar{v}_j\right| \le \lVert u\rVert_H \lVert v\rVert_H
		\end{equation}
		by the Cauchy-Schwarz inequality.
		
		Lastly, $B$ is coercive because for any $u \in H$,
		\begin{equation}
			B(u,u) = \sum_{j\ne0}j^2|u_j|^2 = \lVert u_j\rVert_H^2.
		\end{equation}
		Hence, the Lax-Milgram Theorem implies that, given $f \in \dot{L}^2(-\pi,\pi) \subseteq H^{-1} \subseteq H^*$, there is exists a unique $u \in H$ such that $B(u,v) = f(v)$ for all $v \in H$. That is, there exists a unique $u \in H$ such that $-u'' = f$.
		
		\questionpart Let $T : \dot{L}^2(-\pi, \pi) \to H$ denote the solution operator of (\ref{eq:q2_problem}), which exists by 2.1). Then $T$ is compact as an operator on $\dot{L}^2(-\pi,\pi)$.
		
		\begin{proof}
			Given $f \in \dot{L}^2(-\pi,\pi)$, there exists $u \in H$ such that
			\begin{equation}
				\lVert Tf \rVert_H = \lVert u \rVert_H,
			\end{equation}
			and $B(u,v) = f(v)$ for all $v \in H$. In particular, if we take $v = u$, we obtain
			\begin{equation}
				\lVert u \rVert_H^2 = f(u) = \sum_{j\ne 0}f_j\bar{u}_j \le \lVert u \rVert_H \lVert f\rVert_{L^2},
			\end{equation} 
			which implies that $\lVert Tf\rVert_H \le \lVert f\rVert_{L^2}$, so $T$ is bounded.
			
			As we showed in 2.1), $Tf = u$ if and only if $f_j = j^2u_j$, where $\{u_j\} = \varphi(u)$, and $\{f_j\} = \psi(f)$. Thus, $u_j = \frac{f_j}{j^2}$.
			
			Let $A$ be a bounded set in $\dot{L}^2(-\pi,\pi)$. Then $T(A)$ is bounded in $H$ because $T$ is a bounded operator from $\dot{L}^2(-\pi,\pi)$ to $H$. Then there exists $M > 0$ such that $\lVert u \rVert_H \le M$ for all $u \in T(A)$.
			
			Thus, for any $J > 0$ and any $u \in T(A)$,
			\begin{equation}
				\label{eq:u_h_l2_est}
				\sum_{|j| > J} |u_j|^2 \le \frac{1}{J^2}\sum_{|j|>J}j^2|u_j^2| \le \frac{M}{J^2},
			\end{equation}
			where $\{u_j\} = \varphi(u)$. We recall from Homework 3 that
			\begin{equation}
				\lVert u \rVert_{L^2}^2 = \sum_{j\ne0}|u_j|^2,
			\end{equation}
			so the tail of the series form of the $L^2$ norm of $u$ is uniformly small for $u \in T(A)$, which is what allows us to establish the compactness of $T$.
			
			Let $\{u^n\}$ be a sequence in $T(A)$. The space $H$ is a Hilbert space by Homework 3, and $\{u^n\}$ is bounded because $T(A)$ is bounded, so there exists a weakly convergent subsequence of $\{u^n\}$, call it $\{u^{n_k}\}$. Then $\{u^{n_k}_j\}_k$ converges for all $j$, where $\{u^{n_k}_j\}$ are the Fourier coefficients of $u^{n_k}$ with respect to $\{e_j\}$. In particular, each such sequence is Cauchy.
			
			Let $\varepsilon > 0$ be given. By (\ref{eq:u_h_l2_est}), we can choose $J$ large enough that
			\begin{equation}
				\sum_{|j| > J} |u^{n_k}_j|^2 < \varepsilon^2
			\end{equation}
			for all $k$. We can also choose $N$ large enough that $k,\ell>N$ implies that $|u^{n_k}_j - u^{n_\ell}_j|^2 < \frac{\varepsilon^2}{2J}$ for all $|j| \le J$. Then
			\begin{align}
				\lVert u^{n_k} - u^{n_\ell}\rVert_{L^2}^2 &= \sum_{|j|\le J}|u^{n_k}_j - u^{n_\ell}_j|^2 + \sum_{|j| > J}|u^{n_k}_j - u^{n_\ell}_j|^2 \\
				&\le \varepsilon^2 + 2\sum_{|j|>J}|u^{n_k}_j|^2 + 2\sum_{|j|>J}|u^{n_\ell}_j|^2\\
				&\le 5\varepsilon^2,
			\end{align}
			so $k,\ell > N$ implies that $\lVert u^{n_k} - u^{n_\ell}\rVert_H < \sqrt{5}\varepsilon$.
			
			This implies that $\{u^{n_k}\}_k$ is Cauchy in $L^2(-\pi,\pi)$. Since $L^2(-\pi,\pi)$ is a Hilbert space, it follows that $\{u^{n_k}\}$ converges to some $u \in L^2(-\pi,\pi)$. It remains to show that $u \in \dot{L}^2(-\pi,\pi)$.
			
			We recall from Homework 3 that $\{u^{n_k}_j\}_{j\ne0} = \varphi(u^{n_k})$, so $\overline{u^{n_k}_j} = u^{n_k}_{-j}$ for all $k$ and all $j\ne0$, and $u^{n_k}_0 = 0$ for all $k$. Taking the limit as $k\to\infty$, we get $\bar{u}_j = \bar{u}_{-j}$ for all $j \ne 0$, and $u_0 = 0$. Thus, $\bar{u} = u$ by the same reasoning used several times in Homework 3, and $\mathrm{mean}(u)\propto (u,e_0)=u_0=0$; therefore, $u \in \dot{L}^2(-\pi,\pi)$ by definition.
			
			Thus, every sequence in the bounded set $T(A)$ has a convergent subsequence, so $T(A)$ is pre-compact in $H$. This implies that $T$ is compact by definition.
		\end{proof}
		
		\questionpart The operator $T$ is self-adjoint.
		\begin{proof}
			Let $f,g \in H$, and define $u = Tf$, and $v = Tg$. Then, by the same reasoning as in 2.1), if $\{u_j\} = \varphi(u)$, $\{v_j\} = \varphi(v)$, $\{f_j\} = \varphi(f)$, and $\{g_j\} = \varphi(g)$, then
			\begin{equation}
				u_j = \frac{f_j}{j^2}, \qquad v_j = \frac{g_j}{j^2}.
			\end{equation}
			Thus,
			\begin{equation}
				(Tf, g)_H = (u, g)_H = \sum_{j\ne0}j^2u_j\bar{g}_j = \sum_{j\ne0}j^2f_j\bar{v}_j = (f,v)_H = (f,Tg)_H,
			\end{equation}
			so $T$ is self-adjoint.
		\end{proof}
		
		\questionpart
		Suppose that $Tf = \lambda f$ for $f \in H$, with $f\ne0$. Note that since $T$ is self-adjoint, we must have $\lambda \in \R$. By the reasoning in 2.1), we must have $j^{-2}f_j = \lambda f_j$ for all $j\ne0$, where $\{f_j\} = \varphi(f)$. Then either $f_j = 0$ or $\lambda = j^{-2}$ for all $j \ne 0$. We cannot have $f_j = 0$ for all $j$, because then $f = 0$ by the linearity of $\varphi^{-1}$.
		
		Thus, there exists some $k > 0$ such that $f_{k} \ne0$, which implies that $\lambda = k^{-2}$. The equation $\lambda f_j = j^{-2}f_j$ for all $j$ implies that $f_j = 0$ for all $j \ne \pm k$. Since $f_{-k} = \bar{f}_{k}$, it follows that $f = f_{k}e_{k} + \bar{f}_ke_{-k}$. Supposing that $f_j = a + ib$, we must have
		\begin{align}
			f(x) &= \frac{1}{\sqrt{2\pi}}\left((a+ib)e^{ikx} + (a-ib)e^{-ikx}\right) \\
			&= \frac{1}{\sqrt{2\pi}}\big(a\cos(kx) - b\sin(kx) +ib\cos(kx) + ia\sin(kx)\\
			&\quad{}+ a\cos(kx) -b\sin(kx) - ib\cos(kx) -ia\sin(kx)\big) \\
			&= \frac{1}{\sqrt{2\pi}}\left(2a\cos(kx) - 2b\sin(kx)\right).
		\end{align}
		Thus, if $\lambda$ is an eigenvalue of $T$, then $\lambda = k^{-2}$ for some integer $k\ne0$, and the corresponding eigenvectors must be linear combinations of $\cos(kx)$ and $\sin(kx)$; that is, the corresponding eigenspace is $\mathrm{span}\{\cos(kx), \sin(kx)\}$. 
		
		It is not hard to see by reversing the above logic that the converse is true, namely, that $k^{-2}$ is an eigenvalue of $T$ for all integers $k \ne0$, and its corresponding eigenspace is $\mathrm{span}\{\cos(kx),\sin(kx)\}$.
		
		\questionpart
		Let $c \in \C$ be given. Then for any $j > 0$, there exists $a,b \in R$ such that $ce_j + \bar{c}e_{-j} = a\cos(jx) + b\sin(jx)$; indeed, by the calculation in 2.4), we just need to choose $a = \sqrt{\frac{2}{\pi}}\Re(c)$, and $b = -\sqrt{\frac{2}{\pi}}\Im(c)$. Therefore, given $u \in H$, the partial sum of the Fourier series for $u$ can be written as a linear combination of elements of the set $\mathcal{B} = \{\cos(jx),\sin(jx)\}_{j>0}$. Since $H$ is a Hilbert space, the Fourier series of $u$ converges to $u$, so $u$ is the limit of a sequence of elements of $\mathrm{span}(\mathcal{B})$.
		
		In other words, $\mathcal{B}$ is a basis for $H$. In fact, it is an orthogonal basis, as we show now. Let $\{c^k_j\}_j = \varphi(\cos(kx))$, and let $\{s^k_j\}_j = \varphi(\sin(kx))$. Then, by the Euler formula relating $e^{ijx}$ to $\sin(x)$ and $\cos(x)$,
		\begin{equation}
			\label{eq:euler}
			c^k_j = \begin{cases}
				\sqrt{\frac{\pi}{2}} & j = |k| \\
				0 & \text{otherwise},
			\end{cases}\qquad
			s^k_j = \begin{cases}
				-i\sqrt{\frac{\pi}{2}}\mathrm{sgn}(j) & j = |k| \\
				0 & \text{otherwise},
			\end{cases}
		\end{equation}
		where $\mathrm{sgn}(j)$ is the sign of $j$. Hence, for $k, \ell > 0$,
		\begin{align}
			(\cos(kx), \cos(\ell x))_H &= \sum_{j\ne0}j^2c^k_j\overline{c^\ell_j} = \begin{cases}
				0 & k \ne \ell \\
				\frac{\pi}{2}k^2 & k = \ell,
			\end{cases}\\
			(\cos(kx), \sin(\ell x))_H &= \sum_{j\ne0}j^2c^k_j\overline{s^\ell_j} = \begin{cases}
				0 & k\ne \ell \\
				\frac{\pi}{2}\left(-i + i\right)k^2 = 0 & k = \ell
			\end{cases} = 0,\\
			(\sin(kx), \sin(\ell x))_H &= \sum_{j\ne0}j^2s^k_j\overline{s^\ell_j} = \begin{cases}
				0 & k\ne \ell \\
				\frac{\pi}{2}k^2 & k = \ell.
			\end{cases}
		\end{align}
		Therefore, $\mathcal{B}$ is orthogonal in $H$. Moreover, we can clearly modify the elements of $\mathcal{B}$ so that they are orthonormal; if we set $\mathcal{B}' = \left\{\sqrt{\frac{2}{\pi}}j^{-1}\cos(jx), \sqrt{\frac{2}{\pi}}j^{-1}\sin(jx)\right\}$, then $\mathcal{B}'$ is an orthonormal basis for $H$.
		
		Finally, we note that $\mathcal{B}'$ is the orthonormal basis that diagonalizes $T$ in the sense of the spectral theorem for self-adjoint, compact operators. Let $c^k = \cos(kx)$, and let $s^k = \sin(kx)$. Define $u^k = T(c^k)$, and $v^k = T(s^k)$, and define $\{u^k_j\} = \varphi(u^k)$, and $\{v^k_j\} = \varphi(v^k)$. By the reasoning in 2.1), and by (\ref{eq:euler}),
		\begin{equation}
			u^k_j = k^{-2}c^k_j \implies u^k = k^{-2}c^k, \qquad v^k_j = k^{-2}s^k_j \implies v^k = k^{-2}s^k;
		\end{equation}
		Recall from 2.4) that the eigenvalues of $T$ are $k^{-2}$, where $k$ is a positive integer, with corresponding eigenfunctions $s^k$ and $c^k$. Thus, we have shown that the set of eigenfunctions $\mathcal{B}'$ is an orthonormal basis for $H$, and for any eigenfunction $f \in \mathcal{B}'$, we have $Tf = \lambda f$, where $\lambda$ is the eigenvalue corresponding to $f$.
	\end{arabicparts}
\end{document}