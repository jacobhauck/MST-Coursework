\documentclass{homework}
\usepackage{enumitem}

\newcommand{\hwclass}{Math 6108}
\newcommand{\hwname}{Jacob Hauck}
\newcommand{\hwtype}{Homework}

\newcommand{\R}{\textbf{R}}
\newcommand{\dee}{\;\text{d}}
\newcommand{\eps}{\varepsilon}
\newcommand{\pl}[2]{\frac{\partial #1}{\partial #2}}
\newcommand{\dl}[2]{\frac{\text{d} #1}{\text{d} #2}}
\newcommand{\sgn}{\text{sgn}}
\newcommand{\bigoh}{\mathcal{O}}

\newcommand{\hwnum}{4}

\begin{document}
	\maketitle
	
	\question
	Define the \textbf{Fourier transform operator} $\fourier: L^1(\R) \to L^\infty(\R)$ by
	\begin{equation}
		\label{eq:fourier_transform}
		\fourier(f)(y) = \frac{1}{\sqrt{2\pi}}\int_{-\infty}^\infty e^{iyx}f(x)\dee x.
	\end{equation}
	
	\begin{arabicparts}
		\questionpart We note that the function $x \mapsto e^{iyx}f(x)$ is clearly integrable if $f$ is, so the integral in (\ref{eq:fourier_transform}) exists for all $y$. We show that $\fourier (f) \in L^\infty(\R)$ as claimed, and $\norm{\fourier f}_{L^\infty} \le \frac{1}{\sqrt{2\pi}}\norm{f}_{L^1}$. Indeed, for $y \in \R$,
		\begin{align}
			|\fourier(f)(y)| &= \frac{1}{\sqrt{2\pi}}\left|\int_{-\infty}^\infty e^{iyx}f(x)\dee x\right| \\
			&\le \frac{1}{\sqrt{2\pi}}\int_{-\infty}^\infty \left|e^{iyx}f(x)\right|\dee x = \frac{1}{\sqrt{2\pi}}\int_{-\infty}^\infty |f(x)|\dee x = \frac{1}{\sqrt{2\pi}}\norm{f}_{L^1}.
		\end{align}
		Therefore, $\norm{\fourier f}_{L^\infty} \le \frac{1}{\sqrt{2\pi}}\norm{f}_{L^1}$.
		
		\questionpart Suppose that $f \in C^2(\R)$, and $f, f', f'' \in L^1(\R)$, and $f(x), f'(x), f''(x)\to 0$ as $x\to\pm\infty$. Then there exists a constant $C$ such that $|y^2\fourier(f)(y)| \le C$ for all $y \in \R$. Furthermore, $\fourier(f) \in L^1(\R)$.
		
		\begin{proof}
			Since $f'' \in L^1(\R)$, we can take its Fourier transform, which yields
			\begin{equation}
				\fourier(f'')(y) = \frac{1}{\sqrt{2\pi}}\int_{-\infty}^\infty e^{iyx}f''(x)\dee x.
			\end{equation}
			We can integrate by parts because $f', f\in L^1(\R)$ and are continuous, and $f(x), f'(x)\to 0$ as $x \to\pm \infty$. This gives
			\begin{align}
				\fourier(f'')(y) &= \frac{1}{\sqrt{2\pi}}\left[f'(x)e^{iyx}\Big\vert_{-\infty}^\infty - iy\int_{-\infty}^\infty e^{iyx}f'(x)\dee x\right]\\
				&= \frac{iy}{\sqrt{2\pi}}\left[-f(x)e^{iyx}\Big\vert_{-\infty}^\infty + iy\int_{-\infty}^\infty e^{iyx}f(x)\dee x\right] \\
				&= -y^2\fourier(f)(y).
			\end{align}
			By the reasoning in 1.1), it follows that
			\begin{equation}
				|y^2\fourier (f)(y)| = |\fourier(f'')(y)| \le \frac{1}{\sqrt{2\pi}}\norm{f''}_{L^1}
			\end{equation}
			for all $y \in \R$.
			
			Thus, if $C = \frac{1}{\sqrt{2\pi}}\norm{f''}_{L^1}$, then $|\fourier(f)(y)|\le \frac{C}{y^2}$ for all $y \in \R$. On the other hand, $\fourier(f) \in L^\infty(\R)$ by part 1.1), so $\fourier(f)$ is dominated by the integrable function
			\begin{equation}
				\phi(y) = \begin{cases}
					\norm{\fourier(f)}_{L^\infty} & y \in [-1,1], \\
					\frac{C}{y^2} & \text{otherwise}.
				\end{cases}
			\end{equation}
			By the integral comparison test, $\fourier(f) \in L^1(\R)$.
		\end{proof}
		
		\questionpart Formally, $\fourier^2(f)(y) = f(-y)$.
		
		\begin{proof}
			We note that if $f \in C^1\cap L^1(\R)$, and $f' \in L^1(\R)$, and $f(x)\to 0$ as $x\to\pm\infty$, then we can use integration by parts to show that
			\begin{align}
				\fourier(f')(y) &= \frac{1}{\sqrt{2\pi}}\int_{-\infty}^\infty e^{iyx}f'(x)\dee x = \frac{1}{\sqrt{2\pi}}\left[e^{iyx}f(x)\Big\vert_{-\infty}^\infty - iy\int_{-\infty}^\infty e^{iyx}f(x)\dee x\right] \\
				&= -iy\fourier(f)(y).
			\end{align}
			On the other hand, let $f\in L^1(\R)$, and define $g(x) = ixf(x)$. If $g \in L^1(\R)$ as well, then
			\begin{align}
				\dl{}{y} \fourier(f)(y) &= \dl{}{y}\int_{-\infty}^\infty e^{iyx}f(x)\dee x =\int_{-\infty}^\infty \pl{}{y}\left[e^{iyx}f(x)\right]\dee x \\
				&= \int_{-\infty}^\infty e^{iyx}ixf(x)\dee x = \fourier(g)(y).
			\end{align}
			If we take $f(x) = e^{-ax^2}$, then $f$ satisfies the above assumptions. Since $f'(x) = -2axf(x)$,
			\begin{equation}
				2ai \dl{}{y}\fourier(f)(y) = 2ai\fourier(i(\cdot)f(\cdot))(y) = \fourier(-2a(\cdot)f(\cdot))(y) = \fourier(f')(y) =-iy\fourier(f)(y).
			\end{equation}
			Hence, $\fourier(f)(y)$ is the unique solution of the IVP
			\begin{equation}
				u' = -\frac{y}{2a}u,\qquad u(0) = \fourier(f)(0).
			\end{equation}
			The general solution of the differential equation is
			\begin{equation}
				u(y) = u(0)e^{-\frac{y^2}{4a}}.
			\end{equation}
			Since
			\begin{equation}
				\fourier(f)(0) = \frac{1}{\sqrt{2\pi}}\int_{-\infty}^\infty e^{-ax^2}\dee x = \frac{1}{\sqrt{2a}},
			\end{equation}
			it follows that
			\begin{equation}
				\label{eq:gaussian_fourier_transform}
				\fourier(f)(y) = \frac{1}{\sqrt{2a}} e^{-\frac{y^2}{4a}}.
			\end{equation}
			Thus, if $\phi_a(x) = e^{-ax^2}$, then, formally,
			\begin{equation}
				\fourier(1)(y) = \fourier\left(\lim_{a\to 0^+}\phi_a\right)(y) = \lim_{a\to 0^+}\fourier(\phi_a)(y)= \lim_{a\to 0^+}\frac{1}{\sqrt{2a}}e^{-\frac{y^2}{4a}}.
			\end{equation}
			We would like to interpret the last limit formally as a constant multiple of the Dirac delta function. Clearly,
			\begin{equation}
				\lim_{a\to0^+} \frac{1}{\sqrt{2a}}e^{-\frac{y^2}{4a}} = \begin{cases}
					0 & y \ne 0, \\
					\infty & y = 0.
				\end{cases}
			\end{equation}
			At the same time, for any $a > 0$,
			\begin{equation}
				\int_{-\infty}^\infty \frac{1}{\sqrt{2a}}e^{-\frac{y^2}{4a}}\dee y = \frac{1}{\sqrt{2a}}\sqrt{4a\pi} = \sqrt{2\pi},
			\end{equation}
			so it makes sense formally that we should have $\fourier(1)(y) = \sqrt{2\pi}\delta(y)$.
			
			Now, if we consider applying the Fourier transform twice to a function $f$, we get
			\begin{align}
				\fourier\fourier(f)(y) &= \frac{1}{2\pi}\int_{-\infty}^\infty\int_{-\infty}^\infty e^{ixy}e^{izx} f(z)\dee z\dee x \\
				&= \frac{1}{\sqrt{2\pi}}\int_{-\infty}^\infty f(z) \frac{1}{\sqrt{2\pi}}\int_{-\infty}^\infty e^{ix(y+z)}\dee x\dee z\\
				&= \frac{1}{\sqrt{2\pi}}\int_{-\infty}^\infty f(z) \fourier(1)(y+z)\dee z \\
				&= \int_{-\infty}^\infty f(z)\delta(y+z)\dee z \\
				&= \int_{-\infty}^\infty f(z-y)\delta(z)\dee z \\
				&= f(-y).
			\end{align}
		\end{proof}
		
		\questionpart Define $g(y) = f(-y)$ for some function $f$. Based on the formal result from part 1.3), we see immediately that
		\begin{equation}
			\fourier^4(f)(y) = \fourier^2(\fourier^2(f))(y) = \fourier^2(g)(y) = g(-y) = f(y).
		\end{equation}
		Since $f$ was arbitrary, it follows formally that $\fourier^4 = I$, the identity operator.
		
		\questionpart Let $p(x) = x^4$. By the Spectral Mapping Theorem,
		\begin{equation}
			p(\sigma(\fourier)) = \sigma(p(\fourier)).
		\end{equation}
		Since $p(\fourier) = \fourier^4 = I$, the spectrum of $p(\fourier)$ is just $\sigma(I) = \{1\}$, as the operator $I - \lambda I = (1-\lambda)I$ is invertible, with inverse $\frac{1}{1-\lambda}I$, if and only if $\lambda \ne 1$. Therfore, if $\lambda \in \sigma(\fourier)$, then $p(\lambda) = 1$, that is, $\lambda^4 = 1$. The possible solutions of this equation are $1,-1,i,-i$, so $\sigma(\fourier) \subseteq \{1,-1,i,-i\}$.
		
		\questionpart If we reuse the result in equation (\ref{eq:gaussian_fourier_transform}) with $a = \frac{1}{2}$, we see that if $f(x) = e^{-\frac{1}{2}x^2}$, then
		\begin{equation}
			\fourier(f)(y) = e^{-\frac{1}{2}y^2}
		\end{equation}
		as well. Thus, $\fourier f = f$, so $f$ is an eigenfunction of $\fourier$ with corresponding eigenvalue $1$.
	\end{arabicparts}
\end{document}