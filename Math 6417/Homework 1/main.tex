\documentclass{homework}
\usepackage{enumitem}

\newcommand{\hwclass}{Math 6108}
\newcommand{\hwname}{Jacob Hauck}
\newcommand{\hwtype}{Homework}

\newcommand{\R}{\textbf{R}}
\newcommand{\dee}{\;\text{d}}
\newcommand{\eps}{\varepsilon}
\newcommand{\pl}[2]{\frac{\partial #1}{\partial #2}}
\newcommand{\dl}[2]{\frac{\text{d} #1}{\text{d} #2}}
\newcommand{\sgn}{\text{sgn}}
\newcommand{\bigoh}{\mathcal{O}}

\newcommand{\hwnum}{1}

\begin{document}
	\maketitle
	
	\question Let $f$ be continuous on $[0,1] \times \R$ and satisfy $|f(x,u) - f(x, v)| \le L |u-v|$ for all $x \in [0,1]$ and $u,v \in \R$, where $0 \le L < 8$.
	
	For $\alpha, \beta \in \R$, consider the boundary value problem
	\begin{equation}
		\label{eq:bvp}
		\begin{gathered}
			-u''(x) = f(x,u(x)) \quad \text{if } x \in (0,1) \\
			u(0)=\alpha \qquad u(1) = \beta.
		\end{gathered}
	\end{equation}
	This problem has one and only one solution $u \in C^2[0,1]$.
	
	Indeed, define
	\begin{equation}
		G(x,\xi) = \begin{cases}
			\xi(1-x) & 0 \le \xi \le x \le 1 \\
			x(1-\xi) & 0 \le x \le \xi \le 1
		\end{cases}
	\end{equation}
	and also consider the integral equation
	\begin{equation}
		\label{eq:integral}
		u(x) = \alpha (1-x) + \beta x + \int_0^1G(x,\xi)f(\xi, u(\xi))\dee\xi\quad\text{if } x\in [0,1].
	\end{equation}
	We show that if $u \in C^2[0,1]$, then $u$ solves (\ref{eq:bvp}) if and only if $u$ solves (\ref{eq:integral}), and that there is a unique solution $u \in C^2[0,1]$ of (\ref{eq:integral}) by the Banach Fixed Point Theorem. Then the claim follows.
	
	\begin{enumerate}[label=\textbf{(\roman*)}]
		\questionpart If $u \in C^2[0,1]$, then $u$ is a solution of (\ref{eq:bvp}) if and only if $u$ is a solution of (\ref{eq:integral}).
		\begin{proof}
			Suppose that $u \in C^2[0,1]$ is a solution of (\ref{eq:bvp}). Then, using integration by parts,
			\begin{alignat*}{2}
				\int_0^1G(x,\xi)f(\xi, u(\xi))\dee\xi &{}={}&& -\int_0^x \xi(1-x)u''(\xi)\dee\xi - \int_x^1 x(1-\xi)u''(\xi)\dee\xi \\
				&{}={}&& -(1-x)\left[\xi u'(\xi)\Big\vert_0^x - \int_0^xu'(\xi)\dee\xi\right] - x\left[(1-\xi)u'(\xi)\Big\vert_x^1 + \int_x^1u'(\xi)\dee\xi\right] \\[0.5em]
				&{}={}&& -(1-x)xu'(x) + (1-x)(u(x) - u(0)) \\ 
				&&&{}+ x(1-x)xu'(x) - x(u(1) - u(x)) \\[0.5em]
				&{}={}&& -\alpha(1-x) - \beta x + u(x)
			\end{alignat*}
			for any $x \in [0,1]$. Therefore, $u$ solves (\ref{eq:integral}).
			
			Conversely, suppose that $u \in C^2[0,1]$ is a solution of (\ref{eq:integral}). Then differentiating both sides of (\ref{eq:integral}) implies that
			\begin{equation}
				\label{eq:diff_u1}
				u'(x) = \beta - \alpha + \frac{\text{d}}{\text{d} x}\int_0^x \xi(1-x)f(\xi, u(\xi))\dee\xi + \frac{\text{d}}{\text{d} x}\int_x^1 x(1-\xi)f(\xi,u(\xi))\dee\xi
			\end{equation}
			for $x \in (0,1)$. Since the integrands in both integrals above are obviously continuous and have a continuous partial derivative with respect to $x$ on $[0,1]^2$, the action of the derivative on the integrals gives
			\begin{align*}
				\label{eq:diff_u2}
				u'(x) &= \beta - \alpha + x(1-x)f(x,u(x)) -\int_0^x\xi f(\xi,u(\xi))\dee\xi - x(1-x)f(x,u(x))+\int_x^1(1-\xi)f(\xi,u(\xi))\dee\xi \\
				&= \beta - \alpha - \int_0^x\xi f(\xi,u(\xi))\dee\xi + \int_x^1(1-\xi)f(\xi,u(\xi))\dee\xi \tag{\arabic{equation}}\stepcounter{equation}
			\end{align*}
			for $x \in(0,1)$. Since $f$ is continuous, the integrands in the above integrals are continuous, and, upon differentiating both sides again, the Fundamental Theorem of Calculus implies that
			\begin{equation}
				\label{eq:diff_u3}
				u''(x) = -xf(x,u(x)) - (1-x)f(x,u(x)) = -f(x,u(x))
			\end{equation}
			for $x \in (0,1)$. Lastly, note that the definition of $G$ implies that $G(0,\xi) = 0 = G(1,\xi)$ for all $\xi \in [0,1]$. Thus, $u(0) = \alpha$, and $u(1) = \beta$, so $u$ solves (\ref{eq:bvp}).
		\end{proof}
		\questionpart There is one and only one solution $u \in C^2[0,1]$ of (\ref{eq:integral}).
		\begin{proof}
			First, note that $G$ is continuous on $[0,1]^2$. Indeed, it is obviously continuous on the regions $\{x < \xi\}$ and $\{\xi < x\}$ by definition, and we have
			\begin{equation}
				\lim_{\substack{(x,\xi)\to(x_0,x_0)\\x \le \xi}} G(x,\xi) = x_0(1-x_0) = \lim_{\substack{(x,\xi)\to (x_0,x_0)\\x \ge \xi}} G(x,\xi)
			\end{equation}
			for any $x_0 \in [0,1]$. Thus, $G$ is continuous on $\{x=\xi\}$ as well, and, consequently, on all of $[0,1]^2$.
			
			Second, for $u\in C[0,1]$, define
			\begin{equation}
				Au(x) = \alpha(1-x) + \beta x + \int_0^1G(x,\xi)f(\xi, u(\xi))\dee\xi.
			\end{equation}
			Since $f$ and $u$ are both continuous, it follows that $f(\cdot, u(\cdot))$ is continuous and therefore bounded on $[0,1]$ by, say, $M > 0$. Then
			\begin{align*}
				\left|\int_0^1 G(x,\xi)f(\xi, u(\xi))\dee\xi - \int_0^1G(y,\xi)f(\xi,u(\xi))\dee\xi\right| &\le M \int_0^1|G(x,\xi) - G(y,\xi)|\dee\xi \\
				&\le M\left[\int_0^x  \xi|x-y|\dee\xi + \int_x^1 |x-y|(1-\xi)\dee\xi\right] \\
				&\le 2M |x-y|
			\end{align*}
			Hence, $Au$ is the sum of a polynomial and a Lipschitz function, so $Au \in C[0,1]$, and $A : C[0,1] \to C[0,1]$.
			
			Third, $A$ is a contraction on $C[0,1]$ in the uniform metric $\rho$ on $C[0,1]$. Indeed, for $u, v \in C[0,1]$,
			\begin{align}
				\rho(Au,Av) &= \max_{x\in[0,1]}\left|\int_0^1G(x,\xi)\big[f(\xi, u(\xi))-f(\xi,v(\xi))\big]\dee\xi\right| \\
				&\le \max_{x\in[0,1]} L\int_0^1|G(x,\xi)| \cdot|u(\xi) - v(\xi)|\dee\xi \\
				&\le L \cdot\left(\max_{x\in[0,1]}\int_0^1 |G(x,\xi)|\dee\xi\right)\rho(u,v).
			\end{align}
			By the Extreme Value Theorem,
			\begin{equation}
				\begin{aligned}
					p(x) = \int_0^1|G(x,\xi)|\dee\xi &= \int_0^x \xi(1-x)\dee\xi + \int_x^1x(1-\xi)\dee\xi = \frac{1}{2}\left[x^2(1-x) + x(1-x)^2\right] \\
					&= \frac{1}{2}x(1-x)
				\end{aligned}
			\end{equation}
			achieves its maximum for $x \in [0,1]$ either when $x \in \{0,1\}$, which implies $p(x) = 0$, or else when 
			\begin{equation}
				0 = p'(x) = \frac{1}{2}(1-x - x)
			\end{equation}
			that is, when $x = \frac{1}{2}$, in which case $p(x) = \frac{1}{8}$. Thus, $p(x) \le \frac{1}{8}$ for $x \in [0,1]$, and
			\begin{equation}
				\rho(Au,Av) \le 8L \rho(u,v).
			\end{equation}
			Since $8L < 1$ by hypothesis, it follows that $A$ is a contraction on $C[0,1]$.
			
			Fourth, by the Banach Fixed Point Theorem, there is a unique fixed point $u \in C[0,1]$ of $A$. By the definition of $A$, however, $u$ is a fixed point of $A$ if and only if it is a solution of (\ref{eq:integral}). Thus, (\ref{eq:integral}) has a unique solution $u \in C[0,1]$. 
			
			Since $C^2[0,1] \subseteq C[0,1]$, it follows that if $u \in C^2[0,1]$, then (\ref{eq:integral}) has a unique solution in $C^2[0,1]$, namely, $u$. Thus, to finish the proof, we need to show that $u'$ and $u''$ exist and are continuous.
			
			The calculations on the right-hand sides of (\ref{eq:diff_u1}, \ref{eq:diff_u2}, \ref{eq:diff_u3}) relied only on the fact that $u$ was continuous (so that $f(\cdot, u(\cdot))$ would be continuous) and solved (\ref{eq:integral}), so they apply to $u$ here as well. Thus, $u'$ and $u''$ exist, and
			\begin{equation}
				u''(x) = -f(x, u(x)),
			\end{equation}
			which is continuous on $[0,1]$. Therefore $u \in C^2[0,1]$.
		\end{proof}
	\end{enumerate}
	
	\question Let $u(x,t)$ be a smooth solution of the generalized heat equation
	\begin{equation}
		\label{eq:generalized_heat}
		\begin{aligned}
			u_t - \nabla \cdot (A(x) \nabla u) &= 0,& &\qquad (x,t) \in \Omega\times(0,\infty) \\
			u \big\vert_{t=0} &= u_0,& &\qquad x \in \Omega
		\end{aligned}
	\end{equation}
	where $\Omega \subset \R^n$ is a smooth bounded domain, $A : \R^n \to \R^{n\times n}$ is a positive definite matrix function, and $u_0 \in L^\infty(\Omega)$.
	
	 
	\begin{enumerate}[label=\textbf{(\roman*)}]
	 	\questionpart If $u\big\vert_{\partial\Omega} = 0$, then
	 	\begin{equation}
	 		\lVert u(\cdot, t) \rVert_{L^\infty} \le \lVert u_0\rVert_{L^\infty}
	 	\end{equation}
 		\begin{proof}
	 		Applying the vector calculus identity $\nabla \cdot (\phi B) = \nabla \phi^T B + \phi\nabla \cdot B$, where $\phi : \R^n \to \R$ is a differentiable scalar function, and $B : \R^n \to \R^n$ is a differentiable vector function, to the quantity $u^{2k-1}A(x)\nabla u$, where $k \ge 1$ is an integer, we obtain the identity
	 		\begin{equation}
	 			\label{eq:key_identity}
	 			\nabla\cdot(u^{2k-1}A(x)\nabla u) = (2k-1)u^{2(k-1)}\nabla u^TA(x)\nabla u + u^{2k-1}\nabla\cdot(A(x)\nabla u)
	 		\end{equation}
	 		Multiplying both sides of (\ref{eq:generalized_heat}) by $u^{2k-1}$ and integrating both sides over $\Omega$ gives
	 		\begin{equation}
	 			\int_\Omega u^{2k-1}u_t - \nabla \cdot (A(x)\nabla u) \dee x = 0
	 		\end{equation}
	 		for $t > 0$. Using (\ref{eq:key_identity}) and the fact that $\pl{(u^{2k})}{t} = 2ku^{2k-1}u_t$, we get
	 		\begin{equation}
	 			\label{eq:key_integral}
	 			\int_\Omega \pl{(u^{2k})}{t} \dee x = 2k\int_\Omega \nabla\cdot(u^{2k-1}A(x)\nabla u)\dee x - 2k(2k-1)\int_\Omega u^{2(k-1)}\nabla u^TA(x)\nabla u \dee x.
	 		\end{equation}
			Since $A(x)$ is positive definite by hypothesis, the integrand of the second term on RHS(\ref{eq:key_integral}) is pointwise nonnegative; hence, the entire second term is nonpositive because $k \ge 1$. Applying the Divergence Theorem to the first term, we obtain the inequality
			\begin{equation}
				\label{eq:key_inequality}
				\int_\Omega \pl{(u^{2k})}{t} \dee x \le 2k \int_{\partial\Omega} u^{2k-1}A(x)\nabla u\cdot \textbf{n} \dee S.
			\end{equation}
			where $S$ is the surface measure on $\partial \Omega$, and $\textbf{n}$ is the outward unit normal vector to $\Omega$. Using the assumptions $u\big\vert_{\partial\Omega} = 0$ and $k \ge 1$, we see that the integrand in RHS(\ref{eq:key_inequality}) is equal to 0 over the domain of integration $\partial \Omega$. Hence, $\text{RHS}(\ref{eq:key_inequality}) = 0$. Since $u$ is smooth, the time derivative commutes with the integral on LHS(\ref{eq:key_inequality}), so we deduce that
			\begin{equation}
				\dl{}{t}\lVert u(\cdot,t)\rVert_{L^{2k}}^{2k} = \dl{}{t} \int_\Omega u^{2k}\dee x = \int_\Omega \pl{(u^{2k})}{t}\dee x \le 0
			\end{equation}
			for $t > 0$. This implies that
			\begin{equation}
				\lVert u(\cdot, t) \rVert_{L^{2k}}^{2k} \le \lVert u(\cdot, 0)\rVert_{L^{2k}}^{2k} \iff \lVert u(\cdot, t)\rVert_{L^{2k}} \le \lVert u_0\rVert_{L^{2k}}
			\end{equation}
			for $t > 0$ and $k \ge 1$ an integer. Taking the limit as $k \to \infty$ on both sides and applying Proposition 2.18 from Arbogast and Bona, we obtain the desired result.
		\end{proof}
			
		\questionpart Suppose that $u \big\vert_{\partial\Omega} = g$, a nonzero, smooth function on $\partial\Omega$. Let $v$ be a smooth solution of the equation
		\begin{equation}
			\label{eq:generalized_laplace}
			\begin{aligned}
				\nabla\cdot(A(x)\nabla v) &= 0, \qquad x\in \Omega \\
				v \big\vert_{\partial\Omega} &= g.
			\end{aligned}
		\end{equation}
		Then $u-v$ is a smooth solution of (\ref{eq:generalized_heat}) such that $u-v\big\vert_{\partial\Omega} = 0$. Hence, by the previous problem,
		\begin{equation}
			\lVert u(\cdot,t)-v\rVert_{L^\infty} \le \lVert u(\cdot, 0) - v\rVert_{L^\infty} = \lVert u_0 - v\rVert_{L^\infty}
		\end{equation}
		for $t > 0$. Interpreting this inequality, we might say that $u$ does not deviate from $0$ by no more than the initial value $u_0$ in $L^\infty$ norm (the previous situation) but, rather, that $u$ deviates from the \textit{equilibrium} $v$ by no more than than the initial value $u_0$ in $L^\infty$ norm.
	\end{enumerate}

\end{document}