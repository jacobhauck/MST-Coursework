\documentclass{homework}
\usepackage{enumitem}

\newcommand{\hwclass}{Math 6108}
\newcommand{\hwname}{Jacob Hauck}
\newcommand{\hwtype}{Homework}

\newcommand{\R}{\textbf{R}}
\newcommand{\dee}{\;\text{d}}
\newcommand{\eps}{\varepsilon}
\newcommand{\pl}[2]{\frac{\partial #1}{\partial #2}}
\newcommand{\dl}[2]{\frac{\text{d} #1}{\text{d} #2}}
\newcommand{\sgn}{\text{sgn}}
\newcommand{\bigoh}{\mathcal{O}}

\usepackage{amsthm}
\newtheorem{lemma}{Lemma}

\newcommand{\hwnum}{3}

\begin{document}
	\maketitle
	
	\question
	Let $B(\cdot,\cdot)$ be a continuous, bilinear form on a real Hilbert space $H$. Suppose that $B$ is coercive in the sense that there is some $\alpha > 0$ such that $|B(x,x)| \ge \alpha\lVert x\rVert^2$ for all $x \in H$.
	\begin{arabicparts}
		\questionpart 
		Let $y \in H$. Then the map $f_y : H \to \R$ defined by $f_y(x) = B(x,y)$ is a bounded linear functional on $H$. Consequently, there exists a unique $w \in H$ such that $B(x,y) = f_y(x) = (x,w)$ for all $x \in H$.
		\begin{proof}
			Firstly, it is clear that $f_y$ is linear; indeed, given $a_1, a_2 \in \R$ and $x_1, x_2 \in H$,
			\begin{equation}
				f_y(a_1x_1+a_2x_2) = B(a_1x_1+a_2x_2,y) = a_1B(x_1,y) + a_2B(x_2,y) = a_1f_y(x_1) + a_2f_y(x_2)
			\end{equation}
			by the bilinearity of $B$.
			
			Secondly, $B(\cdot, y)=f_y$ must be continuous because $B$ is continuous. Hence, $f_y$ is bounded.
			
			Thirdly, by the Riesz representation theorem, there exists a unique $w \in H$ such that $B(x,y) = f_y(x) = (x,w)$ for all $x \in H$.
		\end{proof}
		
		\questionpart 
		Given $y \in H$, by 1.1), there is a unique $w$ such that $B(x,y) = (x,w)$ for all $x \in H$; this defines a function $A: H \to H$, where $Ay = w$. Then $A$ is a bounded, linear operator on $H$, that is, $A \in B(H)$.
		\begin{proof}
			Let $a_1, a_2 \in \R$ and $y_1, y_2 \in H$. Then for all $x \in H$,
			\begin{equation}
			\begin{aligned}
				(x, A(a_1y_1+a_2y_2)) &= B(x,a_1y_1 + a_2y_2) = a_1B(x,y_1) + a_2B(x,y_2) = a_1(x,Ay_1) + a_2(x,Ay_2) \\
				&= (x, a_1Ay_1 + a_2Ay_2).
			\end{aligned}
			\end{equation}
			Thus, $w=A(a_1y_1+a_2y_2)$ and $w' = a_1Ay_1 + a_2Ay_2$ satisfy the property that $B(x,a_1y_1+a_2y_2) = (x,w) = (x,w')$ for all $x \in H$. Since there is only one element of $H$ that can satisfy this property by the Riesz representation theorem, it follows that $w=w'$, that is, $A(a_1y_1 + a_2y_2) = a_1Ay_1 + a_2Ay_2$. Therefore, $A$ is linear.
			
			Note that $B$ is continuous if and only if (see, e.g., Theorem 8.10 assumption (a) in Arbogast and Bona) there exists some $M > 0$ such that
			\begin{equation}
				|B(x,y)| \le M\lVert x\rVert \lVert y\rVert, \quad \text{for all } x, y\in H.
			\end{equation}
			Let $y \in H$. Then
			\begin{equation}
				\lVert Ay\rVert = \left|\left(\frac{Ay}{\lVert Ay\rVert}, Ay\right)\right|=\left|B\left(\frac{Ay}{\lVert Ay\rVert},y \right)\right| \le M \lVert y\rVert.
			\end{equation}
			Since $y$ was arbitrary, it follows that $A$ is bounded, and $\lVert A \rVert \le M$. Thus, $A$ is a bounded, linear operator on $H$.
		\end{proof}
		
		\questionpart
		$A$ is bounded below in the sense that there exists $\gamma > 0$ such that $\lVert A y\rVert \ge \gamma \lVert y\rVert$ for all $y \in H$.
		\begin{proof}
			This follows from the coercivity of $B$: for all $y \in H$,
			\begin{equation}
				\lVert Ay\rVert \lVert y\rVert \ge |(y, Ay)| = |B(y, y)| \ge \alpha \lVert y \rVert^2,
			\end{equation}
			so $\lVert Ay\rVert \ge \alpha \lVert y\rVert$ for all $y \in H$, as claimed.
		\end{proof}
		
		\questionpart
		$A$ is one-to-one, and the range of $A$ is closed.
		\begin{proof}
			Let $y_1, y_2 \in H$, and suppose that $Ay_1 = Ay_2$. Then, by the previous part,
			\begin{equation}
				\lVert y_1 - y_2\rVert \le \frac{1}{\gamma}\lVert A(y_1 - y_2)\rVert = \frac{1}{\gamma}\lVert Ay_1 - Ay_2\rVert = 0.
			\end{equation}
			Therefore, $y_1 = y_2$. This shows that $A$ is one-to-one.
			
			Let $R(A)$ denote the range of $A$. To show that $R(A)$ is closed, we need to show that if $\{w_n\} \subset R(A)$ is any sequence that converges to $w$, then $w\in R(A)$. To this end, let $\{w_n\} \subseteq R(A)$ be a convergent sequence, and let $w$ be its limit. Since $w_n \in R(A)$ for all $n$, there exists $y_n \in H$ such that $w_n = Ay_n$ for all $n$. We can use the coercivity of $B$ to show that $\{y_n\}$ is convergent.
			
			Indeed, for all $m,n$ and all $x\in H$, the definition of $A$ implies that $|B(x,y_n-y_m)| = |(x,w_n-w_m)| \le \lVert x\rVert \lVert w_n - w_m\rVert$. Since $\{w_n\}$ converges, it is Cauchy; hence,
			\begin{align}
				&\forall \varepsilon > 0: \exists N: n,m > N \rightarrow \lVert w_n - w_m\rVert < \varepsilon \\
				\implies & \forall \varepsilon > 0: \exists N: n,m > N \rightarrow \forall x \in H : |B(x,y_n-y_m)| \le \lVert x \rVert \lVert w_n - w_m\rVert < \lVert x\rVert\varepsilon \\
				\implies & \forall \varepsilon > 0: \exists N: n,m > N \rightarrow \alpha \lVert y_n - y_m\rVert^2 \le |B(y_n-y_m, y_n-y_m)| < \lVert y_n -y_m\rVert\varepsilon \\
				\implies & \forall \varepsilon > 0: \exists N: n,m > N \rightarrow \lVert y_n -y_m\rVert < \frac{\varepsilon}{\alpha},
			\end{align}
			which implies that $\{y_n\}$ is Cauchy. Therefore, there exists $y \in H$ such that $y_n \to y$ as $n \to \infty$. Let $x \in H$, and let $\varepsilon > 0$ be given. By the continuity of $B$ and the inner product and the convergence of $\{y_n\}$ and $\{w_n\}$, there exists $n$ large enough that $|B(x,y-y_n)| < \frac{\varepsilon}{2}$, and $|(x,w-w_n)| < \frac{\varepsilon}{2}$. Then
			\begin{equation}
			\begin{aligned}
				|B(x,y) - (x,w)| &= |B(x,y-y_n) + B(x,y_n) - (x,w_n) -(x,w-w_n)| \\
				&\le |B(x,y-y_n)| + |(x,w-w_n)| < \varepsilon.
			\end{aligned}
			\end{equation}
			Since $\varepsilon> 0$ was arbitrary and $x \in H$ was arbitrary, it follows that $B(x,y) = (x,w)$ for all $x \in H$. This implies that $w = Ay$ by the definition of $A$, and $w \in R(A)$. Since the convergent sequence $\{w_n\}\subseteq R(A)$ was arbitrary, and its limit $w \in R(A)$, it follows that $R(A)$ is closed.
		\end{proof}
		
		\questionpart
		$A$ is onto.
		\begin{proof}
			Suppose that $x \in R(A)^\perp$, that is, $(x, w) = 0$ for all $w \in R(A)$. This implies that $(x, Ay) = 0$ for all $y \in H$, which is equivalent to saying that $B(x, y) = 0$ for all $y \in H$. In particular, if we choose $y = x$, then $\lVert x\rVert^2 \le \frac{1}{\alpha}|B(x, x)| = 0$. Therefore, $x = 0$. This shows that $R(A)^\perp = \{0\}$ because $x$ was arbitrary. 
			
			Let $y \in H$. Since $R(A)$ is a closed subspace of $H$ by (1.4), there exists a best approximation $w \in R(A)$ of $y$, which satisfies the property $(y - w, x) = 0$ for all $x \in R(A)$ (Theorem 3.7 and Corollary 3.8 in Arbogast and Bona). That is, $y-w \in R(A)^\perp$. Since $R(A)^\perp =\{0\}$ by the above, it follows that $y-w = 0$, and $y=w \in R(A)$. Since $y$ was arbitrary and $R(A) \subseteq H$, it follows that $R(A) = H$, that is, $A$  is onto.
		\end{proof}
		
		\questionpart
		$A$ is invertible.
		\begin{proof}
			By the previous two parts, $A$ is bijective, so it has a set-theoretic inverse function $A^{-1}$. By 1.2), $A$ is bounded. Therefore, by the open mapping theorem, $A$ maps open sets to open sets, which means that the preimage of an open set under $A^{-1}$ is open, that is, $A^{-1}$ is continuous. Therefore, $A$ is invertible.
		\end{proof}
		
		\questionpart
		Given $f \in H^{*}$, the Riesz representation theorem implies that there exists a unique $w \in H$ such that $f(x) = (x,w)$ for all $x \in H$, and we can view $H^{*}$ and $H$ as the same under the correspondence $f \leftrightarrow w$.
		
		\questionpart
		Consider the equation $B(x,y) = f(x)$ for all $x \in H$, where $f \in H^{*}$. By the remark in part 1.7), we can choose $w \in H$ such that $f(x) = (x,w)$ for all $x \in H$. Then the equation is equivalent to $B(x,y) = (x,w)$ for all $x \in H$. If $y$ is a solution of this equation, then, by the definition of $A$, we must have $Ay = w$. Using the invertibility of $A$, we obtain $y = A^{-1}w$ as the unique solution of the equation. Viewing $f$ and $w$ as the same under the correspondence in 1.7), we might also write $y = A^{-1}f$.
	\end{arabicparts}
	
	\question
	\newcommand{\ltwo}{L^2(-\pi,\pi)}
	\newcommand{\hm}{{H^{-1}}}
	Let $e_j \in \ltwo$ be defined by $e_j(x) = \frac{1}{\sqrt{2\pi}}e^{ijx}$ for $j \in \Z$. Define
	\begin{equation}
		H = \left\{f \in \ltwo : f = \bar{f}\text{ and }f = \sum_{j\ne 0}f_je_j \text{ for some } \{f_j\}\subseteq \C \text{ such that } \sum_{j\ne 0} j^2|f_j|^2 < \infty \right\},
	\end{equation}
	and
	\begin{equation}
		\hm = \left\{f(x) = \sum_{j\ne 0}f_je_j(x) : \sum_{j\ne 0}j^{-2}|f_j |^2 < \infty \text{ and } f = \bar{f}\right\}.
	\end{equation}
	\begin{arabicparts}
		\questionpart
		$H$ and $\hm$ are real Hilbert spaces when equipped with the inner products
		\begin{equation}
			(f,g)_H = \sum_{j\ne 0}j^2f_j\bar{g}_j,\qquad(f,g)_\hm = \sum_{j\ne 0}j^{-2}f_j\bar{g}_j.
		\end{equation}
		
		\begin{proof}
		Before we can show that $H$ and $\hm$ are real Hilbert spaces, we need to make sense of their definitions (and of the definitions of the inner products). For the definition $H$, there isn't much trouble -- its elements are elements of $\ltwo$. For the definition $H^{-1}$, the series might not converge to an element of $\ltwo$ (say, $f_j = 1$ for all $j$), so it's not clear in what sense the convergence of the series should be taken (at least, I don't feel like working it out here, as it won't be critical to the rest of these exercises). Nevertheless, we can understand $H$ in terms of a corresponding space of coefficient sequences, which is isomorphic to $H$ as a real vector space. By analogy, then, we can understand $H^{-1}$ in terms of its corresponding space of coefficient sequences, which will suffice for our purposes. We will come back to the inner products later.
		
		To understand the spaces of coefficient sequences, we introduce the following Lemma showing that $H$ is a real vector space isomorphic to its space of coefficient sequences.
		\begin{lemma}
		Define
		\begin{equation}
			S_H = \left\{\{f_j\}_{j \ne 0} \subseteq \C : \sum_{j\ne 0}j^2|f_j|^2 < \infty \textnormal{ and } \forall j \in \Z \setminus\{0\}, f_j = \bar{f}_{-j} \right\}.
		\end{equation}
		Then $S_H$ is a real vector space, and there is an isomorphism (of real vector spaces) $\varphi : H \to S_H$ such that if $\varphi(f) = \{f_j\}$, then
		\begin{equation}
			f = \sum_{j\ne 0}f_je_j.
		\end{equation}
		
		\newcommand{\basis}{\mathcal{B}}
		\newcommand{\n}{\frac{1}{\sqrt{2\pi}}}
		\begin{proof}
			We need to prove that $\varphi$ is well-defined, bijective, and, after showing that $H$ and $S_H$ are real vector spaces, that $\varphi$ is a linear mapping between them.
			
			\textbf{Step 1: definition of $\varphi$}
			
			Let $f \in H$. Then $f \in \ltwo$. Recalling from our lecture that $\basis = \left\{e_j\right\}_{j\in\Z}$ is an orthonormal basis for $\ltwo$, it follows that there is exactly one sequence of coefficients $\{g_j\}\subseteq\C$ such that
			\begin{equation}
				f = \sum_{j\in\Z}g_je_j.
			\end{equation}
			Since there must be \textit{some} sequence of coefficients $\{f_j\} \subseteq\C$ such that
			\begin{equation}
				f = \sum_{j\ne 0}f_je_j,\qquad \sum_{j\ne0}j^2|f_j|^2,
			\end{equation}
			it follows by the uniqueness of $\{g_j\}$ that $g_0 = 0$, and $g_j = f_j$ for all $j \ne 0$. Furthermore, since $f = \bar{f}$, it follows that
			\begin{equation}
				\bar{f}(x) = \n\overline{\sum_{j\ne 0}f_je^{ijx}} = \n\sum_{j\ne 0}\bar{f}_je^{-ijx}=\n\sum_{j\ne0}\bar{f}_{-j}e^{ijx} = f(x).
			\end{equation}
			That is, $f = \sum\limits_{j\ne 0}\bar{f}_{-j}e_j$; by the uniqueness of $\{g_j\}$ again, we must have $\bar{f}_{-j} = g_j = f_j$ for all $j \ne 0$. Hence, $\{f_j\} \in S_H$. Since $\{f_j\}$ is uniquely determined by $f$ by the uniqueness of $\{g_j\}$, it follows that we can define a function $\varphi : H \to S_H$ by $\varphi(f) = \{f_j\}$.
			
			\textbf{Step 2: $\varphi$ is a one-to-one correspondence}
			
			First, suppose that $\varphi(f) = \{f_j\} = \varphi(g)$ for some $f,g \in H$ and $\{f_j\} \in S_H$. Then, by the definition of $\varphi$,
			\begin{equation}
				f = \sum_{j\ne 0}f_je_j = g,
			\end{equation}
			so $\varphi$ is one-to-one.
			
			Second, let $\{f_j\} \in S_H$. Since $\ltwo$ is a Hilbert space, its Riesz-Fischer map $F: \ltwo \to \ell^2(\Z)$ corresponding to the orthonormal basis $\basis$ is an isomorphism. If we set $g_j = f_j$ for $j \ne 0$, and $g_0 = 0$, then we have
			\begin{equation}
				\sum_{j\in\Z}|g_j|^2 = \sum_{j\ne 0}|f_j|^2 \le \sum_{j\ne 0}j^2|f_j|^2 < \infty
			\end{equation}
			because $\{f_j\} \in S_H$. Therefore, $\{g_j\} \in \ell^2(\Z)$, and $f = F^{-1}(\{g_j\}) \in \ltwo$. By the definition of $F$, we have
			\begin{equation}
				f = F^{-1}(\{g_j\}) = \sum_{j\in\Z}g_je_j = \sum_{j\ne 0}f_je_j.
			\end{equation}
			Since $\{f_j\} \in S_H$, we have $f_j = \bar{f}_{-j}$, so
			\begin{equation}
				\bar{f}(x) = \n\overline{\sum_{j\ne0}f_je^{ijx}} = \n\sum_{j\ne0}\bar{f}_je^{-ijx} = \n\sum_{j\ne0}\bar{f}_{-j}e^{ijx} = \n\sum_{j\ne0}f_je^{ijx} = f(x),
			\end{equation}
			so $f = \bar{f}$. Since $\{f_j\}\in S_H$, we also have $\sum\limits_{j\ne0}j^2|f_j|^2 < \infty$. Therefore, $f \in H$ by definition.
			
			Finally, $\varphi(f) = \{f_j\}$ because $f = \sum\limits_{j\ne 0}f_je_j$, and $\varphi(f)$ is, by definition, the unique sequence of coefficients in $S_H$ that make that statement true. 
			
			Thus, $\varphi$ is one-to-one and onto.
			
			\textbf{Step 3: $H$ and $S_H$ are real vector spaces}
			
			$H$ is a nonempty subset (it contains 0) of the real vector space $\ltwo$, and $S_H$ is a nonempty subset (it contains 0) of the real vector space $\C^{\Z \setminus \{0\}}$ (the space of all sequences of complex numbers). Thus, it suffices to show that $H$ and $S_H$ are subspaces of these two vector spaces. 
			
			Let $\{f_j\}, \{g_j\} \in S_H$, and let $\alpha,\beta\in\R$. Then $\{h_j\} = \alpha\{f_j\} + \beta\{g_j\} \in S_H$ because $\bar{h}_{-j} = \alpha\bar{f}_{-j} + \beta\bar{g}_{-j} = \alpha f_j + \beta g_j = h_j$, and
			\begin{align}
				\sum_{j\ne 0}j^2|h_j|^2 &= \sum_{j\ne0}j^2|\alpha f_j +\beta g_j|^2 = \sum_{j\ne 0}j^2(\alpha^2|f_j|^2 + \alpha\beta(f_j\bar{g}_j + \bar{f}_jg_j) +\beta^2|g_j|^2)\\
				&\le \alpha^2\sum_{j\ne 0}j^2|f_j|^2 + 2\alpha\beta\left(\sum_{j\ne0}j^2|f_j|^2\right)^\frac{1}{2}\left(\sum_{j\ne0}j^2|g_j|^2\right)^\frac{1}{2} + \beta^2\sum_{j\ne0}j^2|g_j|^2 < \infty
			\end{align}
			by the Cauchy-Schwarz inequality. Thus, $S_H$ is a real vector subspace of the space of all sequences of complex numbers.
			
			Now let $f,g \in H$, and $\alpha,\beta\in\R$. Set $\{f_j\} = \varphi(f)$, and $\{g_j\} = \varphi(g)$. Then $\{h_j\} = \alpha\{f_j\} + \beta\{g_j\}\in S_H$, so we can define $h =\varphi^{-1}(\{h_j\})$. By the definition of $h$, we have
			\begin{equation}
				\alpha f + \beta g = \alpha \sum_{j\ne 0}f_je_j + \beta\sum_{j\ne 0}g_je_j = \sum_{j\ne 0}(\alpha f_j + \beta g_j)e_j = h \in H.
			\end{equation}
			Therefore, $H$ is a (real) vector subspace of $\ltwo$.
			
			\textbf{Step 4: $\varphi$ is linear}
			
			Let $f,g\in H$, and let $\alpha,\beta\in \R$. Define $\{f_j\} = \varphi(f)$, and $\{g_j\} = \varphi(g)$. Then
			\begin{equation}
				\alpha f + \beta g = \alpha \sum_{j\ne 0}f_je_j + \beta\sum_{j\ne0}g_je_j = \sum_{j\ne0}(\alpha f_j + \beta g_j)e_j.
			\end{equation}
			This implies that $\alpha\varphi(f) + \beta\varphi(g) = \{\alpha f_j + \beta g_j\} = \varphi(\alpha f + \beta g)$ by the definition of $\varphi$. Thus, $\varphi$ is linear.
		\end{proof}
		\end{lemma}
		The Lemma establishes that $H$ is essentially the same as $S_H$ as a vector space. If we define by analogy
		\begin{equation}
			S_\hm = \left\{\{f_j\}_{j\ne0}\subseteq \C : \sum_{j\ne0}j^{-2}|f_j|^2 < \infty \text{ and } \forall j\in\Z\setminus\{0\}, f_j = \bar{f}_{-j}\right\},
		\end{equation}
		then we can understand $\hm$ to be a real vector space isomorphic to $S_\hm$. Note that $S_\hm$ is indeed a real vector space; like $S_H$, it is a nonempty subset (it contains 0) of the vector space of all sequences of complex numbers, and if $\{f_j\}, \{g_j\} \in S_\hm$, and $\alpha,\beta\in\R$, then $\{h_j\} = \alpha\{f_j\} + \beta\{g_j\} \in S_H$ because $\bar{h}_{-j} = \alpha\bar{f}_{-j} + \beta\bar{g}_{-j} = \alpha f_j + \beta g_j = h_j$, and
		\begin{align}
			\sum_{j\ne 0}j^2|h_j|^2 &= \sum_{j\ne0}j^{-2}|\alpha f_j +\beta g_j|^2 = \sum_{j\ne 0}j^{-2}(\alpha^2|f_j|^2 + \alpha\beta(f_j\bar{g}_j + \bar{f}_jg_j) +\beta^2|g_j|^2)\\
			&\le \alpha^2\sum_{j\ne 0}j^{-2}|f_j|^2 + 2\alpha\beta\left(\sum_{j\ne0}j^{-2}|f_j|^2\right)^\frac{1}{2}\left(\sum_{j\ne0}j^{-2}|g_j|^2\right)^\frac{1}{2} + \beta^2\sum_{j\ne0}j^{-2}|g_j|^2 < \infty
		\end{align}
		by the Cauchy-Schwarz inequality. From here on, we will assume that there exists an isomorphism (of real vector spaces) $\psi: \hm \to S_\hm$.
		
		Now we turn to the issue of equipping $H$ and $\hm$ with inner products. The given inner products are defined in terms of the sequence representations of elements of $H$ and $\hm$; this is well-defined due to the (actual) isomorphism between $H$ and $S_H$ and the (assumed) isomorphism between $\hm$ and $S_\hm$. It also allows us to work with $S_\hm$ without needing to worry about how to interpret its elements -- we only need to work with the well-defined sequence representations in $S_\hm$.
		
		We need to show that $H$ and $\hm$ are real inner product spaces when equipped with $(\cdot,\cdot)_H$ and $(\cdot,\cdot)_\hm$ as inner products. We wrap this into the following Lemma.
		\begin{lemma}
			For $G \in \{H, \hm\}$, define 
			\begin{equation}
				\rho = \rho_G = \begin{cases}
					\varphi & G = H, \\
					\psi & G = \hm,
				\end{cases}\qquad
				s = s_G = \begin{cases}
					1 & G = H, \\
					-1 & G = \hm.
				\end{cases}
			\end{equation}
			Then $G$ is a real inner product space with inner product $(\cdot,\cdot)_G$ defined by
			\begin{equation}
				(f,g)_G = \sum_{j\ne0}j^{2s}f_j\bar{g}_j, \qquad f,g \in G,
			\end{equation}
			where $\{f_j\} = \rho(f), \{g_j\}=\rho(g) \in S_G$ are the coefficients of $f$ and $g$ in $S_G$.
		\end{lemma}
		\begin{proof}
			As we have already remarked, the uniqueness of $\{f_j\}$ and $\{g_j\}$ implies that $(f,g)_G$ is well-defined. We still need to show that the series converges to a real number, and that $(\cdot,\cdot)_G$ is symmetric, linear in the first argument, and positive definite.
			
			\textbf{Step 1: $(f,g)_G$ is a real number}
			
			Let $f,g \in G$ with $\{f_j\} = \rho(f)$, and $\{g_j\} = \rho(g)$. Then the series for $(f,g)_G$ converges absolutely because
			\begin{equation}
				\sum_{j\ne 0}j^{2s}|f_j|\cdot|\bar{g}_j| \le \left(\sum_{j\ne0}j^{2s}|f_j|^2\right)^\frac{1}{2}\left(\sum_{j\ne0}j^{2s}|g_j|^2\right)^\frac{1}{2} < \infty
			\end{equation}
			by the Cauchy-Schwarz inequality.
			
			Next, $(f,g)_G \in \R$ because
			\begin{align}
				\overline{(f,g)_G} = \overline{\sum_{j\ne0}j^{2s}f_j\bar{g}_j} = \sum_{j\ne0}j^{2s}\bar{f}_jg_j = \sum_{j\ne0}j^{2s}f_{-j}\bar{g}_{-j} = \sum_{j\ne 0}j^{2s}f_j\bar{g}_j = (f,g)_G,
			\end{align}
			and only real numbers are equal to their own complex conjugate.
			
			\textbf{Step 2: $(\cdot,\cdot)_G$ is symmetric}
			
			Let $f,g \in G$, and let $\{f_j\} = \rho(f)$, and $\{g_j\} = \rho(g)$. Then
			\begin{equation}
				(f,g)_G = \sum_{j\ne0}j^{2s}f_j\bar{g}_j = \sum_{j\ne0}j^{2s}f_{-j}\bar{g}_{-j} = \sum_{j\ne0}g_j\bar{f}_j = (g,f)_G.
			\end{equation}
			Therefore, $(\cdot,\cdot)_G$ is symmetric.
			
			\textbf{Step 3: $(\cdot,\cdot)_G$ is linear in the first argument}
			
			Let $f,g,h\in G$, and let $\{f_j\} = \rho(f)$, $\{g_j\} = \rho(g)$, and $\{h_j\} = \rho(h)$. Let $\alpha, \beta \in \R$. Then $\rho(\alpha f + \beta g) = \alpha\rho(f) + \beta\rho(g)$ because $\rho$ is linear, so
			\begin{equation}
				(\alpha f + \beta g, h)_G = \sum_{j\ne0}j^{2s}(\alpha f_j + \beta g_j)\bar{h}_j = \alpha\sum_{j\ne0}j^{2s}f_j\bar{h}_j + \beta \sum_{j\ne0}f_j\bar{h}_j = \alpha(f,h)_G + \beta(g,h)_G,
			\end{equation}
			that is, $(\cdot,\cdot)_G$ is linear in the first argument.
			
			\textbf{Step 4: $(\cdot,\cdot)_G$ is positive definite}
			
			Let $f \in G$, and let $\{f_j\} = \rho(f)$. Then
			\begin{equation}
				(f,f)_G = \sum_{j\ne 0} j^{2s}|f_j|^2 \ge 0
			\end{equation}
			because each term of the series is nonnegative. Moreover, if $f = 0$, then clearly $|f_j|^2=0$, so each term is 0, and $(f,f)_G = 0$. Conversely, if $(f,f)_G = 0$, then, since each term of the series for $(f,f)_G$ is nonnegative, it must be that each term is 0. This implies that $f_j = 0$ for all $j\ne 0$, that is, $\{f_j\} = 0$ in $S_G$. Therefore $f = \rho^{-1}(0) = 0$ by the linearity of $\rho^{-1}$.
			
			This shows that $(\cdot,\cdot)_G$ is positive definite.
		\end{proof}
		
		Now we know that $H$ and $\hm$ are real inner product spaces, so there is only one more thing to show in order to prove that they are real Hilbert spaces: they need to be complete with respect to the norms $\lVert\cdot\rVert_H$ and $\lVert\cdot\rVert_\hm$ induced by their inner products. For the sake of organization, we put this in one last Lemma.
		
		\begin{lemma}
			For $G \in \{H,H^{-1}\}$ and $s = s_G$, $\rho = \rho_G$ defined as in Lemma 2, the space $G$ is complete with respect to the norm $\lVert\cdot\rVert_G$ induced by the inner product $(\cdot,\cdot)_G$ from Lemma 2.
			
			\begin{proof}
				Let $\{f^n\}_{n=1}^\infty$ be a Cauchy sequence in $G$ with respect to $\lVert \cdot \rVert_G$. We need to show that there exists $f\in G$ such that $f^n \to f$ as $n \to \infty$ in $\lVert \cdot\rVert_G$. To do this, we first identify a candidate element, then we show that the sequence converges to the candidate.
				
				\textbf{Step 1: identifying a candidate limit}
				
				Given $\varepsilon > 0$, we can choose $N$ such that $n,m > N$ implies that
				\begin{equation}
					\varepsilon^2 > \lVert f^n - f^m\rVert_G^2 = \sum_{j\ne0}j^{2s}|f^n_j - f^m_j|^2,
				\end{equation}
				where $\{f^n_j\} = \rho(f^n)$. Since each term in the summation is nonnegative, this means that $|j^sf^n_j - j^sf^m_j| < \varepsilon$ for all $n, m > N$. Thenl $\{j^sf^n_j\}_{n=1}^\infty$ is Cauchy for all $j \ne 0$. Thus, by the completeness of $\C$, the sequence $\{j^sf^n_j\}$ converges to a limit $j^sf_j \in \C$ as $n \to\infty$.
				
				\newcommand{\idx}{\mathcal{J}_J}
				Since $j^s\bar{f}^n_j = j^sf^n_{-j}$ for all $j$ and all $n$, taking the limit as $n \to \infty$ and using the continuity of the complex conjugate function, we get $j^s\bar{f}_j = j^sf_{-j}$ for all $j\ne0$. Since $j^s \ne 0$, it follows that $\bar{f}_j = f_{-j}$ for all $j\ne 0$.
				
				Furthermore, by the convergence of $\{j^sf^n_j\}$ to $j^sf_j$, for all $J > 0$, we can choose $n$ large enough that $|j^sf_j - j^sf^n_j|^2 \le J^{-1}$ for all $0 < |j| \le J$. Then
				\begin{align}
					\sum_{0<|j|\le J}j^{2s}|f_j|^2 &= \sum_{0<j\le |J|}|j^sf_j - j^sf^n_j|^2 + j^{2s}f_j\bar{f}^n_j +j^{2s}\bar{f}_jf^n_j + j^{2s}|f^n_j|^2 \\
					&= \sum_{0<|j|\le J} |j^sf_j - j^sf^n_j|^2 + j^{2s}(f_j-f^n_j)\bar{f}^n_j + j^{2s}\overline{(f_j-f^n_j)}f^n_j + 3j^{2s}|f^n_j|^2\\
					&\le 2 + 2\left(\sum_{0<|j|\le J}|j^sf_j-j^sf^n_j|^2\right)^\frac{1}{2}\left(\sum_{0<|j|\le J}j^{2s}|f^n_j|^2\right)^\frac{1}{2} + 3\sum_{0<|j|\le J}j^{2s}|f^n_j|^2\\
					&\le 2 + 2\sqrt{2}\lVert f^n\rVert_G + 3\lVert f^n\rVert_G^2.
				\end{align}
				Since $\{f^n\}$ is Cauchy in $\lVert\cdot\rVert_G$, it must be bounded; that is, there exists $M > 0$ such that $\lVert f^n\rVert_G \le M$ for all $n$. Then
				\begin{equation}
					\sum_{0<|j|\le J}j^{2s}|f_j|^2 \le 2 + 2\sqrt{2}M + 3M^2
				\end{equation}
				for all $J > 0$. This implies that
				\begin{equation}
					\sum_{j\ne 0}j^{2s}|f_j|^2 \le 2 + 2\sqrt{2}M+3M^2 < \infty.
				\end{equation}
				Therefore, $\{f_j\} \in S_G$, and we can define our candidate limit as $f = \rho^{-1}(\{f_j\})$.
				
				\textbf{Step 2: showing that the candidate is the limit}
				
				Let $\varepsilon > 0$ be given. Since $\{f^n\}$ is Cauchy, we can choose $N$ such that $n,m > N$ implies that $\lVert f^n - f^m\rVert_G^2 < \varepsilon$. By the convergence of $\{j^sf^n_j\}$, for any $J > 0$, we can choose $m_J > N$ such that for all $0 < |j| \le J$ we have $|j^sf_j-j^sf^{m_J}_j|^2 < \varepsilon J^{-1}$. Then
				\newcommand{\lJ}{{0<|j|\le J}}
				\begin{align}
					\sum_\lJ j^{2s}|f_j - f^n_j|^2 &= \sum_\lJ |j^sf_j - j^sf^n_j| = \sum_\lJ |j^sf_j - j^sf^{m_J}_j + (j^sf^{m_J}_j - j^sf^n_j)|^2 \\
					&= \sum_\lJ |j^sf_j - j^sf^{m_J}_j|^2 + (j^sf_j-j^sf^{m_J}_j)\overline{(j^sf^{m_J}_j-j^sf^n_j)} \\
					&\qquad\qquad\quad{}+ \overline{(j^sf_j-j^sf^{m_J}_j)}(j^sf^{m_J}_j-j^sf^n_j) + |j^sf^{m_J}-j^sf^n_j|^2\\
					&\le 2\varepsilon + 2\left(\sum_\lJ |j^sf_j - j^sf^{m_J}_j|^2\right)^\frac{1}{2}\left(\sum_\lJ|j^sf^{m_J}_j-j^sf^n_j|^2\right)^\frac{1}{2} \\
					&\qquad{}+ \sum_\lJ |j^sf^{m_J}_j -j^sf^n_j|^2 \\
					&\le 2\varepsilon + 2\sqrt{2\varepsilon}\lVert f^{m_J} - f^n\rVert_G + \lVert f^{m_J} -f^n\rVert_G^2 \\
					&\le 2\varepsilon + 2\sqrt{2}\varepsilon + \varepsilon
					\le \left(3+2\sqrt{2}\right)\varepsilon.
				\end{align}
				Since this estimate is independent of $J$, it follows that
				\begin{equation}
					\lVert f - f^n\rVert_G^2 = \sum_{j\ne 0}j^{2s}|f_j-f^n_j| < \left(3+2\sqrt{2}\right)\varepsilon
				\end{equation}			
				whenever $n > N$. Therefore, $f^n \to f$ as $n\to \infty$ in $\lVert \cdot \rVert_G$.
			\end{proof}
		\end{lemma}
		
		The three Lemmas above establish that $H$ and $\hm$ are Hilbert spaces with the given inner products.
		\end{proof}
		
		\questionpart
		Define the bilinear form $B$ on $H$ by
		\begin{equation}
			B(f,g) = \sum_{j\ne0}(ij + j^2)f_j\bar{g}_j,
		\end{equation}
		where $\{f_j\} = \varphi(f)$, and $\{g_j\} = \varphi(g)$.
		Then $B$ satisfies the hypotheses of the Lax-Milgram theorem.
		
		\begin{proof}
			As with the inner products, the isomorphism $\varphi$ ensures that $B(f,g)$ is well-defined in terms of $\{f_j\} = \varphi(f)$ and $\{g_j\} = \varphi(g)$. We need to show that the series for $B$ converges and that $B$ is actually bilinear over $\R$. Then, we need to show that $B$ satisfies the hypotheses of the Lax-Milgram theorem, that is, that $B$ is continuous and coercive.
			
			\textbf{Step 1: $B$ is well-defined and bilinear}
			
			Let $f, g\in H$ with $\{f_j\} = \varphi(f)$ and $\{g_j\} = \varphi(g)$. Then the series for $B(f,g)$ converges absolutely because
			\begin{equation}
				\label{eq:B_bounded}
				\sum_{j\ne0}|(ij+j^2)f_j\bar{g}_j| \le \left(\sum_{j\ne0}j\sqrt{1+j^2}|f_j|^2\right)^\frac{1}{2}\left(\sum_{j\ne0}j\sqrt{1+j^2}|g_j|^2\right)^\frac{1}{2} \le \sqrt{2}\lVert f \rVert_H\lVert g\rVert_H < \infty
			\end{equation}
			by the Cauchy-Schwarz inequality (note that $|ij +j^2| = j\sqrt{1+j^2} \le \sqrt{2}j^2$ for all $j\ne 0$).
			
			Let $f,g,h\in H$, and let $\alpha,\beta \in \R$. Set $\{f_j\} = \varphi(f)$, $\{g_j\} = \varphi(g)$, and $\{h_j\} = \varphi(h)$. By the linearity of $\varphi$, we have $\varphi(\alpha f + \beta g) = \alpha \varphi(f)+ \beta\varphi(g)$; therefore,
			\begin{align}
				B(\alpha f + \beta g, h) &= \sum_{j\ne0}(ij + j^2)(\alpha f_j + \beta g_j)\bar{h}_j =\alpha\sum_{j\ne 0}(ij+j^2)f_j\bar{h}_j + \beta\sum_{j\ne0}(ij+j^2)g_j\bar{h}_j \\
				&= \alpha B(f,h) + \beta B(g,h).
			\end{align}
			Similarly,
			\begin{align}
				B(h, \alpha f + \beta g) &= \sum_{j\ne0}(ij + j^2)h_j\overline{(\alpha f_j + \beta g_j)} =\alpha\sum_{j\ne 0}(ij+j^2)h_j\bar{f}_j + \beta\sum_{j\ne0}(ij+j^2)h_j\bar{g}_j \\
				&= \alpha B(h,f) + \beta B(h,g).
			\end{align}
			Thus, $B$ is bilinear.
			
			\textbf{Step 2: $B$ is continuous and coercive}
			
			We have practically already shown that $B$ is continuous: by (\ref{eq:B_bounded}),
			\begin{equation}
				|B(f,g)| \le \sqrt{2}\lVert f\rVert_H\lVert g\rVert_H
			\end{equation}
			for all $f,g\in H$ with $\{f_j\} = \varphi(f)$ and $\{g_j\} = \varphi(g)$. This implies that $B$ is continuous because $B$ is bilinear.
			
			For coercivity, let $f \in H$, and observe that
			\begin{equation}
				|B(f,f)| = \left|\sum_{j\ne0}(ij+j^2)|f_j|^2\right| = \left(\left[\sum_{j\ne0}j|f_j|^2\right]^2 + \left[\sum_{j\ne0}j^2|f_j|^2\right]^2\right)^\frac{1}{2} \ge \lVert f\rVert_H.
			\end{equation}
			Thus, $B$ is coercive (note that the first summation converges absolutely by a comparison test, by the way: $|j||f_j|^2 \le j^2|f_j|^2$ for all integers $j \ne 0$).
		\end{proof}
		
		\questionpart
		$\hm \subseteq H^*$ if we assign $f \in \hm$ with $\{f_j\} = \psi(f)$ the action
		\begin{equation}
			f(g) = \sum_{j\ne0}f_j\bar{g}_j,
		\end{equation}
		where $\{g_j\}=\varphi(g)$.
		
		\begin{proof}
			Let $f\in H$. Note that because $H$ is a real Hilbert space, an element of $H^*$ should be a real-valued, bounded linear functional. Thanks to the isomorphisms $\varphi$ and $\psi$, the action of $f \in \hm$ is well-defined, but we still need to show that the series converges to a real number. Then, we need to show that $f$ is linear over $\R$ and bounded. Then we will have $f \in H^*$, completing the proof.
			
			\textbf{Step 1: $f(g)$ converges to a real number}
			
			Let $f \in \hm$ and $g\in H$, with $\{f_j\} = \psi(f)$, and $\{g_j\} = \varphi(g)$. Then the series for $f(g)$ converges absolutely because
			\begin{equation}
				\label{eq:f_bounded}
				\sum_{j\ne0}|f_j||\bar{g}_j| \le \left(\sum_{j\ne0}j^{-2}|f_j|^2\right)^\frac{1}{2}\left(\sum_{j\ne0}j^2|g_j|^2\right)^\frac{1}{2} = \lVert f\rVert_\hm\cdot\lVert g\rVert_H
			\end{equation}
			by the Cauchy-Schwarz inequality.
			
			Next, $f(g) \in \R$ because
			\begin{equation}
				\overline{f(g)}=\overline{\sum_{j\ne0}f_j\bar{g}_j} = \sum_{j\ne0}\bar{f}_jg_j = \sum_{j\ne0}f_{-j}\bar{g}_{-j} = \sum_{j\ne0}f_j\bar{g}_j = f(g),
			\end{equation}
			and only real numbers equal their own complex conjugate.
			
			\textbf{Step 2: $f$ is linear and bounded}
			
			Let $f \in \hm$ and $g,h \in H$ with $\{f_j\} = \psi(f)$, and $\{g_j\} = \varphi(g)$, $\{h_j\} = \varphi(h)$. Let $\alpha,\beta\in\R$. Then, by the linearity of $\varphi$, we have $\varphi(\alpha g + \beta h) = \alpha\varphi(g) + \beta\varphi(h)$. Hence,
			\begin{equation}
				f(\alpha g + \beta h) = \sum_{j\ne0} f_j\overline{(\alpha g_j + \beta h_j)} = \alpha\sum_{j\ne0}f_j\bar{g}_j + \beta\sum_{j\ne0}f_j\bar{h}_j = \alpha f(g) + \beta f(h).
			\end{equation}
			Therefore, $f$ is linear.
			
			We have practically already shown that $f$ is bounded. By (\ref{eq:f_bounded}), we have
			\begin{equation}
				|f(g)| \le \lVert f\rVert_\hm\cdot\lVert g\rVert_H
			\end{equation}
			for all $g \in G$. Therefore, $f$ is bounded.
			
			Hence, $f \in H^*$ for all $f \in \hm$, using the action we have assigned to $f$ to view it as a functional on $H$. Thus, $\hm \subseteq H^*$.
		\end{proof}
		
		\questionpart
		For every $f \in \hm$, there exists $u \in H$ such that
		\begin{equation}
			\label{eq:weak_form}
			B(x,u) = f(x) \quad\text{for all } x\in H.
		\end{equation}
		
		\begin{proof}
			After all the work on the previous parts, this is a simple application of the Lax-Milgram theorem. Since $B$ satisfies the assumptions of the theorem by 2.2, it follows by the Lax-Milgram theorem that for all $f \in H^*$, there exists $u \in H$ such that (\ref{eq:weak_form}) is true. Since $\hm \subseteq H^*$ by 2.3, given $f \in \hm$, we have $f \in H^*$, so we can find $u\in H$ such that (\ref{eq:weak_form}) holds.
		\end{proof}
		
		\questionpart
		Given $f \in \hm$, if $u$ satisfies (\ref{eq:weak_form}), then $u$ solves the ODE
		\begin{equation}
			u' -u'' = f.
		\end{equation}
		\begin{proof}
			
		\end{proof}
	\end{arabicparts}
\end{document}