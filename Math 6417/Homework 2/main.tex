\documentclass{homework}
\usepackage{enumitem}

\newcommand{\hwclass}{Math 6108}
\newcommand{\hwname}{Jacob Hauck}
\newcommand{\hwtype}{Homework}

\newcommand{\R}{\textbf{R}}
\newcommand{\dee}{\;\text{d}}
\newcommand{\eps}{\varepsilon}
\newcommand{\pl}[2]{\frac{\partial #1}{\partial #2}}
\newcommand{\dl}[2]{\frac{\text{d} #1}{\text{d} #2}}
\newcommand{\sgn}{\text{sgn}}
\newcommand{\bigoh}{\mathcal{O}}

\newcommand{\hwnum}{2}

\begin{document}
	\maketitle
	
	\question
	
	A continuous function $\sigma : \R \to \R$ is called \textbf{sigmoidal} if there exists $T > 0$ such that
	\begin{equation}
		\sigma(t) = \begin{cases}
			1 & t \ge T, \\
			0 & t \le -T.
		\end{cases}
	\end{equation}
	Let $\sigma$ be sigmoidal in the following.
	
	\begin{arabicparts}
		\questionpart
		Fix $y, \theta, \phi \in \R$, and define
		\begin{equation}
			\label{eq:sigma_lambda_def}
			\sigma_\lambda(x) = \sigma(\lambda(yx + \theta) + \phi), \qquad x \in \R.
		\end{equation}
		Then
		\begin{equation}
			\lim_{\lambda\to\infty}\sigma_\lambda(x) = \gamma(x) = \begin{cases}
				1 & yx + \theta > 0 \\
				0 & yx + \theta < 0 \\
				\sigma(\phi) & yx + \theta = 0.
			\end{cases}
		\end{equation}
		\begin{proof}
			We use proof by cases. For any $x \in \R$, there are exactly three possibilities.
			\begin{enumerate}[label=\Roman*.]
				\item $yx + \theta > 0$.
				
				Set $\Lambda = \frac{T-\phi}{yx + \theta}$. Then $\lambda \ge \Lambda$ implies that $\lambda(yx+\theta) + \phi \ge T$, so $\sigma_\lambda(x) = 1$ because $\sigma$ is sigmoidal. Therefore, $\sigma_\lambda(x) \to 1 = \gamma(x)$ as $\lambda \to\infty$.
				
				\item $yx + \theta < 0$.
				
				Set $\Lambda = \frac{-T-\phi}{yx+\theta}$. Then $\lambda \ge \Lambda$ implies that $\lambda(yx+\theta) + \phi \le -T$, so $\sigma_\lambda(x) = 0$ because $\sigma$ is sigmoidal. Therefore, $\sigma_\lambda(x) \to 0 = \gamma(x)$ as $\lambda \to \infty$.
				
				\item $yx + \theta = 0$.
				
				In this case, $\sigma_\lambda(x) = \sigma(\phi)$ for all $\lambda$, so clearly $\sigma_\lambda(x) \to \sigma(\phi) = \gamma(x)$ as $\lambda \to \infty$.
			\end{enumerate}
		\end{proof}
		
		\newcommand{\zeroset}{{\Pi_{y,\theta}}}
		\newcommand{\posset}{{H_{y,\theta}}}
		\newcommand{\negset}{{N_{y,\theta}}}
		\newcommand{\propstar}{(*)}
		\questionpart 
		For any $y, \theta \in \R$, define
		\begin{equation}
			\zeroset = \{x \in [0,1] : yx + \theta = 0\}, \qquad \posset = \{x \in [0,1] : yx + \theta > 0\}.
		\end{equation}
		If $\mu$ is a finite Borel measure on $[0,1]$, then we say that $\mu$ has property $\propstar$ if and only if
		\begin{equation}
			\tag{$*$}
			\int_0^1\sigma_\lambda(x)\dee\mu(x) = 0 \quad \text{for all}\;\lambda, y, \theta, \phi \in \R.
		\end{equation}
		If $\mu$ has property $\propstar$, then
		\begin{equation}
			0 = \mu(\posset) + \sigma(\phi)\mu(\zeroset) \quad \text{for all}\; y, \theta,\phi \in \R.
		\end{equation}
		\begin{proof}
			Let $y, \theta, \phi \in \R$ be given. Since $\sigma$ is continuous, it is bounded on $[-T,T]$, say by $M > 0$. Since $\sigma$ is sigmoidal, it follows that $\{\sigma(t) : t \notin [-T,T]\} = \{0,1\} \subseteq \{\sigma(t) : t \in [-T,T]\}$, so $|\sigma(t)| \le M$ for all $t \in \R$. This implies that $|\sigma_\lambda(x)| = |\sigma(\lambda(yx+\theta) + \phi)| \le M$ for all $x \in [0,1]$ and all $\lambda$.
			
			The constant function $M$ is integrable on $[0,1]$, and $\sigma_\lambda \to \gamma$ pointwise on $[0,1]$ as $\lambda \to \infty$ by the previous part, so, by the Dominated Convergence Theorem\footnote{$\sigma_\lambda$ is inetgrable because it is continuous},
			\begin{equation}
				\lim_{\lambda\to\infty} \int_0^1\sigma_\lambda \dee\mu = \int_0^1\gamma\dee\mu.
			\end{equation}
			Define $\negset = [0,1]\setminus(\posset \cup \zeroset)$, so that $[0,1] = \posset \cup \zeroset \cup \negset$ disjointly. Then\footnote{Each of $\posset$, $\zeroset$ and $\negset$ is measurable because they are just intervals, and $\mu$ is Borel}
			\begin{align}
				0 &= \int_0^1\gamma\dee\mu = \int_\posset \gamma\dee\mu + \int_\zeroset\gamma\dee\mu + \int_\negset\gamma\dee\mu\\
				&= \mu(\posset) + \sigma(\phi)\mu(\zeroset)
			\end{align}
			because $\gamma\big\vert_\posset = 1$, $\gamma\big\vert_\zeroset = \sigma(\phi)$, and $\gamma\big\vert_\negset = 0$ by definition.
		\end{proof}
		
		\questionpart
		If $\mu$ has property $\propstar$, then $\mu = 0$.
		\begin{proof}
			Let $\mu$ have property $\propstar$. If we choose $\phi = -T$, then $\sigma(\phi) = 0$, so by the previous part, $\mu(\posset) = 0$ for all $y, \theta \in \R$. Then, if we choose $\phi = T$ so that $\sigma(\phi) = 1$, we see that, by the previous part, $\mu(\zeroset) = 0$ for all $y, \theta\in \R$, as well.
			
			A simple series of deductions shows that\footnote{$[a,b)$ is measurable because $\mu$ is Borel} $\mu([a,b)) =0$ for all $0 \le a \le b \le 1$:
			\begin{itemize}
				\item If we choose $y = 0$ and $\theta = 0$, then $\zeroset = [0,1]$, so $\mu([0,1]) = 0$.
				\item Given $a \in [0,1]$, if we choose $y = -1$ and $\theta = a$, then $\posset = [0,a)$, so $\mu([0,a)) = 0$.
				\item Hence, given $a \in [0,1]$, we have $0 = \mu([0,1]) = \mu([0,a)) + \mu([a, 1]) = 0 + \mu([a,1])$, so $\mu([a,1]) = 0$.
				\item Finally, given $0 \le a \le b \le 1$, we have $0 = \mu([a,1]) = \mu([a,b)) + \mu([b,1]) = \mu([a,b)) + 0$, so $\mu([a,b)) = 0$.
			\end{itemize}
			
			The set of finite Borel measures on $[0,1]$ can be shown to be isomorphic as a Banach space\footnote{Using the uniform norm in $C([0,1])$} to $(C([0,1]))^*$ by defining the action of a measure $\mu$ on a continuous function $h \in C([0,1])$ by
			\begin{equation}
				\mu(h) = \int_0^1h\dee\mu.
			\end{equation}
			Hence, we only need to show that $\mu = 0$ as an element of $(C([0,1]))^*$, that is, $\mu(h) = 0$ for all $h \in C([0,1])$.
			
			We start by showing that $\mu\left(\chi_{[a,b)}\right) = 0$ for all $0 \le a \le b \le 1$. This is easy enough because
			\begin{equation}
				\mu\left(\chi_{[a,b)}\right) = \int_0^1 \chi_{[a,b)}\dee\mu = \mu([a,b)) = 0.
			\end{equation}
			Given $h \in C([0,1])$ and a natural number $n$, we can choose $\delta > 0$ such that $|h(t) - h(s)| < \frac{1}{n}$ for all $t,s \in [0,1]$ satisfying $|t-s| < \delta$. Choose a partition $\{t_i\}_{i=1}^{N_n}$ of $[0,1]$ such that $t_i - t_{i-1} < \delta$ for $i =1,2,\dots,N_n$.
			
			Define the function
			\begin{equation}
				h_n(t) = \begin{cases}
					\sum\limits_{i=1}^{N_n} m_i\cdot\chi_{[t_{i-1}, t_i)}(t) & t \in [0,1) \\
					h(1) & t = 1,
				\end{cases}
			\end{equation}
			where $m_i = \min\limits_{t \in [t_{i-1}, t_i]} h(t)$. Since $h$ is continuous, it achieves the value $m_i$ for some $t^* \in [t_{i-1},t_i]$. On the other hand, $h_n(t) = m_i$ for all $t \in [t_{i-1}, t_i)$. For all $t \in [t_{i-1}, t_i)$, we have $|t^* - t| < \delta$ by the construction of the partition, so for all $t \in [t_{i-1}, t_i)$,
			\begin{equation}
				|h(t) - h_n(t)| = |h(t) - m_i| = |h(t) - h(t^*)| < \frac{1}{n}.
			\end{equation}
			Since this holds for all $i$, and $h_n(1) = h(1)$ for all $n$, it follows that $h_n \to h$ uniformly on $[0,1]$.
			
			Clearly $h_n$ is simple by construction, so $h_n$ is measurable for all $n$. Therefore, by the uniform convergence of $h_n$ to $h$,
			\begin{align}
				\mu(h) &= \int_0^1h\dee \mu = \lim_{n\to\infty}\int_0^1h_n\dee\mu \\
				&= \lim_{n\to\infty}\left[h(1)\int_{\{1\}}\dee\mu  + \sum_{i=1}^{N_n}m_i\int_{[t_{i-1},t_i)}\chi_{[t_{i-1}, t_i)} \dee \mu\right] \\
				&= \lim_{n\to\infty}\left[h(1)\mu(\{1\}) + \sum_{i=1}^{N_n}m_i\mu([t_{i-1},t_i))\right] \\
				&= \lim_{n\to\infty}h(1)\mu([1,1]) \\[0.5em]
				&= \lim_{n\to\infty}0 = 0
			\end{align}
			because $\mu([t_{i-1},t_i)) = 0$ for all $i$, and $\mu(\{1\}) = \mu([1,1]) = 0$ by the third bullet near the beginning of the proof.
			
			Thus, $\mu(h) = 0$ for all $h \in C([0,1])$, that is, $\mu = 0$.
		\end{proof}
		
		\newcommand{\nn}{\mathcal{N}}
		\questionpart 
		Let $\nn$ denote the set of functions $G: [0,1] \to \R$ of the form
		\begin{equation}
			G(x) = \sum_{j=1}^N \alpha_j \sigma(y_jx + \theta_j)
		\end{equation}
		for some positive integer $N$ and parameters $\alpha_j, y_j, \theta_j \in \R$, $j=1,2,\dots,N$. Then $\nn$ is dense in $C([0,1])$ in the uniform norm.
		\begin{proof}
			Since $\sigma$ is continuous, it follows easily that $G$ is continuous for all $G \in \nn$. Hence, $\nn \subseteq C([0,1])$. It is obvious that $\nn$ is nonempty, but $\nn$ is also closed under linear combinations of elements of $\nn$ because, given $G,G'\in\nn$ and $p,p' \in \R$, there exist positive integers $N, N'$, parameters $\alpha_j, y_j, \theta_j$, $j=1,2,\dots N$, and parameters $\alpha_j', y_j', \theta_j'$, $j=1,2,\dots,N'$ such that
			\begin{align}
				pG + p'G' &= p\sum_{j=1}^N \alpha_j\sigma(y_jx + \theta_j) + p'\sum_{j=1}^{N'}\alpha_j'\sigma(y_j'x + \theta_j') \\
				&= \sum_{j=1}^{N+N'}\beta_j\sigma(z_jx+\vartheta_j),
			\end{align}
			taking $\beta_j = p\alpha_j$, $z_j = y_j$, $\vartheta_j = \theta_j$ if $j \le N$, and $\beta_j = p'\alpha_{j-N}'$, $z_j = y_{j-N}'$, $\vartheta_j = \theta_{j-N}'$ if $j > N$. Thus, $pG + p'G' \in \nn$ by definition, and $\nn$ is a vector subspace of $C([0,1])$.
			
			Then $\overline{\nn}$ is also a subspace. Suppose that $\overline{\nn} \ne C([0,1])$. Then we can choose some $g \in C([0,1]) \setminus \overline{\nn}$. Let $d = \mathrm{dist}(g, \overline{\nn})$. Then $d > 0$ because $\overline{\nn}$ is closed. By the Mazur separation lemma 1, there exists some $\mu \in (C([0,1]))^*$ such that $\mu(g) = d$, and $\mu(G) = 0$ for all $G \in \overline{\nn}$.
			
			Clearly $\sigma_\lambda \in \nn \subseteq \overline{\nn}$ for any choice of $\lambda, y, \theta, \phi \in \R$; thus, $\mu(\sigma_\lambda) = 0$ for all $\sigma_\lambda$. Using the fact that $\mu$ is a finite Borel measure on $[0,1]$ (as mentioned in the previous part) with
			\begin{equation}
				\mu(h) = \int_0^1h\dee \mu \quad \text{for all}\; h\in C([0,1]),
			\end{equation}
			we see that
			\begin{equation}
				0 = \mu(\sigma_\lambda) = \int_0^1\sigma_\lambda\dee\mu
			\end{equation}
			for all $\sigma_\lambda$. That is, $\mu$ has property $\propstar$. Therefore, by the previous part, $\mu = 0$. This contradicts the fact that $\mu(g) = d \ne 0$. Hence, $\overline{\nn} = C([0,1])$, which is equivalent to saying that $\nn$ is dense in $C([0,1])$.
		\end{proof}
	\end{arabicparts}
	
	\question
	For any positive integer $n$, define the linear functional $\ell_n : C([0,1]) \to \R$ by
	\begin{equation}
		\ell_n(f) = \sum_{j=0}^n w_j^n f(x_j^n),
	\end{equation}
	where $\{x_j^n\}_{j=0}^n$ is a partition of $[0,1]$, and $\{w_j^n\}_{j=0}^n$ is a sequence of real numbers. $\ell_n(f)$ is meant to be a numerical quadrature formula for the integral of $f$ on $[0,1]$ with weight function $w \in L^1([0,1])$.
	\begin{arabicparts}
		\questionpart 
		Equipping $C([0,1])$ with the uniform norm, the functional $\ell_n$ is a bounded, linear functional, and the induced norm of $\ell_n$ is given by
		\begin{equation}
			\lVert \ell_n \rVert = \sum_{j=0}^n |w_j^n|.
		\end{equation}
		\begin{proof}
			Let $f,g \in C([0,1])$, and let $\alpha,\beta \in \R$. Then
			\begin{align}
				\ell_n(\alpha f+ \beta g) &= \sum_{j=0}^n w_j^n(\alpha f(x_j^n) + \beta g(x_j^n)) = \alpha\sum_{j=0}^n w_j^nf(x_j^n) + \beta\sum_{j=0}^n w_j^ng(x_j^n) \\
				&= \alpha\ell_n(f) + \beta\ell_n(g),
			\end{align}
			so $\ell_n$ is linear.
			
			Let $f \in C([0,1])$ have $\lVert f \rVert \le 1$. Then $|f(x)| \le 1$ for all $x \in [0,1]$ by the definition of $\lVert f\rVert$, and
			\begin{equation}
				|\ell_n(f)| \le \sum_{j=0}^n |w_j^n|\cdot|f(x_j^n)| \le \sum_{j=0}^n |w_j^n|.
			\end{equation}
			Since $f$ was arbitrary with norm bounded by $1$,
			\begin{equation}
				\lVert \ell_n\rVert \le \sum_{j=0}^n |w_j^n|.
			\end{equation}
			On the other hand, we can choose $f \in C([0,1])$ such that $\lVert f\rVert = 1$, and $f(x_j^n) = \sgn(w_j^n)$, the sign of $w_j^n$, by considering the piecewise linear function that interpolates between the points $\{(x_j^n, \sgn(w_j^n))\}_{j=0}^n$ (if all the $w_j^n$ are zero for some fixed $n$, then choose $f = 1$). Hence,
			\begin{equation}
				\lVert \ell_n\rVert \ge \frac{|\ell_n(f)|}{\lVert f\rVert} = \left|\sum_{j=0}^n w_j^n f(x_j^n)\right| = \sum_{j=0}^n|w_j^n|
			\end{equation}
			because $|w_j^n| = w_j^n\cdot\sgn(w_j^n) = w_j^nf(x_j^n)$. Therefore,
			\begin{equation}
				\lVert\ell_n\rVert = \sum_{j=0}^n |w_j^n|.
			\end{equation}
		\end{proof}
		
		\questionpart
		Suppose that the formula converges in the sense that
		\begin{equation}
			\label{eq:formaul_convergence}
			\lim_{n\to\infty} \left|\int_0^1 f(x)w(x)\dee x - \ell_n(f)\right| = 0 \quad \text{for all}\; f\in C([0,1]).
		\end{equation}
		Then
		\begin{equation}
			\sup_{n\ge 0} \left(\sum_{j=0}^n |w_j^n|\right) < \infty.
		\end{equation}
		\begin{proof}
			By the Banach-Steinhaus theorem (with $X = C([0,1])$, $Y=\R$, and $T_\alpha =\ell_\alpha$, $\alpha=1,2,\dots$, in the notation of the slides), either $\{\lVert\ell_n\rVert\}_{n=1}^\infty$ is bounded, or $\sup\limits_{n\ge 0}|\ell_n(f)| = \infty$ for some $f \in C([0,1])$. The second possibility is false, because, by the convergence assumption (\ref{eq:formaul_convergence}), the sequence $\{\ell_n(f)\}_{n=1}^\infty$ converges to a finite number for all $f \in C([0,1])$, and convergent sequences are bounded. Therefore, $\{\lVert \ell_n\rVert\}$ is bounded.
			
			Since
			\begin{equation}
				\lVert \ell_n\rVert = \sum_{j=0}^n |w_j^n|,
			\end{equation}
			it follows that
			\begin{equation}
				\left\{\sum_{j=0}^n |w_j^n|\right\}_{n=1}^\infty
			\end{equation}
			is bounded, that is,
			\begin{equation}
				\sup_{n\ge 0} \left(\sum_{j=0}^n |w_j^n|\right) < \infty.
			\end{equation}
		\end{proof}
		
		\questionpart Suppose that
		\begin{equation}
			\label{eq:ell_n_bounded}
			\sup_{n\ge 0} \left(\sum_{j=0}^n|w_j^n|\right) < \infty.
		\end{equation}
		Suppose furthermore that for any polynomial $p(x)$ defined on $[0,1]$
		\begin{equation}
			\label{eq:polynomial_convergence}
			\lim_{n\to\infty}\left|\int_0^1p(x)w(x)\dee x - \ell_n(p)\right| = 0.
		\end{equation}
		Then the quadrature formula $\ell_n$ works on all of $C([0,1])$ in the sense that
		\begin{equation}
			\lim_{n\to\infty}\left|\int_0^1 f(x)w(x)\dee x  - \ell_n(f)\right| = 0 \quad \text{for all} \; f\in C([0,1]).
		\end{equation}
		\begin{proof}
			Let $f \in C([0,1])$, and let $\varepsilon > 0$ be given. By the Weierstrass Approximation Theorem, we can choose a polynomial $p(x)$ defined on $[0,1]$ such that
			\begin{equation}
				\lVert p - f\rVert_{C([0,1])} < \varepsilon,
			\end{equation}
			where $\lVert \cdot \rVert_{C([0,1])}$ is the usual uniform norm.
		
			Furthermore, by (\ref{eq:polynomial_convergence}), we can choose $N$ large enough that
			\begin{equation}
				\left|\int_0^1 p(x)w(x)\dee x - \ell_n(p)\right| < \varepsilon
			\end{equation}
			for all $n > N$, and, by (\ref{eq:ell_n_bounded}), we can choose $M > 0$ such that
			\begin{equation}
				\sum_{j=0}^n |w_j^n| < M
			\end{equation}
			for all $n \ge0$. Therefore, if $n > N$, then
			\begin{align}
				\left|\int_0^1 f(x)w(x)\dee x - \ell_n(f)\right| &\le \left|\int_0^1(f(x) - p(x))w(x)\dee x\right| \\
				&\quad+ \left|\ell_n(p) - \ell_n(f)\right| + \left|\int_0^1p(x)w(x)\dee x - \ell_n(p)\right| \\
				&\le \varepsilon\int_0^1|w(x)|\dee x + \sum_{j=0}^n |p(x_j^n) - f(x_j^n)|\cdot|w_j^n| + \varepsilon \\
				&\le (\lVert w \rVert_{L^1} + M + 1)\varepsilon.
			\end{align}
			Since $\lVert w \rVert_{L^1} + M + 1$ is independent of $n$ and $\varepsilon > 0$ was arbitrary, it follows that
			\begin{equation}
				\lim_{n\to\infty}\left|\int_0^1 f(x)w(x)\dee x - \ell_n(f)\right| = 0.
			\end{equation}
			Thus, the quadrature rule works for the arbitrary continuous function $f \in C([0,1])$, and, consequently, for all functions in $C([0,1])$.
		\end{proof}
	\end{arabicparts}
	
\end{document}