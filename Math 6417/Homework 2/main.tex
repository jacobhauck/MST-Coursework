\documentclass{homework}
\usepackage{enumitem}

\newcommand{\hwclass}{Math 6108}
\newcommand{\hwname}{Jacob Hauck}
\newcommand{\hwtype}{Homework}

\newcommand{\R}{\textbf{R}}
\newcommand{\dee}{\;\text{d}}
\newcommand{\eps}{\varepsilon}
\newcommand{\pl}[2]{\frac{\partial #1}{\partial #2}}
\newcommand{\dl}[2]{\frac{\text{d} #1}{\text{d} #2}}
\newcommand{\sgn}{\text{sgn}}
\newcommand{\bigoh}{\mathcal{O}}

\newcommand{\hwnum}{2}

\begin{document}
	\maketitle
	
	A continuous function $\sigma : \R \to \R$ is called \textbf{sigmoidal} if there exists $T > 0$ such that
	\begin{equation}
		\sigma(t) = \begin{cases}
			1 & t \ge T, \\
			0 & t \le -T.
		\end{cases}
	\end{equation}
	Let $\sigma$ be sigmoidal in the following problems.
	
	\question Let $y \in \R^n$, and $\theta, \phi \in \R$. For $x \in \R^n$, define
	\begin{equation}
		\sigma_\lambda(x; \theta,\phi) = \sigma\left(\lambda\left(y^Tx + \theta\right) + \phi\right).
	\end{equation}
	Then
	\begin{equation}
		\sigma_\lambda(x;\theta,\phi) \to \gamma(x) = \begin{cases}
			1 & y^Tx + \theta > 0 \\
			0 & y^Tx + \theta < 0 \\
			\sigma(\phi) & y^Tx + \theta = 0
		\end{cases} \qquad \text{ as } \lambda \to\infty.
	\end{equation}
	\begin{proof}
		We use proof by cases.
		\begin{enumerate}
			\item If $y^Tx + \theta > 0$, then
			\begin{equation}
				\lambda \ge \frac{T - \phi}{y^Tx + \theta} \implies \lambda(y^Tx + \theta) + \phi \ge T \implies \sigma_\lambda(x;\theta,\phi) = 1,
			\end{equation}
			so $\sigma_\lambda(x;\theta, \phi) \to 1 = \gamma(x)$ as $\lambda \to\infty$.
			\item If $y^Tx + \theta < 0$, then
			\begin{equation}
				\lambda \ge \frac{-T - \phi}{y^T x + \theta}\implies \lambda(y^Tx +\theta) + \phi \le -T \implies \sigma_\lambda(x;\theta,\phi) = 0,
			\end{equation}
			so $\sigma_\lambda(x;\theta,\phi) \to 0 = \gamma(x)$ as $\lambda \to\infty$.
			\item If $y^Tx +\theta =0$, then $\sigma_\lambda(x;\phi,\theta) = \sigma(\phi)$ for all $\lambda$, so $\sigma_\lambda(x;\phi,\theta) \to \sigma(\phi) = \gamma(x)$ as $\lambda \to \infty$.
		\end{enumerate}
	\end{proof}
	
	\question
	\newcommand{\zset}{\Pi_{y,\theta}}
	\newcommand{\pset}{H_{y,\theta}}
	Let $y \in \R^n$, let $\zset = \left\{x\mid y^Tx + \theta = 0\right\}$, and let $\pset=\left\{x\mid y^Tx+\theta > 0\right\}$. If $\mu$ is a finite Borel measure on $[0,1]^n$ such that
	\begin{equation}
		\label{eq:mu_zero_cond}
		\int_{[0,1]^n}\sigma_\lambda(x)\dee\mu(x) = 0 \qquad \text{for all } \lambda, \theta, \phi\in\R,
	\end{equation}
	then
	\begin{equation}
		\sigma(\phi)\mu(\zset) + \mu(\pset) = 0\qquad \text{for all } \lambda, \theta, \phi\in\R.
	\end{equation}
	\begin{proof}
		Fix $\theta,\phi\in\R$. For any $\lambda \in \R$, the function $\sigma_\lambda(\,\cdot\,;\theta,\phi)$ is clearly dominated on $[0,1]^n$ by the constant function $g(x) = \max\limits_{t\in[-T,T]}|\sigma(t)|$. The following statements are true.
		\begin{enumerate}
			\item Open subsets of $[0,1]^n$ (in the relative topology of $[0,1]^n$) are $\mu$-measurable by the definition of a Borel measure.
			\item $g$ is constant, so it is continuous, meaning that $g^{-1}(U)$ is open for any open set $U$ (in the relative topology of $[0,1]^n$).
			\item 1. and 2. imply that $g^{-1}(U)$ is $\mu$-measurable for any open set $U$ (in the relative topology of $[0,1]^n$).
			\item 3. implies that $g$ is $\mu$-measurable.
			\item $[0,1]^n$ is open in the relative topology on $[0,1]^n$, so it is $\mu$-measurable by 1.
			\item $g$ is constant on the $\mu$-measurable set $[0,1]^n$, so $g$ is simple with respect to the measure $\mu$ by definition.
			\item Simple functions are always integrable, so, by 6., $g$ is integrable with respect to $\mu$.
			\item $\sigma$ is continuous, and $x\mapsto\lambda (y^Tx + \theta) + \phi$ is also continuous, so $\sigma_\lambda(\,\cdot\,;\theta,\phi)$, the composition, of the two, is also continuous.
			\item By 8., $\sigma_\lambda(\,\cdot\,;\theta,\phi)$ is continuous on $[0,1]^n$, so $\sigma_\lambda^{-1}(U;\theta,\phi)$ is open for any open set $U$ (in the relative topology of $[0,1]^n$).
			\item 1. and 9. imply that $\sigma_\lambda^{-1}(U;\theta,\phi)$ is open for any open set $U$ (in the relative topology of $[0,1]^n$).
			\item 10. implies that $\sigma_\lambda(\,\cdot\,;\theta,\phi)$ is $\mu$-measurable.
		\end{enumerate}
		Let $\{\lambda_n\}$ be any sequence of real numbers such that $\lambda_n \to \infty$ as $n\to\infty$. Then $\sigma_{\lambda_n}(x;\theta,\phi) \to \gamma(x)$ as $n \to \infty$ for all $x \in [0,1]^n$ by the previous problem, and $\{\sigma_{\lambda_n}(\,\cdot\,;\theta,\phi)\}$ and $g$ satisfy the hypotheses of the Dominated Convergence Theorem, so
		\begin{align}
			0&=\lim_{n\to\infty}\int_{[0,1]^n}\sigma_{\lambda_n}(x;\theta,\phi)\dee\mu(x) \\
			&= \int_{[0,1]^n} \gamma(x)\dee\mu(x) \\
			&= \int_{[0,1]^n \setminus (\zset \cup \pset)} \gamma(x)\dee\mu(x) + \int_{\zset}\gamma(x)\dee\mu(x) + \int_{\pset}\gamma(x)\dee\mu(x)  \\
			&= \int_{[0,1]^n\setminus(\zset\cup\pset)} 0\dee\mu(x) + \int_{\zset}\sigma(\phi)\dee\mu(x) + \int_{\pset}1\dee\mu(x) \\
			&= \sigma(\phi)\mu(\zset) + \mu(\pset).
		\end{align}
	\end{proof}
	
	\question
	Suppose that $\mu$ satisfies (\ref{eq:mu_zero_cond}). Then $\mu = 0$. 
	\begin{proof}
		Define the linear functional $F: L^\infty(\R) \to \R$ by
		\begin{equation}
			F(h) = \int_{[0,1]^n}h\left(y^Tx\right)\dee\mu(x)
		\end{equation}
		First, let $h = \chi_{[a,\infty)}$ for some $a \in \R$. Choose $\theta = -a$, and $\phi = T$. Then $\sigma(\phi) = 1$, and
		\begin{equation}
			h(y^Tx) = 1 \iff y^Tx \ge a \iff y^Tx + \theta  \ge 0 \iff y^Tx \in \zset \cup \pset.
		\end{equation}
		If $y^Tx \notin \zset \cup \pset$, then $h(y^Tx) = \chi_{[a,\infty)}(y^Tx) = 0$, because characteristic functions can only be 0 or 1. Therefore, by the previous problem,
		\begin{align}
			F(h) &= \int_{[0,1]^n}h(y^Tx)\dee\mu(x) \\
			&= \int_{\zset}1\dee \mu(x) + \int_{\pset} 1\dee\mu(x) \\
			&= 1\cdot\mu(\zset) + \mu(\pset) = \sigma(\phi)\mu(\zset) + \mu(\pset) = 0.
		\end{align}
		Second, for $a,b\in \R$, the characteristic function $\chi_{[a,b)}$ can be written as $\chi_{[a,\infty)} - \chi_{[b,\infty)}$. Since $F$ is linear, it follows that $F\left(\chi_{[a,b)}\right) = 0$.
	\end{proof}
	
	\question
	\newcommand{\neural}{\mathcal{N}}
	Let $\neural$ be the set of all functions $G \in C[0,1]$ of the form
	\begin{equation}
		G(x) = \sum_{j=1}^n \alpha_j\sigma(y_jx + \theta_j),
	\end{equation}
	for some $\alpha_j, y_j, \theta_j \in \R$. Then $\neural$ is dense in $C[0,1]$ in the uniform norm.
	\begin{proof}
		It is clear that $\neural$ is a subspace of $C[0,1]$. Then $\overline{\neural}$ is a closed subspace of $C[0,1]$.
		
		Suppose that $\neural$ is \textit{not} dense in $C[0,1]$. Then there exists some function $f \in C[0,1]$ such that $f \notin \overline{\neural}$. Define the linear functional $\mu_0$ on the subspace $\mathrm{span}\{\overline{\neural}, f\}$ by setting
		\begin{equation}
			\mu_0(G + af) = a.
		\end{equation}
		This is well-defined because every element $g \in \mathrm{span}\{\overline{\neural}, f\}$ can be written uniquely in the form $g = G + af$, where $G \in \neural$, and $a \in \R$. Indeed, if we had $G,G' \in \neural$, and $a,a' \in \R$ such that $G + af = G' + a'f$, then $G-G' = (a'-a)f$, and $a \ne a'$ would imply that $f \in \neural$, so $a = a'$, and $G = G'$.
		
		The functional $\mu_0$ is clearly linear because
		\begin{equation}
			\mu_0(r(G + af) + s(G' + a'f)) = \mu_0((rG + sG') + (sa + ra')f) = sa + ra' = s\mu_0(G+af) + r\mu_0(G'+a'f).
		\end{equation}
		Furthermore, $\mu_0(\neural) = \{0\}$ because
		\begin{equation}
			\mu_0(G) = \mu_0(G + 0f) = 0.
		\end{equation}
		Finally, $\mu_0$ is dominated by the sublinear functional $p =\frac{\lVert \cdot \rVert}{\lVert f \rVert}$ because
		\begin{equation}
			|\mu_0(G + af)| = |a| = \frac{\lVert af\rVert}{\lVert f\rVert} \le \frac{\lVert G+af\rVert + \lVert G \rVert}{\lVert f \rVert}
		\end{equation}
	\end{proof}
	
	\question
	
	
\end{document}