\documentclass{homework}
\newcommand{\R}{\textbf{R}}
\newcommand{\dee}{\;\text{d}}
\usepackage{enumitem}

\newcommand{\hwclass}{Math 6418}
\newcommand{\hwname}{Jacob Hauck}
\newcommand{\hwtype}{Homework}

\newcommand{\dist}{\mathcal{D}}

\usepackage{listings}

\newcommand{\hwnum}{2}
\renewcommand{\questiontype}{Problem}

\begin{document}
	\maketitle
	
	\question Let $X_1$, $X_2$, $X_3$ be the times when the three students leave, and let $Y$ be the time when there is only one remaining. 
	\begin{alphaparts}
		\questionpart
		Then
		\begin{equation*}
			Y = \begin{cases}
				X_1 & X_2 \le X_1 \le X_3 \quad\text{or}\quad X_3 \le X_1 \le X_2, \\
				X_1 & X_1 \le X_2 \le X_3 \quad\text{or}\quad X_3 \le X_2 \le X_1, \\
				X_1 & X_1 \le X_3 \le X_2 \quad\text{or}\quad X_2 \le X_3 \le X_1.
			\end{cases}
		\end{equation*}
		Since $X_1, X_2, X_3$ are independent, the joint p.d.f. of $X_1, X_2, X_3$ is
		\begin{equation*}
			f(x_1, x_2, x_3) = 6e^{-x_1 -2x_2 -3x_3}I(x_1\ge0, x_2\ge 0, x_3\ge0).
		\end{equation*}
		Thus,
		\begin{alignat*}{2}
		E[Y] &={}&& \int_0^\infty \int_{x_2}^\infty\int_{x_1}^\infty 6e^{-x_1-2x_2-3x_3}x_1 \dee x_3\dee x_1 \dee x_2 + \int_0^\infty \int_{x_3}^\infty\int_{x_1}^\infty 6e^{-x_1-2x_2-3x_3}x_1\dee x_2 \dee x_1 \dee x_3 \\
		&&&{}+ \int_0^\infty \int_{x_1}^\infty\int_{x_2}^\infty 6e^{-x_1-2x_2-3x_3}x_2\dee x_3\dee x_2 \dee x_1 + \int_0^\infty \int_{x_3}^\infty\int_{x_2}^\infty 6e^{-x_1-2x_2-3x_3}x_2\dee x_1 \dee x_2 \dee x_3 \\
		&&&{}+ \int_0^\infty \int_{x_1}^\infty\int_{x_3}^\infty 6e^{-x_1-2x_2-3x_3}x_3\dee x_2\dee x_3 \dee x_1 + \int_0^\infty \int_{x_2}^\infty\int_{x_3}^\infty 6e^{-x_1-2x_2-3x_3}\dee x_1 \dee x_3 \dee x_2 \\[0.5em]
		&={}&& \int_0^\infty  2e^{-2x_2} \int_{x_2}^\infty x_1 e^{-4x_1}\dee x_1 \dee x_2 + \int_0^\infty 3e^{-3x_3}\int_{x_3}^\infty x_1e^{-3x_1} \dee x_1 \dee x_3 \\
		&&&{}+ \int_0^\infty 2e^{-x_1}\int_{x_1}^\infty x_2 e^{-5x_2}\dee x_2 \dee x_1 + \int_0^\infty 6^{-3x_3}\int_{x_3}^\infty x_2e^{-3x_2}\dee x_2 \dee x_3 \\
		&&&{}+ \int_0^\infty 3e^{-x_1}\int_{x_1}^\infty x_3 e^{-5x_3}\dee x_3 \dee x_1 + \int_0^\infty 6e^{-2x_2}\int_{x_2}^\infty x_3 e^{-4x_3}\dee x_3 \dee x_2 \\[0.5em]
		&={}&& \int_0^\infty  2e^{-2x_2} \left[\frac{1}{4}x_2e^{-4x_2} + \frac{1}{16} e^{-4x_2}\right] \dee x_2 + \int_0^\infty 3e^{-3x_3} \left[\frac{1}{3}x_3e^{-3x_3} + \frac{1}{9}e^{-3x_3}\right]\dee x_3 \\
		&&&{}+ \int_0^\infty 2e^{-x_1}\left[\frac{1}{5}x_1e^{-5x_1} + \frac{1}{25}e^{-5x_1}\right] \dee x_1 + \int_0^\infty 6^{-3x_3}\left[\frac{1}{3}x_3e^{-x_3} + \frac{1}{9}e^{-3x_3}\right] \dee x_3 \\
		&&&{}+ \int_0^\infty 3e^{-x_1}\left[\frac{1}{5}x_1e^{-5x_1} + \frac{1}{25}e^{-5x_1}\right] \dee x_1 + \int_0^\infty 6e^{-2x_2} \left[\frac{1}{4}x_2e^{-4x_2} + \frac{1}{16}e^{-4x_2}\right] \dee x_2 \\[0.5em]
		&={}&& \int_0^\infty 5\left[\frac{1}{5}x_1 + \frac{1}{25}\right]e^{-6x_1}\dee x_1 + \int_0^\infty  8\left[\frac{1}{4}x_2 + \frac{1}{16}\right]e^{-6x_2} \dee x_2 + \int_0^\infty 9 \left[\frac{1}{3}x_3 + \frac{1}{9}\right] e^{-6x_3}\dee x_3 \\[0.5em]
			&={}&& \int_0^\infty \left(6x + \frac{17}{10}\right)e^{-6x}\dee x \\[0.5em]
			&={}&& \frac{1}{6} + \frac{17}{60} = \frac{27}{60} = \frac{9}{20}.
		\end{alignat*}
	
	\questionpart We use the CDF method; let $F$ denote the CDF of $L$. Using the independence of $X_1, X_2, X_3$, we have
	\begin{equation*}
		F(t) = P(L \le t) = P(X_1 \le t, X_2 \le t, X_3 \le t) = (1-e^{-t})(1-e^{-2t})(1-e^{-3t})I(t \ge 0).
	\end{equation*}
	Then the PDF of $L$ is
	\begin{align*}
		f(t) &= F'(t) = e^{-t}(1-e^{-2t})(1-e^{-3t}) +2e^{-2t}(1-e^{-t})(1-e^{-3t}) + 3e^{-3t}(1-e^{-t})(1-e^{-2t}), \quad t \ge 0, \\
		&= e^{-t} - e^{-3t} - e^{-4t} + e^{-6t} + 2e^{-2t} - 2e^{-3t} -2e^{-5t} + 2e^{-6t} + 3e^{-3t} - 3e^{-4t} -3e^{-5t} + 3e^{-6t}, \quad t \ge 0,\\
		&=  e^{-t} + 2e^{-2t} - 4e^{-4t} -5 e^{-5t} + 6e^{-6t}, \quad t \ge 0.
	\end{align*}
	
	\questionpart The time until all students are gone is $L$, so we want $E[L]$:
	\begin{align*}
		E[L] &= \int_{-\infty}^\infty tf(t)\dee t = \int_0^\infty t(e^{-t} + 2e^{-2t} - 4e^{-4t} -5 e^{-5t} + 6e^{-6t})\dee t \\
		&= 1 + \frac{1}{2} - \frac{1}{4} - \frac{1}{5} + \frac{1}{6} = \frac{73}{60}.
	\end{align*}
	\end{alphaparts}
	
	\question 
	\begin{alphaparts}
		\questionpart Let $0 < s < t$. Then, by the independent and identical increments property,
		\begin{align*}
			E[N(s)N(t)] &= E[(N(t)-N(s))(N(s) - N(0))] + E[(N(s))^2] \\
			&= E[N(t) - N(s)]E[N(s)] + \Var[N(s)] + E[N(s)]^2 \\
			&= \lambda(t-s)\lambda s + \lambda s + \lambda^2s^2 \\
			&= \lambda^2 ts + \lambda s.
		\end{align*}
		
		\questionpart Using the result from (a), we have
		\begin{equation*}
			\Cov(N(s), N(t)) = E[N(s)N(t)] - E[N(s)]E[N(t)] = \lambda^2ts + \lambda s - \lambda^2ts = \lambda s,
		\end{equation*}
		and
		\begin{equation*}
			\corr(N(s), N(t)) = \frac{\Cov(N(s), N(t))}{\sqrt{\Var[N(s)]\Var[N(t)]}} = \frac{\lambda s}{\sqrt{\lambda^2 st}} = \sqrt{\frac{s}{t}}.
		\end{equation*}
	\end{alphaparts}
	
	\question
	\newcommand{\I}{\mathcal{I}}
	\begin{alphaparts}
		\questionpart For $n=1,\dots, N(t)$, if $t \in [S_{n-1}, S_n]$, then $N(t) = n-1$. It is always true that $S_{N(t)} \le t < S_{N(t) +1}$, and for such $t$ the variable $N(t)$ is constant. Thus,
		\begin{align*}
			\I_t &= \int_0^t N(u)\dee u = \int_{S_{N(t)}}^t N(t)\dee u + \sum_{n=1}^{N(t)}\int_{S_{n-1}}^{S_n} (n-1)\dee u \\
			&= N(t)(t-S_{N(t)}) + \sum_{n=1}^{N(t)} (n-1)(S_n - S_{n-1}) \\
			&= tN(t) - N(t)S_{N(t)} + (N(t) -1)S_{N(t)} + \sum_{n=1}^{N(t)-1}(n-1)S_n - \sum_{n=1}^{N(t)-1} nS_n - 0\cdot(S_1 - S_0)\\
			&= tN(t) - N(t)S_{N(t)} -\sum_{n=1}^{N(t)-1}S_n \\
			&= tN(t) - \sum_{n=1}^{N(t)}S_n.
		\end{align*}
		This clearly still holds when $N(t) = 0$ if we use the convention $\sum\limits_{n=1}^0 S_n = 0$.
		
		\questionpart Using the law of iterated expectation,
		\begin{align*}
			E\left[\sum_{n=1}^{N(t)}S_n\right] &= E\left[E\left[\sum_{n=1}^{N(t)} S_n \mid N(t)\right]\right] \\
			&= \sum_{k=0}^\infty E\left[\sum_{n=1}^k S_n \mid N(t) = k\right]P(N(t) = k) \\
			&= \sum_{k=1}^\infty E\left[\sum_{n=1}^kS_n \mid N(t)=k\right]\frac{\lambda^kt^k e^{-\lambda t}}{k!}.
		\end{align*}
		Let $U_1, \dots, U_k$ be i.i.d. uniformly-distributed random variables on $[0,t]$. Then the order statistics $U_{(1)}, U_{(2)}, \dots U_{(k)}$ have the same distribution as $S_1, S_2, \dots, S_k$ given that $N(t) = k$. Therefore,
		\begin{equation*}
			\sum_{n=1}^k E[S_n \mid N(t) = k] = \sum_{n=1}^kE[U_{(n)}] = \sum_{n=1}^kE[U_n] = \frac{kt}{2},
		\end{equation*}
		where the second-to-last equation follows from the fact that the order statistics are just a permutation of the original variables. Therefore,
		\begin{equation*}
				E\left[\sum_{n=1}^{N(t)}S_n\right] = \frac{t}{2}\sum_{k=1}^\infty \frac{k\lambda^kt^ke^{-\lambda t}}{k!} = \frac{\lambda t^2}{2}\sum_{k=0}^\infty \frac{\lambda^k t^ke^{-\lambda t}}{k!} = \frac{\lambda t^2}{2}.
		\end{equation*}
		This implies that
		\begin{equation*}
			E[\I_t] = E[tN(t)] - E\left[\sum_{n=1}^{N(t)}S_n\right] = tE[N(t)] - \frac{\lambda t^2}{2} = \lambda t^2 - \frac{\lambda t^2}{2} = \frac{\lambda t^2}{2}.
		\end{equation*}
		Interestingly, if we assume that expectation and integration can be interchanged without bothering to check the conditions of Fubini's theorem for interchanging the expectation integral and the integral over $t$, we do get the same result:
		\begin{equation*}
			E[\I_t] = E\left[\int_0^t N(u)\dee u\right] = \int_0^tE[N(u)]\dee u = \int_0^t \lambda u\dee u = \frac{\lambda t^2}{2}.
		\end{equation*}
	\end{alphaparts}
	
	\question 
	\begin{alphaparts}
		\questionpart The following Python code simulates the event times in a Poisson process with rate $2$:
		
		\lstinputlisting[language=Python, numbers=left, frame=single, basicstyle=\small\ttfamily, showstringspaces=false]{question_4a.py}
		
		Here is a sample path:
		\begin{lstlisting}[frame=single, basicstyle=\small\ttfamily]
Event 1 occurred at time 1.3128430911107525
Event 2 occurred at time 2.3143793553663157
Event 3 occurred at time 2.386534691512221
Event 4 occurred at time 2.902762158742764
Event 5 occurred at time 3.6617031194931187
Event 6 occurred at time 3.9372583178630847
Event 7 occurred at time 4.411532868868305
Event 8 occurred at time 4.940204397991703
Event 9 occurred at time 5.647474128105565
Event 10 occurred at time 5.921486938799033
Event 11 occurred at time 6.012072247183725
Event 12 occurred at time 6.180923158778537
Event 13 occurred at time 6.392788380156063
Event 14 occurred at time 7.94812004510943
Event 15 occurred at time 7.996651539617102
Event 16 occurred at time 8.196842927277718
Event 17 occurred at time 9.061571652189476
Event 18 occurred at time 9.532657239086602
Event 19 occurred at time 11.724406573415784
Event 20 occurred at time 11.903626282914876
		\end{lstlisting}
		
		\questionpart The following Python code simulates a Poisson process with rate $2$ up to time $9$ and reports $N(9)$:
		
		\lstinputlisting[language=Python, numbers=left, frame=single, basicstyle=\small\ttfamily, showstringspaces=false]{question_4b.py}
		
		Here are some sample outputs from running a few times.
		\begin{lstlisting}[frame=single, basicstyle=\small\ttfamily]
N(9) = 13
N(9) = 26
N(9) = 21
N(9) = 24
N(9) = 17
		\end{lstlisting}
	\end{alphaparts}
	
	\question*{Textbook: 9, p. 364}
	Let $X_i$ be the time from starting until failure for machine $i$. We want $P(X_1 < X_2 + t)$. Since the variables are independent, their joint distribution is
	\begin{equation*}
		f(x_1, x_2) = \lambda_1\lambda_2e^{-\lambda_1x_1 -\lambda_2x_2}I(x_1 \ge 0, x_2\ge 0).
	\end{equation*}
	Then
	\begin{align*}
		P(X_1 < X_2 + t) &= \int_0^\infty \int_0^{x_2 + t} f(x_1, x_2)\dee x_1 \dee x_2 \\
		&= \int_0^\infty \lambda_2 e^{-\lambda_2x_2}\int_0^{x_2+t} \lambda_1 e^{-\lambda_1x_1}\dee x_1 \dee x_2 \\
		&= \int_0^\infty \lambda_2 e^{-\lambda_2x_2} \left(1 - e^{-\lambda_1(x_2+t)}\right)\dee x_2 \\
		&= 1- e^{-\lambda_1t}\int_0^\infty \lambda_2e^{-(\lambda_1+\lambda_2)x_2}\dee x_2 \\
		&= 1 - \frac{\lambda_2 e^{-\lambda_1t}}{\lambda_1 + \lambda_2}.
	\end{align*}
	
	\question*{Textbook: 15, p. 365}
	Let $X_n$ be the lifetime of item $n$, for $n =1,2,\dots, 100$. Then $T = X_{(5)}$, the fifth order statistic. The p.d.f. of $X_{(5)}$ is
	\begin{equation*}
		f_5(t) = \frac{100!}{4!\cdot 95!}(F(t))^4(1-F(t))^95f(t),
	\end{equation*}
	where $F$ is the CDF of each $X_n$, and $f =F'$ is the corresponding PDF. In our case, $F(t) = 1-e^{-\lambda t}$ for $t \ge 0$, and $f'(t) = \lambda e^{-\lambda t}$ for $t \ge 0$, where $\lambda = \frac{1}{200}$.
	Then
	\begin{equation*}
		f_5(t) = \lambda \frac{100!}{4!\cdot 95!}\left(1-e^{-\lambda t}\right)^4  e^{-96\lambda t} I(t\ge 0).
	\end{equation*}
	It follows that
	\begin{align*}
		E[T] &= E[X_{(5)}] = \lambda\frac{100!}{4!\cdot 95!}\int_0^\infty t\left(1-e^{-\lambda t}\right)^4  e^{-96\lambda t}\dee t \\
		&= \lambda \frac{100!}{4!\cdot 95!}\int_0^\infty\left[ te^{-96\lambda t}- 4te^{-97\lambda t} + 6te^{-98\lambda t} - 4 te^{-99\lambda t} + te^{-100\lambda t}\right]\dee t\\
		&= \lambda \frac{100!}{4!\cdot 95!}\left[\frac{1}{(96\lambda)^2} - \frac{4}{(97\lambda)^2} + \frac{6}{(98\lambda)^2} - \frac{4}{(99\lambda)^2} + \frac{1}{(100\lambda)^2}\right]\\
		&\approx 10.2062.
	\end{align*}
	Likewise,
	\begin{align*}
		E[T^2] &= E[X_{(5)}^2] = \lambda\frac{100!}{4!\cdot 95!}\int_0^\infty t^2\left(1-e^{-\lambda t}\right)^4  e^{-96\lambda t}\dee t \\
		&= \lambda \frac{100!}{4!\cdot 95!}\int_0^\infty\left[ t^2e^{-96\lambda t}- 4t^2e^{-97\lambda t} + 6t^2e^{-98\lambda t} - 4 t^2e^{-99\lambda t} + t^2e^{-100\lambda t}\right]\dee t\\
		&= \lambda \frac{100!}{4!\cdot 95!}\left[\frac{2}{(96\lambda)^3} - \frac{8}{(97\lambda)^3} + \frac{12}{(98\lambda)^3} - \frac{8}{(99\lambda)^3} + \frac{2}{(100\lambda)^3}\right]\\
		&\approx 125.0043.
	\end{align*}
	Thus,
	\begin{equation*}
		\Var[T] = E[T^2] - E[T]^2 \approx 20.8377.
	\end{equation*}
	
	\question*{Textbook: 21, p. 367}
	Let $X_i$ be the service times for the other customer, for $i=1,2$, and let $Y_i$ the my service times for $i=1,2$. Then the total time that I am in the system is $Y_1 + Y_2$ for my own service, plus $X_1$ while I wait for server 1 to be available, plus $W_2 = \min\{X_2 - Y_1, 0\}$, the amount of time I spend waiting for service 2 to be available. $X_2$ and $Y_2$ are exponential with rate $\mu_2$, and $Y_1$ is exponential with rate $\mu_1$. Since the amount of time the first customer will wait in total for service 1 is exponential, the memoryless property of exponential distributions means that the first customer's time remaining from when I enter the system is still exponential with rate $\mu_1$. Thus, my expected total time in the system is
	\begin{equation*}
		E[Y_1 + Y_2 + X_1 + W_2] = \frac{2}{\mu_1} + \frac{1}{\mu_2} + E[W_2].
	\end{equation*}
	Now we need to find $E[W_2]$. Since $X_2$ and $Y_1$ are independent, their joint PDF is
	\begin{equation*}
		f(x_2, y_1) = \mu_2\mu_1 e^{-\mu_2x_2-\mu_1y_1}.
	\end{equation*}
	Then
	\begin{align*}
		E[W_2] &= \int_0^\infty \int_0^\infty \min\{x_2 - y_1, 0\}f(x_2,y_1)\dee x_2 \dee y_1 \\
		&= \int_0^\infty \mu_1e^{-\mu_1y_1}\int_{y_1}^\infty \mu_2e^{-\mu_2x_2} \dee x_2 \dee y_1 \\
		&= \int_0^\infty \mu_1e^{-\mu_1y_1}e^{-\mu_2y_1}\dee y_1 \\
		&= \frac{\mu_1}{\mu_1 + \mu_2}.
	\end{align*}
	This gives a total expected time in the system of
	\begin{equation*}
		E[Y_1 + Y_2 + X_1 + W_2] = \frac{2}{\mu_1} + \frac{1}{\mu_2} + \frac{\mu_1}{\mu_1 + \mu_2}.
	\end{equation*}
	
	\question*{Textbook: 41, p. 371}
	\begin{alphaparts}
		\questionpart Using the independent and identically-distributed increments property:
		\begin{align*}
			P(N(4) = 4 \mid N(3) = 1) &= P(N(4) - N(3) = 3 \mid N(3) - N(0) = 1) \\
			&= P(N(4) - N(3) = 3) \\
			&= P(N(1) = 3) = \frac{\lambda^3e^{-\lambda}}{3!}.
		\end{align*}
		
		\questionpart Using the independent and identically-distributed increments property:
		\begin{align*}
			\Var[N(8) \mid N(5)=6] &= \Var[N(8) - N(5) + N(5) \mid N(5)=6] \\
			&= \Var[N(8)-N(5) \mid N(5)-N(0)=6] + \Var[N(5) \mid N(5) = 6] \\
			&= \Var[N(8) - N(5)] + 0 \\
			&= \Var[N(3)] = 3\lambda.
		\end{align*}
		
		\questionpart Using the independent and identically-distributed increments property as well as the fact that $N(2)$ is distributed $\mathrm{Binom}(4, 3/5)$ given $N(5)=4$:
		\begin{alignat*}{2}
			P(N(5) =0 \mid N(8) - N(3) = 4) &={}&& P(N(5) - N(3) =0 \mid N(8) - N(3) = 4) \\
			&&&{}\times P(N(3) - N(0) = 0 \mid N(8) - N(3) = 4) \\
			&={}&& P(N(5) - N(3) = 0 \mid N(8) - N(3) = 4)P(N(3) = 0) \\
			&={}&& P(N(2) = 0 \mid N(5) = 4)P(N(3) = 0) \\
			&={}&& \left(\frac{3}{5}\right)^4e^{-3\lambda}
		\end{alignat*}
	\end{alphaparts}
	
	\question*{Textbook: 44, p. 371}
	\begin{alphaparts}
		\questionpart 
		
		\questionpart
	\end{alphaparts}
	
	\question*{Textbook: 50, p. 372}
	\begin{alphaparts}
		\questionpart 
		
		\questionpart
	\end{alphaparts}
\end{document}