\documentclass{homework}
\newcommand{\R}{\textbf{R}}
\newcommand{\dee}{\;\text{d}}
\newcommand{\eps}{\varepsilon}
\newcommand{\pl}[2]{\frac{\partial #1}{\partial #2}}
\newcommand{\dl}[2]{\frac{\text{d} #1}{\text{d} #2}}
\newcommand{\sgn}{\text{sgn}}
\newcommand{\bigoh}{\mathcal{O}}
\usepackage{enumitem}

\newcommand{\hwclass}{Math 6108}
\newcommand{\hwname}{Jacob Hauck}
\newcommand{\hwtype}{Homework}


\renewcommand{\questiontype}{}

\newcommand{\hwnum}{2}

\begin{document}
	\maketitle
	
	Let
	\begin{equation}
		f(x) = \begin{cases}
			1 & x \in [-1, 1], \\
			0 & |x| > 1.
		\end{cases}
	\end{equation}
	
	\question Since $f \in H^s(\R)$ if and only if $g_s \in L^2(\R)$, where $g_s(\xi) = \widehat{f}(\xi)(1+|\xi|^2)^\frac{s}{2}$, we should start by computing $\widehat{f}$:
	\begin{align*}
		\widehat{f}(\xi) &= \frac{1}{\sqrt{2\pi}} \int_{-\infty}^\infty e^{-ix\xi}f(x)\dee x \\
		&= \frac{1}{\sqrt{2\pi}}\int_{-1}^1 e^{-ix\xi}\dee x \\
		&= \frac{1}{\sqrt{2\pi}}\left[\frac{e^{-ix\xi}}{-i\xi}\right]_{-1}^1 \\
		&= \frac{e^{i\xi} - e^{-i\xi}}{i\xi\sqrt{2\pi}}\\
		&= \sqrt{\frac{2}{\pi}}\frac{\sin(\xi)}{\xi}.
	\end{align*}
	We note that $\widehat{f}(0) = \sqrt{\frac{2}{\pi}}$, so $\widehat{f}$ is continuous. Thus,
	\begin{equation*}
		g_s(\xi) =\sqrt{\frac{2}{\pi}}\frac{\sin(\xi)}{\xi}(1+|\xi|^2)^\frac{s}{2}
	\end{equation*}
	is also continuous.
	
	Therefore, for $a > 0$,
	\begin{equation*}
		\int_{-\infty}^\infty |g_s(\xi)|\dee\xi = \int_{-a}^a |g_s(\xi)|^2\dee\xi + 2\int_{a}^\infty |g_s(\xi)|^2\dee \xi
	\end{equation*}
	because $g_s$ is even. The first term is always finite because $g_s$ is continuous, so $g_s \in L^2(\R)$ if and only if the second term is finite.
	
	Since $\frac{(1+|\xi|^2)^s}{|\xi|^{2s}} = \left(\frac{1}{|\xi|^2} + 1\right)^s \to 1$ as $\xi \to \infty$, we can choose $a$ large enough that $(1+|\xi|^2)^s \le 2|\xi|^{2s}$ for all $\xi > a$. Then $|g_s(\xi)|^2 \le \frac{4}{\pi}|\xi|^{2(s-1)}$ for all $\xi > a$, meaning that the second term is finite if $2(s-1) < -1$, that is, if $s < \frac{1}{2}$. Hence, if $s < \frac{1}{2}$, then $f \in H^s$.
	
	Conversely, suppose that $s \ge \frac{1}{2}$. Since $(1+|\xi|^2)^s \ge |\xi|^{2s}$, it follows that
	\begin{equation*}
		\int_\frac{\pi}{4}^\infty |g_s(\xi)|^2 \dee\xi \ge \frac{2}{\pi}\int_\frac{\pi}{4}^\infty \sin^2(\xi)|\xi|^{2s-2}\dee \xi.
	\end{equation*}
	If $|\xi|^{2s-2}$ is increasing, then it is clear that the integral on the right hand side is infinite, which implies that $f \notin H^s$. Suppose that $|\xi|^{2s-2}$ is decreasing, and set $I_1 = \left[\frac{\pi}{4}, \frac{3\pi}{4}\right] \cup \left[\frac{5\pi}{4}, \frac{7\pi}{4}\right] \cup \dots$ and set $I_2 = \left[\frac{\pi}{4}, \infty\right) \setminus I_1$. The sets $I_1$ and $I_2$ consist of consecutive, interleaved intervals of length $\frac{\pi}{2}$; since $|\xi|^{2s-2}$ is decreasing and each interval of $I_1$ precedes an interval of $I_2$, it follows that
	\begin{equation*}
		\int_{I_1} |\xi|^{2s-2}\dee\xi \ge \int_{I_2}|\xi|^{2s-2}\dee\xi,
	\end{equation*}
	so
	\begin{equation*}
		\int_\frac{\pi}{4}^\infty |\xi|^{2s-2} = \int_{I_1}|\xi|^{2s-2}\dee\xi + \int_{I_2}|\xi|^{2s-2}\dee\xi \le 2\int_{I_1} |\xi|^{2s-2}\dee\xi,
	\end{equation*}
	which implies that
	\begin{equation*}
		\int_{I_1}|\xi|^{2s-2}\dee\xi \ge \frac{1}{2}\int_{\frac{\pi}{4}}^\infty |\xi|^{2s-2}\dee\xi = \infty
	\end{equation*}
	because $s \ge \frac{1}{2}$. Therefore,
	\begin{equation*}
		\int_\frac{\pi}{4}^\infty \sin^2(\xi)|\xi|^{2s-2} \dee\xi \ge \int_{I_1}\sin^2(\xi)|\xi|^{2s-2}\dee\xi \ge \frac{1}{2}\int_{I_1}|\xi|^{2s-2}\dee\xi = \infty
	\end{equation*}
	because $\sin^2(\xi) \ge \frac{1}{2}$ for $\xi \in I_1$. This implies that $f \notin H^s$.
	
	Thus, $f\in H^s$ if and only if $s < \frac{1}{2}$.
	
	\question
	
	Let $g \in L^1(\R)$ be a positive function such that $g(1+|\cdot|^2)^\frac{1}{s} \in L^2(\R)$ for all $s < 1$. For example, $g(x) = e^{-x^2}$ works. Define 
	\begin{equation*}
		v(\xi, \eta) = \frac{1}{\lVert g\rVert_{L^1}}\frac{2\sin(\xi)}{\xi\sqrt{1+\xi^2}}\cdot g\left(\frac{\eta}{\sqrt{1+\xi^2}}\right).
	\end{equation*}
	Then
	\begin{equation*}
		\frac{1}{\sqrt{2\pi}}\int_{-\infty}^\infty v(\xi, \eta)\dee\eta = \frac{1}{\lVert g\rVert_{L^1}}\sqrt{\frac{2}{\pi}}\frac{\sin(\xi)}{\xi\sqrt{1+\xi^2}}\int_{-\infty}^\infty
		 g\left(\frac{\eta}{\sqrt{1+\xi^2}}\right)\dee \eta = \sqrt{\frac{2}{\pi}}\frac{\sin(\xi)}{\xi} = \widehat{f}(\xi),
	\end{equation*}
	upon making the substitution $w = \frac{\eta}{\sqrt{1+\xi^2}}$. Furthermore, for $s \in \R$,
	\begin{equation*}
		\lVert v (1 + |\cdot|^2)^\frac{s}{2}\rVert_{L^2}^2 = \frac{4}{\lVert g\rVert_{L^1}^2}\int_{-\infty}^\infty\int_{-\infty}^\infty\frac{\sin^2(\xi)}{\xi^2(1+\xi^2)}g^2\left(\frac{\eta}{\sqrt{1+\xi^2}}\right)(1+\xi^2+\eta^2)^s\dee\eta\dee\xi.
	\end{equation*}
	Substituting $w = \frac{\eta}{\sqrt{1+\xi^2}}$ in the inner integral over $\eta$, we have
	\begin{align*}
		\lVert v (1 + |\cdot|^2)^\frac{s}{2}\rVert_{L^2}^2 &= \frac{4}{\lVert g\rVert_{L^1}^2}\int_{-\infty}^\infty\int_{-\infty}^\infty\frac{\sin^2(\xi)(1+\xi^2)^s}{\xi^2\sqrt{1+\xi^2}}g^2(w)(1+w^2)^s\dee w\dee\xi \\
		&= \frac{4\lVert g(1 + (\cdot)^2)^\frac{s}{2}\rVert_{L^2}^2}{\lVert g\rVert_{L^1}^2}\int_{-\infty}^\infty \frac{\sin^2(\xi)(1+\xi^2)^{s-\frac{1}{2}}}{\xi^2}\dee \xi.
	\end{align*}
	The remaining integral is finite if $2s-1-2 < -1$, that is, if $s < 1$. Choosing $s = 0$, we see that $v \in L^2(\R^2)$, so there exists $u \in L^2(\R^2)$ such that $\widehat{u} = v$ by Fourier inversion; therefore, $u \in H^s(\R^2)$ for all $s < 1$. Restricting to the upper half plane, we can also say that $u \in H^s(\R_+^2)$ for all $s < 1$.
	
	Finally, $u(x,0) = f(x)$ because
	\begin{equation*}
		u(x,0) = \frac{1}{2\pi}\int_{-\infty}^\infty\int_{-\infty}^\infty e^{i\xi x} \widehat{u}(\xi,\eta)\dee\eta\dee\xi = \mathcal{F}^{-1}\left(\frac{1}{\sqrt{2\pi}}\int_{-\infty}^\infty v(\xi,\eta)\dee\eta\right)(x) = \mathcal{F}^{-1}(\hat{f})(x) = f(x).
	\end{equation*}
	
	\question
	
	We note that the same $u$ in part 2 works for all $s < 1$.
	
	\question
	
	If we choose $g(x) = e^{-x^2}$, then $u$ from part 2 is $C^\infty$ on $\R_+^2$. Indeed, for $y>0$, we have
	\begin{equation*}
		u(x,y) = \frac{1}{\pi\sqrt{2\pi}}\int_{-\infty}^\infty\int_{-\infty}^\infty e^{i\xi x}e^{i\eta y}\frac{\sin(\xi)}{\xi\sqrt{1+\xi^2}}e^{-\left(\frac{\eta}{\sqrt{1+\xi^2}}\right)^2}\dee\eta\dee\xi.
	\end{equation*}
	Doing the $\eta$ integral first, we have
	\begin{align*}
		u(x,y) &= \frac{1}{\pi}\int_{-\infty}^\infty e^{ix\xi}\frac{\sin(\xi)}{\xi\sqrt{1+\xi^2}} \frac{1}{\sqrt{2\pi}}\int_{-\infty}^\infty e^{i\eta y}e^{-\left(\frac{\eta}{\sqrt{1+\xi^2}}\right)}\dee\eta\dee\xi \\
		&= \frac{1}{\pi}\int_{-\infty}^\infty e^{ix\xi}\frac{\sin(\xi)}{\xi\sqrt{1+\xi^2}}\fourier^{-1}\left[e^{-\frac{(\cdot)^2}{1+\xi^2}}\right](y)\dee\xi.
	\end{align*}
	Consulting a table of Fourier transforms, we find that
	\begin{equation*}
		u(x,y) = \frac{1}{\pi\sqrt{2}}\int_{-\infty}^\infty e^{ix\xi}\frac{\sin(\xi)}{\xi}e^{-\frac{y^2(1+\xi^2)}{4}}\dee \xi.
	\end{equation*}
	Let $\alpha$ be a multi-index. Then, formally,
	\begin{equation*}
		D^{\alpha}u(x,y) = \frac{1}{\pi\sqrt{2}}\int_{-\infty}^\infty e^{ix\xi}\frac{\sin(\xi)}{\xi}P_\alpha(x,y,\xi)e^{-\frac{y^2(1+\xi^2)}{4}}\dee \xi,
	\end{equation*}
	where $P_\alpha(x,y,\xi)$ is a polynomial function in $x,y,\xi$. Taking the partial derivatives inside the integral is valid, in fact, because the integrand is $C^\infty$ and has exponential decay in $\xi$, \textit{provided that $y > 0$}, which we are assuming. Since this is true of the integrand no matter how many partial derivatives are taken, it follows that $D^\alpha u$ exists and is continuous for all $\alpha$, if $y > 0$. That is, $u \in C^\infty(\R_+^2)$.
\end{document}