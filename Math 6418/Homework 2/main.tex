\documentclass{homework}
\newcommand{\R}{\textbf{R}}
\newcommand{\dee}{\;\text{d}}
\usepackage{enumitem}

\newcommand{\hwclass}{Math 6418}
\newcommand{\hwname}{Jacob Hauck}
\newcommand{\hwtype}{Homework}

\newcommand{\dist}{\mathcal{D}}

\renewcommand{\questiontype}{}

\newcommand{\hwnum}{2}

\begin{document}
	\maketitle
	
	Let
	\begin{equation}
		f(x) = \begin{cases}
			1 & x \in [-1, 1], \\
			0 & |x| > 1.
		\end{cases}
	\end{equation}
	
	\question Since $f \in H^s(\R)$ if and only if $g_s \in L^2(\R)$, where $g_s(\xi) = \widehat{f}(\xi)(1+|\xi|^2)^\frac{s}{2}$, we should start by computing $\widehat{f}$:
	\begin{align*}
		\widehat{f}(\xi) &= \frac{1}{\sqrt{2\pi}} \int_{-\infty}^\infty e^{-ix\xi}f(x)\dee x \\
		&= \frac{1}{\sqrt{2\pi}}\int_{-1}^1 e^{-ix\xi}\dee x \\
		&= \frac{1}{\sqrt{2\pi}}\left[\frac{e^{-ix\xi}}{-i\xi}\right]_{-1}^1 \\
		&= \frac{e^{i\xi} - e^{-i\xi}}{i\xi\sqrt{2\pi}}\\
		&= \sqrt{\frac{2}{\pi}}\frac{\sin(\xi)}{\xi}.
	\end{align*}
	We note that $\widehat{f}(0) = \sqrt{\frac{2}{\pi}}$, so $\widehat{f}$ is continuous. Thus,
	\begin{equation*}
		g_s(\xi) =\sqrt{\frac{2}{\pi}}\frac{\sin(\xi)}{\xi}(1+|\xi|^2)^\frac{s}{2}
	\end{equation*}
	is also continuous.
	
	Therefore, for $a > 0$,
	\begin{equation*}
		\int_{-\infty}^\infty |g_s(\xi)|\dee\xi = \int_{-a}^a |g_s(\xi)|^2\dee\xi + 2\int_{a}^\infty |g_s(\xi)|^2\dee \xi
	\end{equation*}
	because $g_s$ is even. The first term is always finite because $g_s$ is continuous, so $g_s \in L^2(\R)$ if and only if the second term is finite.
	
	Since $\frac{(1+|\xi|^2)^s}{|\xi|^{2s}} = \left(\frac{1}{|\xi|^2} + 1\right)^s \to 1$ as $\xi \to \infty$, we can choose $a$ large enough that $(1+|\xi|^2)^s \le 2|\xi|^{2s}$ for all $\xi > a$. Then $|g_s(\xi)|^2 \le \frac{4}{\pi}|\xi|^{2(s-1)}$ for all $\xi > a$, meaning that the second term is finite if $2(s-1) < -1$, that is, if $s < \frac{1}{2}$. Hence, if $s < \frac{1}{2}$, then $f \in H^s$.
	
	Conversely, suppose that $s \ge \frac{1}{2}$. Since $(1+|\xi|^2)^s \ge |\xi|^{2s}$, it follows that
	\begin{equation*}
		\int_\frac{\pi}{4}^\infty |g_s(\xi)|^2 \dee\xi \ge \frac{2}{\pi}\int_\frac{\pi}{4}^\infty \sin^2(\xi)|\xi|^{2s-2}\dee \xi.
	\end{equation*}
	If $|\xi|^{2s-2}$ is increasing, then it is clear that the integral on the right hand side is infinite, which implies that $f \notin H^s$. Suppose that $|\xi|^{2s-2}$ is decreasing, and set $I_1 = \left[\frac{\pi}{4}, \frac{3\pi}{4}\right] \cup \left[\frac{5\pi}{4}, \frac{7\pi}{4}\right] \cup \dots$ and set $I_2 = \left[\frac{\pi}{4}, \infty\right) \setminus I_1$. The sets $I_1$ and $I_2$ consist of consecutive, interleaved intervals of length $\frac{\pi}{2}$; since $|\xi|^{2s-2}$ is decreasing and each interval of $I_1$ precedes an interval of $I_2$, it follows that
	\begin{equation*}
		\int_{I_1} |\xi|^{2s-2}\dee\xi \ge \int_{I_2}|\xi|^{2s-2}\dee\xi,
	\end{equation*}
	so
	\begin{equation*}
		\int_\frac{\pi}{4}^\infty |\xi|^{2s-2} = \int_{I_1}|\xi|^{2s-2}\dee\xi + \int_{I_2}|\xi|^{2s-2}\dee\xi \le 2\int_{I_1} |\xi|^{2s-2}\dee\xi,
	\end{equation*}
	which implies that
	\begin{equation*}
		\int_{I_1}|\xi|^{2s-2}\dee\xi \ge \frac{1}{2}\int_{\frac{\pi}{4}} |\xi|^{2s-2}\dee\xi = \infty
	\end{equation*}
	because $s \ge \frac{1}{2}$. Therefore,
	\begin{equation*}
		\int_\frac{\pi}{4}^\infty \sin^2(\xi)|\xi|^{2s-2} \dee\xi \ge \int_{I_1}\sin^2(\xi)|\xi|^{2s-2}\dee\xi \ge \frac{1}{2}\int_{I_1}|\xi|^{2s-2}\dee\xi = \infty
	\end{equation*}
	because $\sin^2(\xi) \ge \frac{1}{2}$ for $\xi \in I_1$. This implies that $f \notin H^s$.
	
	Thus, $f\in H^s$ if and only if $s < \frac{1}{2}$.
		
	
	
\end{document}