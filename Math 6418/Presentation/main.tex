\documentclass[]{beamer}

\usepackage{etoolbox}
\usepackage{xcolor}
\newtoggle{handout}
\togglefalse{handout}

\iftoggle{handout}{
	\usepackage{pgfpages}
	\pgfpagesuselayout{4 on 1}[border shrink=5mm]
}{}

\setbeamertemplate{footline}
{
	\hbox{\begin{beamercolorbox}[wd=1\paperwidth,ht=2.25ex,dp=1ex,right]{framenumber}%
			\usebeamerfont{framenumber}\insertframenumber{} / \inserttotalframenumber\hspace*{2ex}
	\end{beamercolorbox}}%
	\vskip0pt%
}

\definecolor{forest}{HTML}{154734}
\definecolor{miner}{HTML}{007A33}
\definecolor{lima}{HTML}{72BF44}
\definecolor{kiwi}{HTML}{BFD730}
\definecolor{mist}{HTML}{B1E4E3}

\setbeamertemplate{blocks}[rounded][shadow=true]
\setbeamertemplate{navigation symbols}{}
\setbeamercolor{frametitle}{fg=forest}
\setbeamercolor{structure}{fg=forest}
\setbeamercolor{block title}{bg=forest,fg=kiwi}
\setbeamercolor{block body}{bg=lima}

\usefonttheme{serif}

\newcommand{\R}{\textbf{R}}
\newcommand{\dee}{\;\text{d}}
\newcommand{\eps}{\varepsilon}
\newcommand{\pl}[2]{\frac{\partial #1}{\partial #2}}
\newcommand{\dl}[2]{\frac{\text{d} #1}{\text{d} #2}}
\newcommand{\sgn}{\text{sgn}}
\newcommand{\bigoh}{\mathcal{O}}


\title{The Fréchet Derivative}
\author{Jacob Hauck}
\institute{Math 6418}
\date{}

\begin{document}
	\frame{\titlepage}
	
	\begin{frame}{Motivation}
		Let $X,Y$ be normed linear spaces. We know a lot about linear maps $A : X \to Y$ -- what can we do about about nonlinear maps?
		\vfill
		\pause
		
		Approximate a nonlinear map $f : U \to Y$ by something linear:
		\pause
		
		given $x_0 \in U$, find a linear operator $A_{x_0} \in B(X,Y)$ such that
		\begin{equation*}
			f(x) - f(x_0) \approx A_{x_0}(x-x_0)
		\end{equation*}
		for $x$ close to $x_0$ (point-slope form)
		\vfill
		\pause
	\end{frame}
	
	\begin{frame}{Motivation}
		Say $X = Y = \R$, then $A_{x_0}$ is given by multiplication by a number $a_{x_0} \in \R$, and a natural choice for $a_{x_0}$ is $f'(x_0)$, as
		\begin{equation*}
			f(x) - f(x_0) \approx f'(x_0)(x-x_0).
		\end{equation*}
		\vfill
		\pause
		
		By definition,
		\begin{equation*}
			\frac{f(x_0+h) - f(x_0) - f'(x_0)h}{h} \to 0 \quad\text{as}\quad h\to 0,
		\end{equation*}
		\pause
		so
		\begin{equation*}
			f(x_0+h) - f(x_0) = f'(x_0)h + \omega(h)h,
		\end{equation*}
		where $\omega(h) \to 0$ as $h \to 0$.
	\end{frame}
	
	\begin{frame}{Definition of the Fréchet Derivative}
		\begin{block}{Definition (Fréchet Derivative)}
		A function $f : U \to Y$ is called \textbf{Fréchet differentiable} at $x \in U$ if there exists $A \in B(X,Y)$ such that
		\begin{equation*}
			\frac{\lVert f(x+h) - f(x) - Ah \rVert_Y}{\lVert h \rVert_X} \to 0 \quad \text{as} \quad \lVert h\rVert_X\to 0,
		\end{equation*}
		in which case $A$ is called the \textbf{Fréchet derivative} of $f$ at $x$, also denoted by
		\begin{equation*}
			A = f'(x) = Df(x)
		\end{equation*}
		\end{block}
		\vfill
		\pause
		
		The Fréchet derivative is the same as the usual derivative if $f \in C^1(\R)$.
	\end{frame}
\end{document}