\documentclass{homework}
\newcommand{\R}{\textbf{R}}
\newcommand{\dee}{\;\text{d}}
\newcommand{\eps}{\varepsilon}
\newcommand{\pl}[2]{\frac{\partial #1}{\partial #2}}
\newcommand{\dl}[2]{\frac{\text{d} #1}{\text{d} #2}}
\newcommand{\sgn}{\text{sgn}}
\newcommand{\bigoh}{\mathcal{O}}
\usepackage{enumitem}

\newcommand{\hwclass}{Math 6108}
\newcommand{\hwname}{Jacob Hauck}
\newcommand{\hwtype}{Homework}


\renewcommand{\questiontype}{}

\newcommand{\hwnum}{1}

\begin{document}
	\maketitle
	
	Define
	\begin{equation}
		k_\varepsilon(x) = \frac{1}{\varepsilon}\chi_{\left[-\frac{\varepsilon}{2}, \frac{\varepsilon}{2}\right]}(x).
	\end{equation}
	
	\question For $u \in \dist'(\R)$, define $u * k_\varepsilon \in \dist'(\R)$ by
	\begin{equation}
		\label{eq:convolution_defn}
		\langle u * k_\varepsilon, \varphi \rangle = \frac{1}{\varepsilon}\int_{-\frac{\varepsilon}{2}}^\frac{\varepsilon}{2}(u * R\varphi)(y)\dee y, \qquad \varphi \in \dist(\R).
	\end{equation}
	Note that the integral is defined because $u * R\varphi$ is continuous. Furthermore, $u*k_\varepsilon$ is a distribution because it is linear and continuous. 
	
	\textbf{Linearity} 
	
	We can verify linearity easily using the linearity of convolution with a test function, the linearity of integration, and the linearity of reflection. If $\alpha, \beta \in \R$ and $\varphi, \psi \in \dist(\R)$, then
	\begin{align}
		\langle u * k_\varepsilon, \alpha \varphi + \beta \psi \rangle &= \frac{1}{\varepsilon}\int_{-\frac{\varepsilon}{2}}^\frac{\varepsilon}{2} (u * R(\alpha\varphi + \beta\psi))(y)\dee y \\
		&= \frac{1}{\varepsilon}\int_{-\frac{\varepsilon}{2}}^\frac{\varepsilon}{2} (u * (\alpha R\varphi + \beta R\psi))(y)\dee y \\
		&= \frac{1}{\varepsilon}\int_{-\frac{\varepsilon}{2}}^\frac{\varepsilon}{2}\big[ \alpha (u * R\varphi)(y) + \beta (u * R\psi)(y)\big] \dee y \\
		&= \alpha \frac{1}{\varepsilon}\int_{-\frac{\varepsilon}{2}}^\frac{\varepsilon}{2} (u * R\varphi)(y)\dee y + \beta \frac{1}{\varepsilon}\int_{-\frac{\varepsilon}{2}}^\frac{\varepsilon}{2}(u*R\psi)(y)\dee y \\
		&= \alpha\langle u * k_\varepsilon, \varphi\rangle + \beta\langle u*k_\varepsilon, \psi\rangle,
	\end{align}
	so $u * k_\varepsilon$ is linear.
	
	\textbf{Continuity}
	
	Let $\varphi_n \to \varphi$ in $\dist(\R)$. Then clearly $R\varphi_n \to R\varphi$ in $\dist(\R)$, and by the continuity of $u$, $u * R\varphi_n \to u * R\varphi$ pointwise. If $\{u*R\varphi_n\}_{n=1}^\infty$ is uniformly bounded on $\left[-\frac{\varepsilon}{2},\frac{\varepsilon}{2}\right]$, then the bounded convergence theorem implies that
	\begin{equation}
		\frac{1}{\varepsilon}\int_{-\frac{\varepsilon}{2}}^\frac{\varepsilon}{2}(u*R\varphi_n)(y)\dee y \to 
		\frac{1}{\varepsilon}\int_{-\frac{\varepsilon}{2}}^\frac{\varepsilon}{2}(u*R\varphi)(y)\dee y;
	\end{equation}
	that is, $\langle u * k_\varepsilon, \varphi_n\rangle \to \langle u*k_\varepsilon, \varphi\rangle$. Thus, $u*k_\varepsilon$ is continuous.
	
	To show that $\{u*R\varphi_n\}$ is uniformly bounded on $\left[-\frac{\varepsilon}{2}, \frac{\varepsilon}{2}\right]$, suppose for the sake of contradiction that it is not. Then for all $m > 0$, there exists $\varphi_{n_m}$ and $x_m \in \left[-\frac{\varepsilon}{2}, \frac{\varepsilon}{2}\right]$ such that
	\begin{equation*}
		m \le |(u*\varphi_{n_m})(x_m)| = |\langle u, \tau_{x_m}R\varphi_{n_m}\rangle|.
	\end{equation*}
	There is a convergent subsequence $\{x_{m_k}\}_{k=1}^\infty$ such that $x_{m_k} \to x \in \left[-\frac{\varepsilon}{2}, \frac{\varepsilon}{2}\right]$. The convergence of $\{x_{m_k}\}$ and the convergence of $\{\varphi_{n_{m_k}}\}$ in $\dist(\R)$ ensure that $\tau_{x_{m_k}}R\varphi_{n_{m_k}} \to \tau_xR\varphi$ in $\dist(\R)$ as $k \to \infty$. Then, by the continuity of $u$,
	\begin{equation*}
		\infty = \lim_{k\to\infty}|\langle u, \tau_{x_{m_k}}R\varphi_{n_{m_k}}\rangle| = |\langle u, \tau_xR\varphi\rangle| < \infty,
	\end{equation*}
	which is a contradiction.
		
	\textbf{Extension}
	
	Definition (\ref{eq:convolution_defn}) is a good definition of convolution at least in the sense that it reduces to convolution with $k_\varepsilon$ for regular distributions. Indeed, suppose that $f \in L^1_\text{loc}(\R)$. Then for any $\varphi \in \dist(\R)$,
	\begin{align}
		\langle f * k_\varepsilon, \varphi\rangle &= \int_{-\infty}^\infty (f * k_\varepsilon)(x) \varphi(x) \dee x \\
		&= \int_{-\infty}^\infty \int_{-\infty}^\infty f(y) k_\varepsilon(x-y) \varphi(x)\dee y \dee x \\
		&= \frac{1}{\varepsilon}\int_{-\infty}^\infty \int_{x-\frac{\varepsilon}{2}}^{x+\frac{\varepsilon}{2}} f(y)\varphi(x)\dee y \dee x.
	\end{align}
	Using the change of variables $y' = y - x$, we get
	\begin{align}
			\langle f * k_\varepsilon, \varphi\rangle &= \frac{1}{\varepsilon}\int_{-\infty}^\infty \int_{-\frac{\varepsilon}{2}}^\frac{\varepsilon}{2} f(y' + x) \varphi(x) \dee y' \dee x \\
			&= \frac{1}{\varepsilon}\int_{-\frac{\varepsilon}{2}}^\frac{\varepsilon}{2} \int_{-\infty}^\infty f(y' + x) \varphi(x) \dee x \dee y'.
	\end{align}
	Using the change of variables $x' = y' + x$, we get
	\begin{align}
		\langle f * k_\varepsilon, \varphi\rangle &= \frac{1}{\varepsilon}\int_{-\frac{\varepsilon}{2}}^\frac{\varepsilon}{2} \int_{-\infty}^\infty f(x') \varphi(x'-y') \dee x' \dee y' \\
		&= \frac{1}{\varepsilon}\int_{-\frac{\varepsilon}{2}}^\frac{\varepsilon}{2} (f * R\varphi)(y')\dee y',
	\end{align}
	which agrees with our definition of $f * k_\varepsilon$ in (\ref{eq:convolution_defn}) if we view $f$ as a distribution.
	
	\question Consider $\delta_0 * k_\varepsilon$ using our definition of convolution from (\ref{eq:convolution_defn}). Since $\delta_0$ is supposed to be the identity for the convolution operator, we expect that $\delta_0 * k_\varepsilon = k_\varepsilon$ (viewing $k_\varepsilon$ as a distribution).
	
	This turns out to be the case. According to the definition in (\ref{eq:convolution_defn}), for any $\varphi \in \dist(\R)$,
	\begin{align}
		\langle\delta_0 * k_\varepsilon, \varphi\rangle &= \frac{1}{\varepsilon}\int_{-\frac{\varepsilon}{2}}^\frac{\varepsilon}{2} (\delta_0 * R\varphi)(y)\dee y &&\\
		&= \frac{1}{\varepsilon}\int_{-\frac{\varepsilon}{2}}^\frac{\varepsilon}{2} R\varphi(y)\dee y  && \text{because $\delta_0$ is identity for convolution}\\
		&= \frac{1}{\varepsilon}\int_{-\frac{\varepsilon}{2}}^\frac{\varepsilon}{2}  \varphi(y)\dee y && \text{using change of variables $y\mapsto -y$}\\
		&= \langle k_\varepsilon, \varphi \rangle.
	\end{align}
	Thus, $\delta_0 * k_\varepsilon = k_\varepsilon$, viewing $k_\varepsilon$ as a distribution.
	
	\question Since $\int k_\varepsilon = 1$ all $\varepsilon$, and $k_\varepsilon(x) \to 0$ as $\varepsilon \to 0$ if $x \ne 0$, it would seem that $k_\varepsilon$ behaves like $\delta_0$ as $\varepsilon \to 0$. Thus, it would make sense that $u * k_\varepsilon \text{``} \to u * \delta_0 = u\text{''}$. That is, it would make sense that $u * k_\varepsilon \to u$ as $\varepsilon \to 0$ in the topology of $\dist'(\R)$.
	
	In fact, this turns out to be the case. Let $\varphi \in \dist(\R)$. Since $u*R\varphi$ is $C^\infty$ and therefore continuous, it has an antiderivative $\psi$. Then
	\begin{align}
		\lim_{\varepsilon \to 0} \langle u * k_\varepsilon, \varphi\rangle &= \lim_{\varepsilon\to 0}\frac{1}{\varepsilon}\int_{-\frac{\varepsilon}{2}}^\frac{\varepsilon}{2} (u * R\varphi)(y)\dee y \\
		&= \lim_{\varepsilon \to 0} \frac{\psi\left(\frac{\varepsilon}{2}\right) - \psi\left(-\frac{\varepsilon}{2}\right)}{\varepsilon} = \psi'(0) \\
		&= (u * R\varphi)(0) = \langle u, \tau_0RR\varphi\rangle \\
		&= \langle u, \varphi \rangle.
	\end{align}
	Hence, $u * k_\varepsilon \to u$ in $\dist'(\R)$ as $\varepsilon \to 0$.
	
	\question For $u, v \in \dist'(\R)$ we are tempted to define $u * v \in \dist'(\R)$ by
	\begin{equation}
		\langle u * v, \varphi \rangle = \langle u, Rv * \varphi\rangle, \qquad \varphi \in \dist(\R).
	\end{equation}
	Unfortunately, although $Rv * \varphi$ is $C^\infty$, it may not be compactly supported. For example, if $Rv = 1$, then $Rv * \varphi = \int\varphi$, which is not compactly supported as long as $\int \varphi \ne 0$. Thus, it may not make sense to take the action of $u$ on $Rv * \varphi$.
	
	Furthermore, this definition would obviously be linear in $\varphi$ by the linearity of convolution and of $u$, but to prove continuity, we would need to prove that $Rv * \varphi_n \to Rv*\varphi$ in $\dist(\R)$ whenever $\varphi_n \to \varphi$ in $\dist(\R)$. So, not only would we need that $Rv*\varphi_n$ is compactly supported, but we would need that each $Rv*\varphi_n$ is supported in the same compact set.
	
	If we could provide some restriction on $v$ so that $Rv$ could allay these concerns, then there would be some hope of defining $u * v$ as above. Alternatively, we might be able to restrict $u$ so that it's action could be extended to a function that is not compactly supported. This is essentially what we did in part 1. with $k_\varepsilon$. Using the fact the $Rk_\varepsilon = k_\varepsilon$, a simple change of variables shows that the proposed definition above and the definition given for convolution with $k_\varepsilon$ are one and the same, assuming that the action of $k_\varepsilon$ is defined by integration (in which case we must use the compact support of $k_\varepsilon$ to deal with the fact that $Rv*\varphi$ might not be compactly supported).
		
	On the bright side, if $f, g\in L^1(\R)$ are viewed as distributions, then $f*g$ as defined above is the same as the usual convolution of $f$ with $g$, which I will denote $f \star g$ to avoid confusion. Indeed, for any $\varphi \in \dist(\R)$,
	\begin{align}
		\langle f * g, \varphi\rangle &= \langle f, Rg * \varphi\rangle \\
		&= \int_{-\infty}^\infty f(x) (Rg \star \varphi)(x) \dee x \qquad \text{($*$ reduces to $\star$ for $Rg$ and $\varphi$)}\\
		&= \int_{-\infty}^\infty\int_{-\infty}^\infty f(x) Rg(y)\varphi(x-y)\dee y \dee x \\
		&= \int_{-\infty}^\infty\int_{-\infty}^\infty f(x)g(-y)\varphi(x-y)\dee y\dee x.
	\end{align}
	If we let $y' = x- y$, then we get
	\begin{align}
		\langle f * g, \varphi\rangle &= \int_{-\infty}^\infty \int_{-\infty}^\infty f(x)g(y'-x)\varphi(y')\dee y' \dee x \\
		&= \int_{-\infty}^\infty \int_{-\infty}^\infty f(x)g(y'-x)\varphi(y')\dee x \dee y' \\
		&= \int_{-\infty}^\infty (f \star g)(y')\varphi(y') \dee y' \\
		&= \langle f\star g, \varphi\rangle.
	\end{align}
	Thus, as distributions, we have $f * g = f\star g$, as claimed.
\end{document}